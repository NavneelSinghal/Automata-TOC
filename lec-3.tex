\documentclass[a4paper]{article}
\usepackage[english]{babel}
\usepackage[a4paper,top=2cm,bottom=2cm,left=2cm,right=2cm,marginparwidth=1.75cm]{geometry}
\usepackage{amsmath}
\usepackage{amsfonts}
\usepackage{amsthm}
\usepackage{amssymb}
\usepackage{graphicx}
\usepackage[colorinlistoftodos]{todonotes}
\usepackage[colorlinks=true, allcolors=blue]{hyperref}
\usepackage{import}
\usepackage{pdfpages}
\usepackage{transparent}
\usepackage{xcolor}
\usepackage{algorithmicx}
\usepackage{algpseudocode}

\setlength{\parindent}{0pt}

\newtheorem{theorem}{Theorem}[section]
\newtheorem{lemma}{Lemma}
\newtheorem{claim}{Claim}
\newtheorem{subclaim}{Subclaim}
\newtheorem{defn}{Definition}
\newtheorem{notn}{Notation}
\newtheorem{obs}{Observation}
\newtheorem{eg}{Example}
\newtheorem{cor}{Corollary}
\newtheorem{note}{Note}

\renewcommand\qedsymbol{$\blacksquare$}
\newcommand{\nl}{\vspace{0.2cm}\\}
\newcommand{\mc}{\mathcal}
\newcommand{\mi}{\mathit}
\newcommand{\mf}{\mathbf}
\newcommand{\mb}{\mathbb}

\newcommand{\incfig}[1]{%
    \def\svgwidth{0.9\columnwidth}
    \import{./figures/}{#1.pdf_tex}
}
\pdfsuppresswarningpagegroup=1

\title{\textbf{COL352 Lecture 3}}
\date{}

\begin{document}
\maketitle
\tableofcontents

\iffalse

\section{Logistics}
\subsection{Lectures}
\begin{enumerate}
    \item Timings - Monday 11AM, Wednesday 11AM, Thursday 12PM
    \item Mode - synchronous, MS Teams
\end{enumerate}
\section{Grading}
\begin{center}
\begin{tabular}{|c|c|}
    \hline
    Class participation     &  5\% \\
    \hline
    Quizzes (best 3 of 4-5) & 30\% \\
    \hline
    HW (best 25 out of 30+) & 25\% \\
    \hline
    Major exam	            & 40\% \\
    \hline
\end{tabular}
\end{center}
\section{The Halting Problem}
\subsection{Statement}
Does there exist a program $H$ that always halts, and given inputs a program $P$ and an input $i$, determine whether $P$ terminates/halts on $i$?
\subsection{Answer}
No, there does not exist such a program.
\subsection{Proof}
Suppose there does exist such a program $H$.\nl
Consider the following program $C$:
\begin{algorithmic}[1]
    \Function{$C$}{$P$}
        \If{\Call{$H$}{$P, P$}}
            \State run an infinite loop
        \Else
            \State \Return
        \EndIf
    \EndFunction
\end{algorithmic}
Now consider what happens when we call $C$ on the input $C$.\nl
Suppose $C(C)$ doesn't halt. Then this means that $H(C, C)$ is false (by the definition of $H$).
Hence, by line 2, $C(C)$ must run forever, and this gives a contradiction.\nl
Hence $C(C)$ must halt. But in that case, we have $H(C, C)$ to be true, and by line $2$, we have a contradiction yet again.\nl
Hence our assumption that such a program $H$ exists is false, and we have proved the claim. $\blacksquare$.

\fi

% start here

\section{Regular Languages}

Recall the definition of a DFA from the last lecture - and consider $\delta$.\\

We'll try to extend $\delta$ to a certain $\hat{\delta}$, so that we get a function from $Q \times \Sigma^*$ to $Q$, such that $\hat{\delta}(q, w)$ is the state reached by starting from $q$ and
following transitions labelled by symbols in $w$.

Formally, let

\[
    \hat{\delta}(q, \epsilon) = q \quad \forall q \in Q
.\]
\[
    \hat{\delta}(q, xa) = \delta(\hat{\delta}(q, x), a) \quad \forall q \in Q, x \in \Sigma^*, a \in \Sigma
.\] 

\begin{defn}
    A DFA $(Q, \Sigma, \delta, q_0, A)$ is said to accept a string $w$ if $\hat{\delta}(q_0, w) \in A$.\\
    It is said to reject a string $w$ if $\hat{\delta}(q_0, w) \not\in A$.
\end{defn}

\begin{defn}
    The language recognised by $D$ is the set $\{w \in \Sigma^* \mid D \text{ accepts } w\}$, and is denoted by $\mc{L}(D)$, and is also sometimes called the language of $D$.
\end{defn}

\begin{defn}
    A language is said to be regular if it is recognised by some DFA.
\end{defn}

\begin{eg}
    $\Sigma = \{0, 1\}$, $L_k = \{w \mid w \text{ is the binary representation of a multiple of }k\}$
    Then $L_k$ is regular for all $k \in \mb{N}$.
\end{eg}

\begin{eg}
    $\Sigma = \{a, b\}$, $END_w = \{x \mid x \text{ ends in } w\}$ for some $w \in \Sigma^*$. Then $END_w$ is regular for all $w \in \Sigma^*$.
\end{eg}

\begin{claim}
    Let $L \subseteq \Sigma^*$ be a regular language. Then $\Sigma^* \setminus L$ is also regular.
\end{claim}
\begin{proof}
    Suppose $D = (Q, \Sigma, \delta, q_0, A)$ be a DFA that accepts $L$. Then consider the DFA $D' = (Q, \Sigma, \delta, q_0, Q \setminus A)$. Consider the language $\mc{L}(D')$.


    $w \in L \iff \hat{\delta}(q_0, w) \in A \iff \hat{\delta}(q_0, w) \not\in Q \setminus A \iff w \not\in \mc{L}(D')$.

        So we have $\mc{L}(D') = \Sigma^* \setminus L$, and hence we are done.
\end{proof}
In other words, the class of regular languages is closed under complementation.
\begin{claim}
If $L_1, L_2 \in \Sigma^*$ are regular languages, then $L_1 \cap L_2$ is also regular. 
\end{claim}
\begin{proof}
    Suppose $D_1 = (Q_1, \Sigma, \delta_1, q_1, A_1)$ is a DFA recognising $L_1$.
    Suppose $D_2 = (Q_2, \Sigma, \delta_2, q_2, A_2)$ is a DFA recognising $L_2$.
    Let $D$ be a DFA defined by 
    \[
        D = (Q_1 \times Q_2, \Sigma, \delta, (q_1, q_2), A_1 \times A_2)
    .\] 
    where $\delta$ is defined as
    \[
        \delta((s_1, s_2), a) = (\delta_1(s_1, a), \delta_2(s_2, a))
    .\] 
    Now we claim that $D$ recognizes $L_1 \times L_2$.
    \begin{subclaim}
        $\hat{\delta}((s_1, s_2), w) = (\hat{\delta_1}(s_1, w), \hat{\delta_2}(s_2, w))$
    \end{subclaim}
    \begin{proof}
        We proceed by induction on $|w|$. When $|w| = 0$, we have $\hat{\delta}((s_1, s_2), \epsilon) = (s_1, s_2) = (\hat{\delta_1}(s_1, w), \hat{\delta_2}(s_2, w))$, so we are done in this
        case. Now suppose $w = xa$ for some $x \in \Sigma^*, a \in \Sigma$.
        Then we have
        \begin{align*}
            \hat{\delta}((s_1, s_2), xa) &= \delta(\hat{\delta}((s_1, s_2), x)) \\
                                  &= \delta((\hat{\delta_1}(s_1, x), \hat{\delta_1}(s_1, x)), a)\\
                                  &= (\delta_1(\hat{\delta_1}(s_1, x), a), \delta_2(\hat{\delta_2}(s_2, x), a))\\
                                  &= (\hat{\delta_1}(s_1, xa), \hat{\delta_2}(s_2, xa))
        \end{align*}
        whence we are done.
    \end{proof}
    Now we use this claim to see the following sequence of equivalences:

    \begin{align*}
        w \in L_1 \cap L_2 &\iff w \in L_1 \land w \in L_2\\
                    &\iff D_1 \text{ accepts } w \land D_2 \text{ accepts } w\\
                    &\iff \hat{\delta_1}(q_1, w) \in A_1 \land \hat{\delta_2}(q_2, w) \in A_2\\
                    &\iff (\hat{\delta_1}(q_1, w), \hat{\delta_2}(q_2, w)) \in A_1 \times A_2\\
                    &\iff D \text{ accepts } w
    \end{align*}
    whence we are done.
\end{proof}

In other words, the class of regular languages is closed under intersection.

\begin{claim}
If $L_1, L_2 \in \Sigma^*$ are regular languages, then $L_1 \cup L_2$ is also regular. 
\end{claim}

\begin{proof}
    We show two proofs.
    \begin{enumerate}
        \item $L_1 \cup L_2 = \Sigma^* \setminus ((\Sigma^* \setminus L_1) \cap (\Sigma^* \setminus L_2))$, so using closure properties under complementation, we are done.
        \item (sketch) use the same DFA as before, but replace $A$ to $\{(s_1, s_2) \mid s_1 \in A_1 \lor s_2 \in A_2\}$ (which is equivalent to saying that $A = Q_1 \times Q_2 \setminus ((Q_1
            \setminus A_1)
            \times (Q_2 \setminus A_2))$)
    \end{enumerate}
\end{proof}

\begin{cor}
    Every finite language $L$ is regular.
\end{cor}
\begin{proof}
\begin{figure}[h]
    \centering
    \incfig{word-matching}
    \caption{Automaton for string matching}
    \label{fig:word-matching}
\end{figure}
    If $|L| = 1$, then we are done (make a DFA to recognize a single word).
    Else, use closure under union and induction on $|L|$.
\end{proof}
\begin{note}
    Note that this doesn't extend to all languages $L$, since we never said that the union of a countable collection of regular languages is regular.
\end{note}

\end{document}
