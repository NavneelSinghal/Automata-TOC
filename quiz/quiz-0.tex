% CUSTOM TEMPLATE FOR SOLUTIONS STARTS
\documentclass[answers]{exam}

\usepackage{graphicx}
\usepackage{float}
\usepackage{amsmath}
\usepackage{amsfonts}
\usepackage{framed}
\usepackage{algorithmicx}
\usepackage{algpseudocode}
\newcommand{\ans}[1]{\begin{framed}{\textbf{Answer:} #1}\end{framed}}
\newcommand{\sol}{\uplevel{\textsc{Solution:}}}
\newenvironment{answer}{%
    \renewcommand{\solutiontitle}{\noindent\textbf{Answer:}\enspace}
    \begin{solution}
        }{%
    \end{solution}
    \renewcommand{\solutiontitle}{\noindent\textbf{Solution:}\enspace}
}
% CUSTOM TEMPLATE FOR SOLUTIONS ENDS

% First we setup the header and footer
\pagestyle{headandfoot}
\runningheadrule
\runningfootrule
\header{COL352 (CSE, IITD, Semester-II-2020-21)}{}{Quiz 0}
\footer{}{\thepage  \, of \numpages}{}

% We want the points for each question displayed on the left
%\pointname{points}
%\pointsinmargin

% Automatically total the points - make sure to compile TWICE
\addpoints

\begin{document}


\vspace{0.1in}


\vspace{0.1in}
% Some general text together with number of questions and total points possible
There are \numquestions\, questions for a total of \numpoints\, points.
\vspace{0.1in}
\hrule
\vspace{0.2in}
\begin{questions}

    \question[10] 

    Let $\Sigma$ be a finite alphabet. Let $L \subseteq \Sigma^*$ be any language over $\Sigma$. Define the relation $=_L$ on $\Sigma^*$ as follows: $x=_Ly$ if and only if for all $z \in \Sigma^*$ either
    both $x\cdot z$ and $y\cdot z$ are in $L$, or both are not in $L$.

    \begin{parts}

        \part Let $\Sigma=\{0,1\}$, and let $E$ be the set of all binary strings containing an equal number of $0$'s and $1$'s. Prove that the equivalence relation $=_E$ has infinitely many equivalence classes. 
        \begin{solution}
            We shall show this by exhibiting an infinite set of strings such that no two distinct strings in the set are equivalent under $=_E$.

            Notation: by $a^n$ we denote the string which is formed by putting $n$ instances of a symbol $a \in \Sigma$ together. Note that $a^n \in \Sigma^*$.

            Consider the set $S = \{0^n \mid n \in \mathbb{Z}, n > 0\}$.

            Suppose for some $0^n, 0^m \in S$ for $m \ne n$, we have $0^n =_E 0^m$.

            Consider the string $z = 1^n \in \Sigma^*$. Then we note that $0^n \cdot 1^n \in \Sigma^*$ has an equal number of $0$s and $1$s, and thus belongs to $E$. However, in the string $0^m \cdot 1^n$, we have $m$ 0s and $n$ 1s, which are not equal, and thus $0^m \cdot 1^n$ is not in $E$.

            However, this is a contradiction to the definition of $=_E$, and thus our assumption about the existence of such $n, m$ is false, and thus we have shown that for any $m \ne n$, we have $0^m \ne_E 0^n$, implying that there exists one equivalence class for each $n$ (corresponding to $0^n$), which implies that the number of equivalence classes is infinite.

        \end{solution}

        \part Let $D=(Q,\Sigma,\delta,q_0,A)$ be any DFA and let $L$ be its language. Prove that the number of equivalence classes of $=_L$ is at most $|Q|$.
        \begin{solution}
            For a state $q$, define $S_q = \{x \in \Sigma^* \mid \hat{\delta}(q_0, x) = q\}$. Note that the union of all these sets over $q$ is $\Sigma^*$, since running a DFA can lead to possibly any of the $|Q|$ states. Also, all of these sets are disjoint.

            Consider the set $W = \{S_q \mid q \in Q \land S_q \ne \emptyset\}$. We claim that there is an surjective mapping from $W$ to the set of equivalence classes of $=_L$. This will imply that the number of equivalence classes of $=_L$ is at most $|W| \le |Q|$, from where we shall be done.

            We will first show that all elements in $S_q$ are equivalent under $=_L$. Note that for
            $x, y \in S_q$, we have $\hat{\delta}(q_0, x \cdot z) = \hat{\delta}(\hat{\delta}(q_0, x), z) = \hat{\delta}(q, z)$ (the first equality follows from the fact that $x \in S_q$, and the second follows from the definition of $\hat{\delta}$ inductively, the pertaining proof is given in the last paragraph). Similarly, we have $\hat{\delta}(q_0, y \cdot z) = \hat{\delta}(q, z)$. Hence depending on whether $\hat{\delta}(q, z) \in A$ or not, the strings $x \cdot z$ and $y \cdot z$ are either both accepted or rejected by $D$, which is equivalent to being (respectively) both in or not in $L$, and thus $x =_L z$.

            We shall now construct a surjective mapping from $W$ to the set of (non-empty, by definition) equivalence classes of $=_L$, which shall solve the problem as claimed above.

            Consider the function $f$ defined as follows:

            $f(S_q) = $ equivalence class of any string in $S_q$. Note that this mapping is well defined by the third paragraph, and is a total function since each $S_q$ is a function and equivalence classes form a partition of the whole set. Now consider any equivalence class $C$ of $S$. Since it is non-empty by definition, there exists an element $c$ in $C$. Now note that $f(S_{\hat{\delta}(q_0, c)}) = C$, which follows from the definition of $S_q$, and thus for each equivalence class $C$ of $=_L$, we have a corresponding pre-image (under this mapping) of $C$ in $W$. Thus this function is a surjective function, as needed in the second paragraph.

            * Proof of the fact that $\hat{\delta}(q, x \cdot y) = \hat{\delta}(\hat{\delta}(x), y)$:
            We proceed by induction on $n = |y|$.
            Base cases:

            $n = 0$: in this case, we have $\hat{\delta}(q, x \cdot y) = \hat{\delta}(q, x \cdot \epsilon) = \hat{\delta}(q, x) = \hat{\delta}(\hat{\delta}(q, x), \epsilon) = \hat{\delta}(\hat{\delta}(q, x), y)$, as needed.

            $n = 1$: in this case, we have $\hat{\delta}(q, x \cdot y) = \delta(\hat{\delta}(q, x), y) = \hat{\delta}(\hat{\delta}(q, x), y)$, since $y$ is a single character.

            $n > 1$: in this case, let $y = z \cdot a$ for some $a$ of length $1$. Then we have $\hat{\delta}(q, x \cdot y) = \hat{\delta}(q, (x \cdot z) \cdot a) = \hat{\delta}(\hat{\delta}(q, x \cdot z), a) = \hat{\delta}(\hat{\delta}(\hat{\delta}(q, x), z), a) = \hat{\delta}(\hat{\delta}(q, x), z \cdot a) = \hat{\delta}(\hat{\delta}(q, x), y)$, as required. The first equality follows by associativity of concatenation, the second by the base case applied on $y$ replaced by $a$, the third by the base case applied on $y$ replaced by $z$, the fourth by the base case applied on $q$ replaced by $\hat{\delta}(q, x)$ and $x, y$ replaced by $z, a$, and the fifth by the equality of $y$ and $z \cdot a$.


        \end{solution}

        \part Using the above two claims, prove that $E$ is not a regular language.
        \begin{solution}

            Suppose, on the contrary, that $E$ is in fact a regular language.

            Then there must exist a DFA $D = (Q, \Sigma, \delta, q_0, A)$ such that $E = \mathcal{L}(D)$ (note that $Q$ here is a finite set of states by the definition of a DFA).

            Then we must have at most $|Q|$ equivalence classes of $=_E$ by Q1.2. However, there are infinitely many equivalence classes of $=_E$ by Q1.1, which is a contradiction, as $Q$ is a finite set of states.

            Hence the assumption that $E$ is regular is wrong, and thus $E$ is not a regular language.
        \end{solution}

    \end{parts}

\end{questions}
\end{document}

