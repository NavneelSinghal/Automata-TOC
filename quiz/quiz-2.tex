% CUSTOM TEMPLATE FOR SOLUTIONS STARTS
\documentclass[answers]{exam}

\usepackage{graphicx}
\usepackage{float}
\usepackage{amsmath}
\usepackage{amsfonts}
\usepackage{framed}
\usepackage{algorithmicx}
\usepackage{algpseudocode}
\newcommand{\ans}[1]{\begin{framed}{\textbf{Answer:} #1}\end{framed}}
\newcommand{\sol}{\uplevel{\textsc{Solution:}}}
\newenvironment{answer}{%
    \renewcommand{\solutiontitle}{\noindent\textbf{Answer:}\enspace}
    \begin{solution}
        }{%
    \end{solution}
    \renewcommand{\solutiontitle}{\noindent\textbf{Solution:}\enspace}
}
\newcommand{\nl}{\vspace{0.2cm}\\}
\newcommand{\mc}{\mathcal}
\newcommand{\mi}{\mathit}
\newcommand{\mf}{\mathbf}
\newcommand{\mb}{\mathbb}
\renewcommand{\L}{\mc{L}}
\newcommand{\hd}{\hat{\delta}}
% CUSTOM TEMPLATE FOR SOLUTIONS ENDS

% First we setup the header and footer
\pagestyle{headandfoot}
\runningheadrule
\runningfootrule
\header{COL352 (CSE, IITD, Semester-II-2020-21)}{}{Quiz 1}
\footer{}{\thepage  \, of \numpages}{}

% We want the points for each question displayed on the left
%\pointname{points}
%\pointsinmargin

% Automatically total the points - make sure to compile TWICE
\addpoints

\begin{document}

\vspace{0.1in}

\vspace{0.1in}
% Some general text together with number of questions and total points possible
There are \numquestions\, questions for a total of \numpoints\, points.
\vspace{0.1in}
\hrule
\vspace{0.2in}
\begin{questions}

    \question[5]

    Prove that the language 
    \[
        \{x\text{ }|\text{ }x\text{ is the binary representation of }n!\text{, without leading 0's, for some }n\in\mathbb{N}\}
    \]
    is not regular.

    \begin{solution}
        Consider the set $S$ of binary representations of $(2^n - 1)!$ for all $n$ in $\mb{N}$.

        We claim that all of these are in different equivalence classes.

        Consider any two distinct natural numbers $n, m$, and wlog assume $n < m$. Suppose $(2^n - 1)!$ has a binary representation $x$, and $(2^m - 1)!$ has a binary representation $y$.

        Consider the binary representation of $(2^n - 1)! \times (2^n)$ and $(2^m - 1)! \times (2^n)$. Clearly, they are $x$ followed by $n$ zeros and $y$ followed by $n$ zeros respectively.

        The first of them is the binary representation of $(2^n)!$, and hence is in the language.

        Now note that
        \[
            (2^m - 1)! < (2^m - 1)! \times 2^n < (2^m - 1)! \times 2^m = (2^m)!
        \]
        Since this implies that this integer lies strictly between two consecutive factorials, the fact that factorials form a monotonically non-decreasing sequence implies that it is not the factorial of any integer. Hence, its binary representation is not in the language.

        So one of $x0^n$ and $y0^n$ is in the language, while the other one is not, so both of them are in different equivalence classes.

        This shows that any two strings in the set S are in different equivalence classes. Since $S$ is infinite, there are infinitely many equivalence classes.

        Since there are infinitely many equivalence classes, the language can't be regular, as needed.
    \end{solution}

    \question[5]

    Let $L_k\subseteq\{0,1\}^*$ be the language defined as 
    \[
        L_k=\{xy\text{ }|\text{ }x\text{, }y\in\{0,1\}^k\text{, and the bitwise-AND of }x\text{ and }y\text{ is }0^k\}\text{.}
    \]
    Observe that $L_k$ is finite, and hence, regular. Prove that for all $k$, any DFA that recognizes $L_k$ has at least $2^k$ states.
    \begin{solution}
        Using the Myhill Nerode theorem, it suffices to show that there are at least $2^k$ equivalence classes of $=_{L_k}$, since the number of states in the minimal DFA equals the number of
        equivalence classes of $=_{L_k}$.

        For this, we shall show that each binary string of size $k$ is in its own equivalence class.

        Consider any two distinct binary strings $x$ and $y$. We say that a string $u$ is dominated by string $v$ (and equivalently, $v$ dominates $u$) iff for all $i$, $u[i] = 1 \implies v[i] = 1$.

        We make two cases:

        \begin{enumerate}
            \item $x$ does not dominate $y$.\nl
                Let $x'$ be the bitwise NOT of $x$ (i.e., $x'[i] = 0$ if and only if $x[i] = 1$). Then $xx'$ is in $L_k$ since the bitwise AND of $x$ and $x'$ is $0^k$ by definition of bitwise NOT.
                Now since $x$ does not dominate $y$, there must exist an $i$ such that $x[i] = 0$ and $y[i] = 1$. Then we have $x'[i] = 1$, so the bitwise AND of $y$ and $x'$ has a $1$ at the
                $i^\mathrm{th}$ bit, which means that $yx'$ is not in $L_k$.
                Since we have exhibited $x'$ such that exactly one of $xx'$ and $yx'$ is in the language and the other isn't, this shows that $x$ and $y$ belong to different equivalence classes of $=_{L_k}$.
            \item $x$ dominates $y$.\nl
                Let $y'$ be the bitwise NOT of $y$. Since $x$ dominates $y$, $y[i] = 1 \implies x[i] = 1$. If the converse were true (i.e., if it were the case that for all $i$, $x[i] = 1 \implies
                y[i] = 1$), then it would imply that $y = x$, which doesn't hold for unequal $x$ and $y$. Hence, there exists at least one $i$ for which the converse is false, i.e., for which $x[i] =
                1$ and $y[i] = 0$. In this case, $x[i] = 1$ and $y'[i] = 1$, so $xy'$ is not in $L_k$. Also, by a similar argument as in the previous case, $yy'$ is in $L_k$. Hence, $x$ and $y$ are in
                different equivalence classes of $=_{L_k}$ again.
        \end{enumerate}

        Since the above cases are exhaustive, we can see that each binary string of length $k$ belongs to a distinct equivalence class of $=_{L_k}$, and from here, it follows that there are at least
        $2^k$ equivalence classes of $=_{L_k}$. Combining this with the first comment, we are done.
    \end{solution}

\end{questions}
\end{document}

