% CUSTOM TEMPLATE FOR SOLUTIONS STARTS
\documentclass[answers]{exam}

\usepackage{graphicx}
\usepackage{float}
\usepackage{amsmath}
\usepackage{amsfonts}
\usepackage{framed}
\usepackage{algorithmicx}
\usepackage{algpseudocode}
\newcommand{\ans}[1]{\begin{framed}{\textbf{Answer:} #1}\end{framed}}
\newcommand{\sol}{\uplevel{\textsc{Solution:}}}
\newenvironment{answer}{%
    \renewcommand{\solutiontitle}{\noindent\textbf{Answer:}\enspace}
    \begin{solution}
        }{%
    \end{solution}
    \renewcommand{\solutiontitle}{\noindent\textbf{Solution:}\enspace}
}
\newcommand{\nl}{\vspace{0.2cm}\\}
\newcommand{\mc}{\mathcal}
\newcommand{\mi}{\mathit}
\newcommand{\mf}{\mathbf}
\newcommand{\mb}{\mathbb}
\renewcommand{\L}{\mc{L}}
\newcommand{\hd}{\hat{\delta}}
% CUSTOM TEMPLATE FOR SOLUTIONS ENDS

% First we setup the header and footer
\pagestyle{headandfoot}
\runningheadrule
\runningfootrule
\header{COL352 (CSE, IITD, Semester-II-2020-21)}{}{Quiz 1}
\footer{}{\thepage  \, of \numpages}{}

% We want the points for each question displayed on the left
%\pointname{points}
%\pointsinmargin

% Automatically total the points - make sure to compile TWICE
\addpoints

\begin{document}

\vspace{0.1in}

\vspace{0.1in}
% Some general text together with number of questions and total points possible
There are \numquestions\, questions for a total of \numpoints\, points.
\vspace{0.1in}
\hrule
\vspace{0.2in}
\begin{questions}

    \question[4] 
    Let $\Sigma=\{0,1\}$, and let $L=\{xy\in\Sigma^*\text{ }|\text{ \# of 0's in }x=\text{ \# of 1's in }y\}$. Prove that $L$ is a regular language.

    \begin{solution}
        We claim that $L = \Sigma^*$.

        Proof:

        Consider any string $x = x[1] x[2] \ldots x[n]$ in $\Sigma^*$. Let $o[i]$ = number of ones in $x[1 \ldots i]$, and $z[i]$ = number of zeros in $x[i \ldots n]$. Define $o[0] = z[n + 1] = 0$ vacuously (for the case when the prefix/suffix is the empty string).

        Also define $f(i) = z[i] - o[i - 1]$ for $1 \le i \le n + 1$.

        We first show that there exists an $i$ such that $f(i) = 0$ and $1 \le i \le n + 1$.

        For this, we claim the following two things:

        Claim 1. $f(i) = z[i] - o[i - 1]$ changes by $1$ at each step when going from $i$ to $i + 1$, i.e., $|f(i) - f(i + 1)| = 1$ for all valid $i$.
        
        Proof: Make two cases on $x[i]$:
        \begin{enumerate}
            \item when $x[i] = 0$, then $o[i] = o[i - 1]$ and $z[i + 1] = z[i] - 1$, so we have $|f(i) - f(i + 1)| = 1$.
            \item when $x[i] = 1$, then $o[i] = o[i - 1] + 1$ and $z[i + 1] = z[i]$, so we again have $|f(i) - f(i + 1)| = 1$.
        \end{enumerate}

        Claim 2. $f(1) \ge 0$ and $f(n + 1) \le 0$

        Proof: Clearly, we have $f(1) = z[1] - o[0] = z[1] \ge 0$ since counts are always non-negative. We also have $f(n + 1) = z[n + 1] - o[n] = -o[n] \le 0$ by the same reason.

        Now we shall implicitly prove and invoke a form of the discrete intermediate value theorem on integers as follows:

        Suppose there exists no $i$ such that $f(i) = 0$. Wlog suppose that $f(1) > 0$ (the other case is analogous). Then we show that $f(i) > 0$ for all valid $i$, which shall contradict Claim 2.

        We shall show this using induction. The base case is true. Now suppose that $f(i) > 0$ for some $i \ge 1$. Note that $f(i + 1) = f(i) + 1$ or $f(i) - 1$. If $f(i + 1) < 0$, the fact that $f(i)
        > 0$ implies that $|f(i + 1) - f(i)| \ge 2$, which is false, hence it must hold that $f(i + 1) > 0$, since it can't be equal to $0$ by assumption. Thus, the induction is complete, so $f(n + 1)
        > 0$, which is a contradiction to Claim 2.

        Hence our assumption was wrong, and $f(i) = 0$ for some $1 \le i \le n + 1$. Consider such an $i$.

        We have $o[i - 1] = z[i]$. So we have the number of zeros in $x[1 \ldots i - 1]$ equal to the number of ones in $x[i \ldots n]$. This shows that $x$ is in $L$. Since $x$ was arbitrary, we have
        $\Sigma^* subseteq L$. Now by definition of a language, $L subseteq \Sigma^*$, so $L = \Sigma^*$.

        Now it is easy to make a DFA for $L = \Sigma^*$: for instance, consider $(Q, \Sigma, \delta, q_0, A) = (\{q_0\}, \{0, 1\}, \delta, q_0, \{q_0\})$ where $\delta$ is defined as $\delta(q_0, 0)
        = \delta(q_0, 1) = q_0$. (Note: in fact, any DFA with $Q = A$ and $|Q| > 0$ works).

        Thus $L$ is indeed a regular language as needed.
    \end{solution}

    \question[6]
    Let $L\subseteq\Sigma^*$ be any regular language. Prove that the language $L'=\{z\text{ }|\exists x,y\in\Sigma^*\text{ such that }z=xy\text{ and }yx\in L\}$ is regular.

    \begin{solution}
        Suppose the $DFA D = (Q, \Sigma, \delta, q_0, A)$ recognizes $L$.

        Consider the following languages:
        \[
            L_q = \{x in \Sigma^* \mid \hd(q, x) \in A\}
        \]
        \[
            M_q = \{x in \Sigma^* \mid \hd(q_0, x) = q\}
        \]

        Claim 1: Both $L_q$ and $M_q$ are regular languages.

        Proof:

        Construct DFAs as follows:
        
        \begin{enumerate}
            \item $L_q$: consider the DFA $D_q = (Q, \Sigma, \delta, q, A)$.
            \item $M_q$: consider the DFA $D'_q = (Q, \Sigma, \delta, q_0, \{q\})$
        \end{enumerate}

        The fact that these DFAs correspond to these languages follows from the very definition of these languages.

        So $L_q M_q$ is a regular language (closure under concatenation).

        So, $L'' =  \bigcup_{q \in Q} L_q M_q$ is a regular language due to closure under finite union.

        Now we claim that $L'' = L'$; this shall complete the proof since we have already shown $L''$ is regular.

        For that we shall need the following claims:

        \begin{enumerate}
            \item Any string in $L''$ is in $L'$:

                Consider any string $x$ in $L''$. Then for some state $q \in Q$, $x \in L_q M_q$. So there exist two strings $x'$ and $x''$ in $L_q$ and $M_q$ such that $x = x' x''$ by definition of
                language concatenation. So we have $\hd(q_0, x'' x') = \hd(\hd(q_0, x''), x') = \hd(q, x')$ which is in $A$, by the definition of $L_q$ and $M_q$. So $x'' x'$ is accepted by $D$, and
                hence belongs to $L$. Thus $x$ is in $L'$ (since $x = x' x''$ and $x'' x'$ is in $L$).

            \item Any string in $L'$ is in $L''$.

                Consider any string $x$ in $L'$. Then there exist strings $x', x'' \in \Sigma^*$ such that $x = x' x''$ and $x'' x'$ is in $L$. Consider $q = \hd(q_0, x'')$. Then since $x'' x'$ is in
                $L$, we have $A \ni \hd(q_0, x'' x') = \hd(\hd(q_0, x''), x') = \hd(q, x')$. By definition of $q$, $x''$ is in $M_q$, and by the result we just obtained (that $\hd(q, x') \in A$), we
                have $x'$ is in $L_q$. Hence, $x = x' x''$ is in $L_q M_q$, which is a subset of $L''$. This establishes the fact that any string $x$ in $L'$ is in $L''$.
        \end{enumerate}

        Using the first claim, we have $L'' \subseteq L'$ and using the second claim, we have $L' \subseteq L''$, so combining these we have $L' = L''$, and we are done.


    \end{solution}

\end{questions}
\end{document}

