\documentclass[a4paper]{article}
\usepackage[english]{babel}
\usepackage[a4paper,top=2cm,bottom=2cm,left=2cm,right=2cm,marginparwidth=1.75cm]{geometry}
\usepackage{amsmath}
\usepackage{amsfonts}
% \usepackage{amsthm}
\usepackage{amssymb}
\usepackage{graphicx}
\usepackage[colorinlistoftodos]{todonotes}
\usepackage[colorlinks=true, allcolors=blue]{hyperref}
\usepackage{import}
\usepackage{pdfpages}
\usepackage{transparent}
\usepackage{xcolor}
\usepackage{algorithmicx}
\usepackage{algpseudocode}

\usepackage{thmtools}
\usepackage{enumitem}
\usepackage[framemethod=TikZ]{mdframed}

\usepackage{xpatch}

\usepackage{boites}
\makeatletter
\xpatchcmd{\endmdframed}
{\aftergroup\endmdf@trivlist\color@endgroup}
{\endmdf@trivlist\color@endgroup\@doendpe}
{}{}
\makeatother

%\usepackage[poster]{tcolorbox}
%\allowdisplaybreaks
%\sloppy

\usepackage[many]{tcolorbox}

\xpatchcmd{\proof}{\itshape}{\bfseries\itshape}{}{}

% to set box separation
\setlength{\fboxsep}{0.8em}
\def\breakboxskip{7pt}
\def\breakboxparindent{0em}

\newenvironment{proof}{\begin{breakbox}\textit{Proof.}}{\hfill$\square$\end{breakbox}}
\newenvironment{ans}{\begin{breakbox}\textit{Answer.}}{\end{breakbox}}
\newenvironment{soln}{\begin{breakbox}\textit{Solution.}}{\end{breakbox}}

% \tcolorboxenvironment{proof}{
%     blanker,
%     before skip=\topsep,
%     after skip=\topsep,
%     borderline={0.4pt}{0.4pt}{black},
%     breakable,
%     left=12pt,
%     right=12pt,
%     top=12pt,
%     bottom=12pt,
% }
%
% \tcolorboxenvironment{ans}{
%     blanker,
%     before skip=\topsep,
%     after skip=\topsep,
%     borderline={0.4pt}{0.4pt}{black},
%     breakable,
%     left=12pt,
%     right=12pt,
% }

\mdfdefinestyle{enclosed}{
    linecolor=black
    ,backgroundcolor=none
    ,apptotikzsetting={\tikzset{mdfbackground/.append style={fill=gray!100,fill opacity=.3}}}
    ,frametitlefont=\sffamily\bfseries\color{black}
    ,splittopskip=.5cm
    ,frametitlebelowskip=.0cm
    ,topline=true
    ,bottomline=true
    ,rightline=true
    ,leftline=true
    ,leftmargin=0.01cm
    ,linewidth=0.02cm
    ,skipabove=0.01cm
    ,innerbottommargin=0.1cm
    ,skipbelow=0.1cm
}

\mdfsetup{%
    middlelinecolor=black,
    middlelinewidth=1pt,
roundcorner=4pt}

\setlength{\parindent}{0pt}

\mdtheorem[style=enclosed]{theorem}{Theorem}
\mdtheorem[style=enclosed]{lemma}{Lemma}[theorem]
\mdtheorem[style=enclosed]{claim}{Claim}[theorem]
\mdtheorem[style=enclosed]{ques}{Question}
\mdtheorem[style=enclosed]{defn}{Definition}
\mdtheorem[style=enclosed]{notn}{Notation}
\mdtheorem[style=enclosed]{obs}{Observation}
\mdtheorem[style=enclosed]{eg}{Example}
\mdtheorem[style=enclosed]{cor}{Corollary}
\mdtheorem[style=enclosed]{note}{Note}

% \let\thetheorem=\relax
% \let\thelemma=\relax
% \let\theclaim=\relax
% \let\theques=\relax
% \let\thedefn=\relax
% \let\thenotn=\relax
% \let\theobs=\relax
% \let\thecor=\relax
% \let\thenote=\relax

% \renewcommand\qedsymbol{$\blacksquare$}
\newcommand{\nl}{\vspace{0.2cm}\\}
\newcommand{\mc}{\mathcal}
\newcommand{\mi}{\mathit}
\newcommand{\mf}{\mathbf}
\newcommand{\mb}{\mathbb}
\renewcommand{\L}{\mc{L}}
\newcommand{\hd}{\hat{\delta}}
\newcommand{\produces}{\implies}
\newcommand{\derives}{\stackrel{*}{\implies}}
\newcommand{\changesto}{\vdash}
\newcommand\Vtextvisiblespace[1][.3em]{%
    \mbox{\kern.06em\vrule height.3ex}%
    \vbox{\hrule width#1}%
    \hbox{\vrule height.3ex}
}
\newcommand{\blank}{{\Vtextvisiblespace[0.7em]}}
\newcommand{\leftend}{\triangleright}

\newcommand{\incfig}[1]{%
    \def\svgwidth{\columnwidth}
    \import{./figures/}{#1.pdf_tex}
}
\pdfsuppresswarningpagegroup=1

\title{\textbf{COL352 Lecture 24}}
\date{}

\begin{document}
\maketitle
\tableofcontents

\section{Recap}

Completed PDAs and grammars last time.

\section{Definitions}

\begin{defn}
    A \underline{Deterministic Turing Machine} (DTM) is a 8-tuple of the form $(Q, \Sigma, \Gamma, \delta, q_0, \blank, q_{acc}, q_{rej})$ where
    \begin{enumerate}
        \item $Q$ is a finite set of states
        \item $\Sigma$ is a finite alphabet
        \item $\Gamma \supsetneq \Sigma$ is the tape alphabet
        \item $q_0 \in Q$ is the initial state
        \item $\blank \in \Gamma \setminus \Sigma$ is a blank symbol (tape initially contains input $x \in \Sigma^*$ followed by infinitely many $\blank$).
        \item $q_{acc} \in Q$ is the accepting state, $q_{rej} \in Q$ is the rejecting state, and $q_{acc} \ne q_{rej}$
        \item $\delta : (Q \setminus \{q_{acc}, q_{rej}\}) \times \Gamma \to Q \times \Gamma \times \{L, R\}$.
            Here the first input is the current state, second input is current tape content, the first component of the output is the new state, second is new tape content, and third is the
            direction in which we need to move.
    \end{enumerate}
    If the machine is reading the leftmost tape cell and it makes a leftward move, the location of the head doesn't change.
\end{defn}

\begin{defn}
    A DTM $M$ accepts $x \in \Gamma^*$ if $\exists x_1, x_2 \in \Gamma^* : (\epsilon, q_0, x) \changesto^*_M (x_1, q_{acc}, x_2)$.
\end{defn}

\begin{defn}
    $L \in \Sigma^*$ is said to be recognised by $M$ if $\forall x \in \Sigma^*$ : $x \in L \iff M$ accepts $x$.
\end{defn}

    \begin{defn}
        $L$ is said to be Turing-recognisable if it is recognised by some DTM.
    \end{defn}

    \begin{defn}
        A Turing machine $M$ is called a \underline{decider} if it halts on every input $x \in \Sigma^*$, i.e.,
        \[
            \forall x \in \Sigma^* : \exists x_1, x_2 \in \Gamma^* : (\epsilon, q_0, x) \changesto^* (x_1, q_{acc}, x_2) \text{ or } (\epsilon, q_0, x) \changesto^* (x_1, q_{rej}, x_2)
        \]
        A language $L \in \Sigma^*$ is said to be \underline{decided}  by $M$ if $M$ is a decider and $L$ is recognised by $M$.
    \end{defn}

    \begin{defn}
        $L \in \Sigma^*$ is said to be \underline{decidable} if it is decided by some DTM $M$.
    \end{defn}

    \begin{defn}
        A $2-$tape DTM is $(Q, \Sigma, \Gamma, \delta, q_0, \blank, q_{acc}, q_{rej})$, where $Q, \Sigma, \Gamma, q_0, \blank, q_{acc}, q_{rej}$ are as before, and $\delta : (Q \setminus
        \{q_{acc}, Q_{rej}\}) \times \Gamma^2 \to Q \times \Gamma^2 times \{L, R, S\}^2$.
    \end{defn}
    For now, $S$ is used since moving both tape heads preserves parity of sum of locations, except for the borders.

    The language recognised or decided by 2-tape DTM is defined analogously.

    \section{Content}

    \begin{ques}
        What is the set of instantaneous descriptions of a DTM?
    \end{ques}

    \begin{ans}
        Set of instantaneous description = $\Gamma^* \times Q \times \Gamma^*$, and a state $(x_1, q, x_2)$ is the following:
        $x_1$ is the part of the tape that is strictly to the left of the current head, and $x_2$ is the remaining part (leaving out the trailing blanks).
    \end{ans}

    Exercise: Formally define the ``changes to" relation $\changesto_M$ on the set of instantaneous descriptions.\nl
    $\changesto^*_M$ is the reflexive transitive closure of $\changesto_M$.\nl

    \begin{claim}
        $\forall L \in \Sigma^*$, $L$ is decidable $\implies$ $L$ is Turing-recognisable.
    \end{claim}
    \begin{proof}
        Definition.
    \end{proof}

    Note that we will see later on that the opposite direction is not true.\nl

    \subsection{2-tape DTM}

    We now have two tapes.

    Observe that

    \begin{theorem}
        $L$ recognised by a DTM $\iff$ $L$ recognised by a 2-tape DTM.\nl
        $L$ decided by a DTM $\implies$ $L$ decided by a 2-tape DTM.
    \end{theorem}
    
    \begin{proof}
        The forward implication is obvious (just don't use the second tape).\nl
        Consider the instantaneous description of the 2-tape DTM -- $(x_1, y_1)$ and $(x_2, y_2)$ being the corresponding strings before and not before the tape head on both tapes, and let
        $\square$ be the blank symbol of the 2-tape DTM.\nl
        Our single tape DTM would have the following:
        $\leftend x_1 \to y_1 \square \ldots \square \# x_2 \to y_2 \square \ldots \square 
        \blank \ldots $\nl
        Call the 2-tape one $M_2$ and the 1-tape one $M_1$.\nl
        $M_1 = (Q, \Sigma, \Gamma, \delta, q_0, \square, q_{acc}, q_{rej})$\nl
        $M_2 = (Q', \Sigma, \Gamma \cup \{\leftend, \to, \#, \blank\}, q_0', \blank, q_{acc}', q_{rej}')$\nl
        \paragraph{Behaviour of $M_1$}
        \begin{enumerate}
            \item Preprocess phase\nl
                To establish the invariant, we need to change $x \blank \ldots $ to $\leftend \to x \# \to \blank \ldots$
            \item Simulation phase\nl
                We shall store (state of $M_2$, symbol from $\Gamma$, symbol from $\Gamma$, $L/R/S$, $L/R/S$).\nl
                For each transition of $M_2$, $M_1$ makes three passes over the tape.
                \begin{enumerate}
                    \item Insert $\square$ before the $\#$ and overwrite the leftmost $\blank$ by $\square$.
                    \item Read the symbols after the two $\to$s and save them in the state.
                    \item Execute the transitions of $M_2$ in $M_1$ (doesn't need a complete pass of its own).
                    \item Implement the transitions on the tape: overwrite the cells next to the $\to$s and move the $\to$s.
                \end{enumerate}
        \end{enumerate}
    \end{proof}

    \end{document}
