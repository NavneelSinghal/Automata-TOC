\documentclass[a4paper]{article}
\usepackage[english]{babel}
\usepackage[a4paper,top=2cm,bottom=2cm,left=2cm,right=2cm,marginparwidth=1.75cm]{geometry}
\usepackage{amsmath}
\usepackage{amsfonts}
% \usepackage{amsthm}
\usepackage{amssymb}
\usepackage{graphicx}
\usepackage[colorinlistoftodos]{todonotes}
\usepackage[colorlinks=true, allcolors=blue]{hyperref}
\usepackage{import}
\usepackage{pdfpages}
\usepackage{transparent}
\usepackage{xcolor}
\usepackage{algorithmicx}
\usepackage{algpseudocode}

\usepackage{thmtools}
\usepackage{enumitem}
\usepackage[framemethod=TikZ]{mdframed}

\usepackage{xpatch}

\usepackage{boites}
\makeatletter
\xpatchcmd{\endmdframed}
{\aftergroup\endmdf@trivlist\color@endgroup}
{\endmdf@trivlist\color@endgroup\@doendpe}
{}{}
\makeatother

%\usepackage[poster]{tcolorbox}
%\allowdisplaybreaks
%\sloppy

\usepackage[many]{tcolorbox}

\xpatchcmd{\proof}{\itshape}{\bfseries\itshape}{}{}

% to set box separation
\setlength{\fboxsep}{0.8em}
\def\breakboxskip{7pt}
\def\breakboxparindent{0em}

\newenvironment{proof}{\begin{breakbox}\textit{Proof.}}{\hfill$\square$\end{breakbox}}
\newenvironment{ans}{\begin{breakbox}\textit{Answer.}}{\end{breakbox}}
\newenvironment{soln}{\begin{breakbox}\textit{Solution.}}{\end{breakbox}}

% \tcolorboxenvironment{proof}{
%     blanker,
%     before skip=\topsep,
%     after skip=\topsep,
%     borderline={0.4pt}{0.4pt}{black},
%     breakable,
%     left=12pt,
%     right=12pt,
%     top=12pt,
%     bottom=12pt,
% }
% 
% \tcolorboxenvironment{ans}{
%     blanker,
%     before skip=\topsep,
%     after skip=\topsep,
%     borderline={0.4pt}{0.4pt}{black},
%     breakable,
%     left=12pt,
%     right=12pt,
% }

\mdfdefinestyle{enclosed}{
    linecolor=black
    ,backgroundcolor=none
    ,apptotikzsetting={\tikzset{mdfbackground/.append style={fill=gray!100,fill opacity=.3}}}
    ,frametitlefont=\sffamily\bfseries\color{black}
    ,splittopskip=.5cm
    ,frametitlebelowskip=.0cm
    ,topline=true
    ,bottomline=true
    ,rightline=true
    ,leftline=true
    ,leftmargin=0.01cm
    ,linewidth=0.02cm
    ,skipabove=0.01cm
    ,innerbottommargin=0.1cm
    ,skipbelow=0.1cm
}

\mdfsetup{%
    middlelinecolor=black,
    middlelinewidth=1pt,
roundcorner=4pt}

\setlength{\parindent}{0pt}

\mdtheorem[style=enclosed]{theorem}{Theorem}
\mdtheorem[style=enclosed]{lemma}{Lemma}[theorem]
\mdtheorem[style=enclosed]{claim}{Claim}[theorem]
\mdtheorem[style=enclosed]{ques}{Question}
\mdtheorem[style=enclosed]{defn}{Definition}
\mdtheorem[style=enclosed]{notn}{Notation}
\mdtheorem[style=enclosed]{obs}{Observation}
\mdtheorem[style=enclosed]{eg}{Example}
\mdtheorem[style=enclosed]{cor}{Corollary}
\mdtheorem[style=enclosed]{note}{Note}

% \let\thetheorem=\relax
% \let\thelemma=\relax
% \let\theclaim=\relax
% \let\theques=\relax
% \let\thedefn=\relax
% \let\thenotn=\relax
% \let\theobs=\relax
% \let\thecor=\relax
% \let\thenote=\relax

% \renewcommand\qedsymbol{$\blacksquare$}
\newcommand{\nl}{\vspace{0.2cm}\\}
\newcommand{\mc}{\mathcal}
\newcommand{\mi}{\mathit}
\newcommand{\mf}{\mathbf}
\newcommand{\mb}{\mathbb}
\renewcommand{\L}{\mc{L}}
\newcommand{\hd}{\hat{\delta}}

\newcommand{\incfig}[1]{%
    \def\svgwidth{\columnwidth}
    \import{./figures/}{#1.pdf_tex}
}
\pdfsuppresswarningpagegroup=1

\title{\textbf{COL352 Lecture 10}}
\date{}

\begin{document}
\maketitle
\tableofcontents

\section{Recap}

Game playing version of pumping lemma.

\section{Definitions}

\begin{defn}
    $D_1 = (Q_1, \Sigma, \delta_1, q_1, A_1)$ is isomorphic to $D_2 = (Q_2, \Sigma, \delta, q_2, A_2)$ if $\exists$ a bijection $h : Q_1 \to Q_2$ such that:
    \begin{enumerate}
        \item $h(q_1) = q_2$
        \item $q \in A \iff h(q) \in A_2$
        \item $h(\delta_1(q, a)) = \delta_2(h(q), a)$
    \end{enumerate}
\end{defn}

Observe that if $D_1, D_2$ are isomorphic, then $\L(D_1) = \L(D_2)$.\nl
\begin{defn}
    Let $D = (Q, \sigma, \delta, q_0, A)$ be a DFA. Define the relation $\sim_D$ (read ``equivalent" wrt $D$) on $\Sigma^*$ as

    $$x \sim_D y \iff \hd(q_0, x) = \hd(q_0, y)$$
\end{defn}

\begin{defn}
    Let $L \subseteq \Sigma^*$ be \underline{any} language. Define the relation $=_L$ (read ``equivalent" wrt $L$) on $\Sigma^*$ as:

    $x =_L y \iff \forall z \in \Sigma^*$, either $L$ contains both $xz, yz$ or $L$ doesn't contain any of $xz, yz$.
\end{defn}

\section{Content}

Consider the language $L$ from the last lecture.

\begin{claim}
    $L$ is not regular.
\end{claim}

\begin{proof}
    If $L$ is regular, then $L' = L \cap \L(ab^*c^*)$ is regular. $L' = \{ab^nc^n \mid n \in \mathbb{N} \cup \{0\}\}$ - this is not regular by the pumping lemma.
\end{proof}

\begin{ques}
    If $L$ is regular, what is your opponent's strategy?
\end{ques}

\begin{ans}
    $L = \L(D)$ for some DFA $D = (Q, \Sigma, \delta, q_0, A)$. Choose $p = |Q|$. Run $D$ on $s$ which is given in step 2. Find the earliest revisit to some state say $q$. The first revisit
    guarantees $|xy| \le p$, the fact that there is a revisit implies $|y| > 0$, and DFA gives that the resulting string is in $L$.
\end{ans}

Now we show a way to characterize the class of regular languages (necessary and sufficient conditions). As a benefit, if $L$ is regular, it gives a systematic way to construct a DFA $D$ for it,
and it also gives us the minimum possible number of states among all DFAs that recognize $L$.\nl
All DFAs that recognize $L$ and have no more states than $D$ are actually isomorphic to $D$.

\begin{claim}
    If $x \sim_D y$ then $\forall z \in \Sigma^*$, we have $xz \sim_D yz$.
\end{claim}

\begin{proof}
    $\hd(q_0, xz) = \hd(\hd(q_0, x), z) = \hd(\hd(q_0, y), z) = \hd(q_0, yz)$.
\end{proof}

\begin{claim}
    If $x \sim_D y$ then for all $z \in \Sigma^*$, $D$ either accepts both $xz, yz$ or $D$ rejects $xz, yz$.
\end{claim}

\begin{proof}
    Follows from the claim above and the fact that the state is either accepting or rejecting.
\end{proof}

\begin{claim}
    $\sim_D$ is an equivalence relation.
\end{claim}

\begin{proof}
    Exercise.
\end{proof}

\begin{claim}
    $\sim_L$ is an equivalence relation.
\end{claim}
\begin{proof}
    Recall COL202.
\end{proof}

\begin{claim}
    If $x =_L y$, then $\forall z \in \Sigma^*, xz =_L yz$.
\end{claim}

\begin{proof}
    Induction on $|z|$.
\end{proof}

\begin{eg}
    Let $\Sigma = \{0, 1\}$, $L = \{x \mid x \text{ is the binary representation of a multiple of } 7\}$
\end{eg}

\begin{ques}
    How does $=_L$ look like? When is $x =_L y$?
\end{ques}

\begin{ans}
    $x =_L y \iff x \equiv y \pmod 7$.
        \begin{proof}
            Suppose $x \equiv y \pmod 7$. Then $xz = x \cdot 2^{|z|} + z \equiv y \cdot 2^{|z|} + z \equiv yz \pmod 7$.
            Now suppose $x =_L y$. Suppose $x \not\equiv y \pmod 7$. Then $x \times 2^3 \not\equiv y \times 2^3 \pmod 7$. So $\exists z, |z| \le 3$ such that $x \times 2^3 + z$ is
            divisible by $7$ but $y \times 2^3 + z$ is not
            divisible by $7$. Considering $x 0^{3-|z|}z$ is in $L$ but $y 0^{3-|z|}z$ is not in $L$. So $x \not=_L y$.
        \end{proof}
\end{ans}

\begin{theorem}
    Let $L \subseteq \Sigma^*$ be a regular language and $D$ be a DFA for $L$. Then $\forall x, y \in \Sigma^*$, $x \sim_D y \implies x =_L y$. In other words, $\sim_D$ refines $=_L$.
\end{theorem}

\begin{proof}
    Follows from the previous claim and definition of $=_L$.
\end{proof}

\begin{theorem}
    \textbf{Myhill-Nerode Theorem (John Myhill, Anil Nerode, 1958)}\nl

\end{theorem}

\begin{proof}

\end{proof}

\end{document}
