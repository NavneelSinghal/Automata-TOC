\documentclass[a4paper]{article}
\usepackage[english]{babel}
\usepackage[a4paper,top=2cm,bottom=2cm,left=2cm,right=2cm,marginparwidth=1.75cm]{geometry}
\usepackage{amsmath}
\usepackage{amsfonts}
\usepackage{amsthm}
\usepackage{amssymb}
\usepackage{graphicx}
\usepackage[colorinlistoftodos]{todonotes}
\usepackage[colorlinks=true, allcolors=blue]{hyperref}
\usepackage{import}
\usepackage{pdfpages}
\usepackage{transparent}
\usepackage{xcolor}
\usepackage{algorithmicx}
\usepackage{algpseudocode}

\setlength{\parindent}{0pt}

\newtheorem{theorem}{Theorem}[section]
\newtheorem{lemma}{Lemma}
\newtheorem{claim}{Claim}
\newtheorem{defn}{Definition}
\newtheorem{notn}{Notation}
\newtheorem{obs}{Observation}
\newtheorem{eg}{Example}

\renewcommand\qedsymbol{$\blacksquare$}
\newcommand{\nl}{\vspace{0.2cm}\\}

\newcommand{\incfig}[1]{%
    \def\svgwidth{\columnwidth}
    \import{./figures/}{#1.pdf_tex}
}
\pdfsuppresswarningpagegroup=1

\title{\textbf{COL352 Lecture 2}}
\date{}

\begin{document}
\maketitle
\tableofcontents

\iffalse

\section{Logistics}
\subsection{Lectures}
\begin{enumerate}
    \item Timings - Monday 11AM, Wednesday 11AM, Thursday 12PM
    \item Mode - synchronous, MS Teams
\end{enumerate}
\section{The Halting Problem}
\subsection{Statement}
Does there exist a program $H$ that always halts, and given inputs a program $P$ and an input $i$, determine whether $P$ terminates/halts on $i$?
\subsection{Answer}
No, there does not exist such a program.
\subsection{Proof}
Suppose there does exist such a program $H$.\nl
Consider the following program $C$:
\begin{algorithmic}[1]
    \Function{$C$}{$P$}
        \If{\Call{$H$}{$P, P$}}
            \State run an infinite loop
        \Else
            \State \Return
        \EndIf
    \EndFunction
\end{algorithmic}
Now consider what happens when we call $C$ on the input $C$.\nl
Suppose $C(C)$ doesn't halt. Then this means that $H(C, C)$ is false (by the definition of $H$).
Hence, by line 2, $C(C)$ must run forever, and this gives a contradiction.\nl
Hence $C(C)$ must halt. But in that case, we have $H(C, C)$ to be true, and by line $2$, we have a contradiction yet again.\nl
Hence our assumption that such a program $H$ exists is false, and we have proved the claim. $\blacksquare$.

\fi

% start here

\section{Goal of this lecture}

To mathematically define a computational problem

\section{Deterministic Finite Automata}

\begin{notn}
    An alphabet is a finite set $\Sigma$ of symbols.
\end{notn}

\begin{notn}
By $\Sigma^n$, we the set of strings of length $n$ constructed from symbols in $\Sigma$.
\end{notn}

When $n = 0$, just to maintain consistency, we define $\epsilon$ to be an empty string, and $\Sigma^0 = \{\epsilon\}$.\\

\begin{notn}
$\Sigma^*$ is defined to be $\bigcup_{i = 0}^\infty \Sigma^i$.
\end{notn}

\begin{claim}
    The set $\Sigma^*$ is in fact countably infinite.
\end{claim}

\begin{proof}
    Note that there are precisely $|\Sigma|^n$ strings of length $n$.\\

    Consider any ordering of strings in $\Sigma^n$ and map one string each to each element in the set $\{1 + \sum_{i = 0}^{n - 1}|\Sigma|^i, \ldots, \sum_{i = 0}^m |\Sigma|^i\}$ (it helps to
    visualize this as chunks of the integer number line).\\

    Note that the range is a partition of $\mathbb{N}$, and thus, this map is a bijection from $\Sigma^*$ to $\mathbb{N}$, whence we are done.\\

    Constructing an injective map from $\Sigma^*$ to $\mathbb{N}$ is enough though.
\end{proof}

\begin{defn}
    A language over $\Sigma$ is any subset of $\Sigma^*$.
\end{defn}

\begin{notn}
    If $x, y$ are strings in $\Sigma^*$ (i.e. over $\Sigma$), we denote by $x \cdot y$ (or more simply, $xy$), the concatenation of $x$ and $y$.
\end{notn}
\begin{notn}
    If $L_1, L_2$ are languages over $\Sigma$, then by $L_1 \cdot L_2$ (or more simply $L_1 L_2$) we denote the set $\{s_1 s_2 \mid s_1 \in L_1, s_2 \in L_2\}$.
\end{notn}

\begin{defn}
    Let $\Gamma, \Sigma$ be any two finite non-empty sets. A computational problem is defined as a function $f : \Gamma^* \to \Sigma^*$.
\end{defn}

\begin{eg}
    Suppose $\Sigma = \{0, 1, \ldots, 9\}$, and $\Gamma = \{0, 1, \ldots, 9, \times\}$.
    Define $f(x)$ to be the prime factorization of a number here (lexicographically maximal, with least length), considering $0$ to be a prime.
    Then $f$ is a computational problem.
\end{eg}

\begin{eg}
    Let $\Sigma = \Gamma = \{0, 1\}$. Define 
    \begin{align*}
        f(x) = \begin{cases}
            1 \quad \text{ if } $x$ \text{ represents the adjacency matrix of a connected graph in row major order}\\
            0 \quad \text{ otherwise}
        \end{cases}
    \end{align*}
    Then $f$ is a computational problem.
\end{eg}

\begin{eg}
    A Python program is also a computational problem (output is the set of all possible bytecodes and syntax errors)
\end{eg}

\begin{defn}
    A decision problem is a function from $\Sigma^*$ to $\{0, 1\}$.
\end{defn}

\begin{eg}
    The second last example in the previous definition is a decision problem.
\end{eg}

Associate with any decision problem $f : \Sigma^* \to \{0, 1\}$ the language $L = \{x \in \Sigma^* \mid f(x) = 1\}$.\\
Conversely, we can associate a decision problem with every language (does $x$ belong to the language $L$?).\\
Hence decision problems are equivalent to languages (in some sense).

\begin{eg}
    $\Sigma = \{0, 1\}$, $f(x) = 1$ if $x$ is the binary representation of a number divisible by $5$, and 0 otherwise. $L = \{x \in \Sigma^* \mid f(x) = 1\}$.
\end{eg}

How do we solve this problem?\\

Suppose $x = x[1 \ldots n]$. Then we can just keep track of $x[1 \ldots i] \pmod 5$ as follows:

\begin{algorithmic}
    \Function{Solve}{$x[1 \ldots n]$}
        \State $q \gets 0$
        \For{$i \in \{1 \ldots n\}$}
            \If{$x[i] = 0$}
                \State $q \gets 2q \pmod 5$
            \Else
                \State $q \gets 2q + 1 \pmod 5$
            \EndIf
        \EndFor
        \If{$q = 0$}
            \State \Return True
        \Else
            \State \Return False
        \EndIf
    \EndFunction
\end{algorithmic}

Another way of looking at this is to construct a node for all possible values of $q$, the set of which is $\mathbb{Z}_5$. 

\begin{figure}[ht]
    \centering
    \incfig{modfsm}
    \caption{Automaton for the modulo 5 check}
    \label{fig:modfsm}
\end{figure}

\begin{eg}
    $\Sigma = \{a, b\}, L = \{x \in \Sigma^* \mid x \text{ ends in } ab\}$.
\end{eg}

Idea: keep track of last two characters used.

\newpage

\begin{figure}[ht]
    \centering
    \incfig{lasttwofsm}
    \caption{Automaton for checking last two characters}
    \label{fig:lasttwofsm}
\end{figure}

\begin{defn}
    A deterministic finite automaton (DFA) is a 5-tuple $(Q, \Sigma, \delta, q_0, A)$ where
    \begin{itemize}
        \item $Q$ is a finite nonempty set, called the \underline{set of states}.
        \item $\Sigma$ is a finite nonempty alphabet.
        \item $\delta$ is a function $\delta : Q \times \Sigma \to Q$, called the \underline{transition function}.
        \item $q_0$ is the \underline{initial state}.
        \item $A$ is the set of all \underline{accepting states}. 
    \end{itemize}
\end{defn}

\end{document}
