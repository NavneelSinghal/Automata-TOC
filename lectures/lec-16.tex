\documentclass[a4paper]{article}
\usepackage[english]{babel}
\usepackage[a4paper,top=2cm,bottom=2cm,left=2cm,right=2cm,marginparwidth=1.75cm]{geometry}
\usepackage{amsmath}
\usepackage{amsfonts}
% \usepackage{amsthm}
\usepackage{amssymb}
\usepackage{graphicx}
\usepackage[colorinlistoftodos]{todonotes}
\usepackage[colorlinks=true, allcolors=blue]{hyperref}
\usepackage{import}
\usepackage{pdfpages}
\usepackage{transparent}
\usepackage{xcolor}
\usepackage{algorithmicx}
\usepackage{algpseudocode}

\usepackage{thmtools}
\usepackage{enumitem}
\usepackage[framemethod=TikZ]{mdframed}

\usepackage{xpatch}

\usepackage{boites}
\makeatletter
\xpatchcmd{\endmdframed}
{\aftergroup\endmdf@trivlist\color@endgroup}
{\endmdf@trivlist\color@endgroup\@doendpe}
{}{}
\makeatother

%\usepackage[poster]{tcolorbox}
%\allowdisplaybreaks
%\sloppy

\usepackage[many]{tcolorbox}

\xpatchcmd{\proof}{\itshape}{\bfseries\itshape}{}{}

% to set box separation
\setlength{\fboxsep}{0.8em}
\def\breakboxskip{7pt}
\def\breakboxparindent{0em}

\newenvironment{proof}{\begin{breakbox}\textit{Proof.}}{\hfill$\square$\end{breakbox}}
\newenvironment{ans}{\begin{breakbox}\textit{Answer.}}{\end{breakbox}}
\newenvironment{soln}{\begin{breakbox}\textit{Solution.}}{\end{breakbox}}

% \tcolorboxenvironment{proof}{
%     blanker,
%     before skip=\topsep,
%     after skip=\topsep,
%     borderline={0.4pt}{0.4pt}{black},
%     breakable,
%     left=12pt,
%     right=12pt,
%     top=12pt,
%     bottom=12pt,
% }
% 
% \tcolorboxenvironment{ans}{
%     blanker,
%     before skip=\topsep,
%     after skip=\topsep,
%     borderline={0.4pt}{0.4pt}{black},
%     breakable,
%     left=12pt,
%     right=12pt,
% }

\mdfdefinestyle{enclosed}{
    linecolor=black
    ,backgroundcolor=none
    ,apptotikzsetting={\tikzset{mdfbackground/.append style={fill=gray!100,fill opacity=.3}}}
    ,frametitlefont=\sffamily\bfseries\color{black}
    ,splittopskip=.5cm
    ,frametitlebelowskip=.0cm
    ,topline=true
    ,bottomline=true
    ,rightline=true
    ,leftline=true
    ,leftmargin=0.01cm
    ,linewidth=0.02cm
    ,skipabove=0.01cm
    ,innerbottommargin=0.1cm
    ,skipbelow=0.1cm
}

\mdfsetup{%
    middlelinecolor=black,
    middlelinewidth=1pt,
roundcorner=4pt}

\setlength{\parindent}{0pt}

\mdtheorem[style=enclosed]{theorem}{Theorem}
\mdtheorem[style=enclosed]{lemma}{Lemma}[theorem]
\mdtheorem[style=enclosed]{claim}{Claim}[theorem]
\mdtheorem[style=enclosed]{ques}{Question}
\mdtheorem[style=enclosed]{defn}{Definition}
\mdtheorem[style=enclosed]{notn}{Notation}
\mdtheorem[style=enclosed]{obs}{Observation}
\mdtheorem[style=enclosed]{eg}{Example}
\mdtheorem[style=enclosed]{cor}{Corollary}
\mdtheorem[style=enclosed]{note}{Note}

% \let\thetheorem=\relax
% \let\thelemma=\relax
% \let\theclaim=\relax
% \let\theques=\relax
% \let\thedefn=\relax
% \let\thenotn=\relax
% \let\theobs=\relax
% \let\thecor=\relax
% \let\thenote=\relax

% \renewcommand\qedsymbol{$\blacksquare$}
\newcommand{\nl}{\vspace{0.2cm}\\}
\newcommand{\mc}{\mathcal}
\newcommand{\mi}{\mathit}
\newcommand{\mf}{\mathbf}
\newcommand{\mb}{\mathbb}
\renewcommand{\L}{\mc{L}}
\newcommand{\hd}{\hat{\delta}}
\newcommand{\produces}{\implies}
\newcommand{\derives}{\stackrel{*}{\implies}}
\newcommand{\changesto}{\vdash}


\newcommand{\incfig}[1]{%
    \def\svgwidth{\columnwidth}
    \import{./figures/}{#1.pdf_tex}
}
\pdfsuppresswarningpagegroup=1

\title{\textbf{COL352 Lecture 16}}
\date{}

\begin{document}
\maketitle
\tableofcontents

\section{Recap}

Discussion about grammars, parse trees, yields, and context free languages.

\section{Definitions}

\begin{defn}
    A (non-deterministic) pushdown automaton ((N)PDA) is a 6-tuple $(Q, \Sigma, \Gamma, \Delta, q_0, A)$ where
    \begin{enumerate}
        \item $Q$ -- finite nonempty set of states
        \item $\Sigma$ -- finite nonempty input alphabet
        \item $\Gamma$ -- finite stack alphabet
        \item $q_0 \in Q$ -- initial state
        \item $A$ -- set of accepting states
        \item $\Delta \subseteq Q \times \Sigma_\epsilon \times \Gamma_\epsilon \times Q \times \Gamma_\epsilon$, where $X_\epsilon$ is defined as $X \cup \{\epsilon\}$
    \end{enumerate}
\end{defn}

Note that in an NFA, $\Delta \subseteq Q \times (\Sigma \cup \{\epsilon\}) \times Q$.

Informally, a string is accepted if $\exists$ a run which reads the entire string and does what ?????\nl

\begin{defn}
    Let $P = (Q, \Sigma, \Gamma, \Delta, q_0, A)$ be a PDA. An instantaneous description (i.d.) of $P$ is a tuple $(q, x, \alpha)$ where $q \in Q$, $x \in \Sigma^*$, $\alpha \in \Gamma^*$. (The set
    of instantaneous descriptions is $Q \times \Sigma^* \times \Gamma^*$).
\end{defn}

Informally, it consists of the current state, the string left to be read, and the description of the current stack.\nl

\begin{defn}
    Let $P = (Q, \Sigma, \Gamma, \Delta, q_0, A)$ be a PDA. The relation $\changesto_P$ (read as ``changes to") is defined on the set of i.d.s as follows:\nl
    If $(q, a, B, q', B') \in \Delta$, then $(q, ax, B\alpha) \changesto_P (q', x, B'\alpha)$, and no other pairs of i.d.s are related.\nl
    In other words:
    \begin{align*}
        (q, y, \beta) \changesto_P (q', y', \beta') &\iff\\ &\exists a \in \Sigma_\epsilon, B \in \Gamma_\epsilon, \alpha \in \Gamma^*, B' \in \Gamma_\epsilon \text{ such that } y = ay', \beta =
    B\alpha, \beta' = B'\alpha, (q, a, B, q', B') \in \Delta
    \end{align*}
\end{defn}

\begin{defn}
    $\changesto^*_P$ is defined as the reflexive transitive closure of $\changesto$ (read as ``changes to in finitely many steps").
\end{defn}

\begin{defn}
    $x \in \Sigma^*$ is said to be accepted by PDA $P = (Q, \Sigma^*, \Gamma, \Delta, q_0, A)$ iff
    $$(q_0, x, \epsilon) \changesto^*_P (q, \epsilon, \alpha)$$
    for some $q \in A$ and some $\alpha \in \Gamma^*$.
\end{defn}

\section{Content}

We begin with an informal discussion first.\nl

Note that we look at two kinds of objects - languages and recognizers.\nl

\begin{center}
\begin{tabular}{|c|c|c|}
    \hline
    Language Class  & Recognizer                            & Generator             \\
    \hline
    Regular         & DFA/NFA                               & Regular expressions   \\
    \hline
    Context-free    & Non deterministic pushdown automaton  & Grammar               \\
    \hline
\end{tabular}
\end{center}

Note that unlike DFA/NFAs (which have the same power), NPDAs are more powerful that DPDAs.\nl

An NFA looks like a tape where we have a read head pointing to a location in the string, with some state associated to the read head, and transitions can either go to the next position (in the
string) or remain at the same position.\nl

A PDA is similar to this, except that it has a stack as well.\nl

$(q, a, B, q', B')$ is a transition in a PDA.

\begin{enumerate}
    \item $q$ is the current state.
    \item $a$ is the current input.
    \item $B$ is the top of the stack that can be popped ($\epsilon$ if not popping).
    \item $q'$ is the next state.
    \item $B'$ is the top of the stack after pushing $B'$ on the stack ($\epsilon$ if we don't push anything on the stack).
\end{enumerate}

\begin{claim}
    A PDA can recognize $\{a^n b^n \mid n \in \mb{N} \cup \{0\}\}$.
\end{claim}

Intuition: try to think in terms of the stack. When is it empty? Do we need it to be empty?

See definitions section for the formal definition.

\begin{eg}
    $\Sigma = \{0, 1\}$, $L$ is the set of all palindromes.
    $\Gamma = \{0, 1, \bot\}, Q = \{q_0, q_1, q_2, q_3\}$. Define in the usual sense (see lecture).
\end{eg}

\begin{claim}
    $x \in L \iff (q_0, x, \epsilon) \changesto_P^* (q_3, \epsilon, \epsilon)$.
\end{claim}

\begin{proof}
    Exercise.
\end{proof}

\end{document}
