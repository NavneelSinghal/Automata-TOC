\documentclass[a4paper]{article}
\usepackage[english]{babel}
\usepackage[a4paper,top=2cm,bottom=2cm,left=2cm,right=2cm,marginparwidth=1.75cm]{geometry}
\usepackage{amsmath}
\usepackage{amsfonts}
% \usepackage{amsthm}
\usepackage{amssymb}
\usepackage{graphicx}
\usepackage[colorinlistoftodos]{todonotes}
\usepackage[colorlinks=true, allcolors=blue]{hyperref}
\usepackage{import}
\usepackage{pdfpages}
\usepackage{transparent}
\usepackage{xcolor}
\usepackage{algorithmicx}
\usepackage{algpseudocode}

\usepackage{thmtools}
\usepackage{enumitem}
\usepackage[framemethod=TikZ]{mdframed}

\usepackage{xpatch}

\usepackage{boites}
\makeatletter
\xpatchcmd{\endmdframed}
{\aftergroup\endmdf@trivlist\color@endgroup}
{\endmdf@trivlist\color@endgroup\@doendpe}
{}{}
\makeatother

%\usepackage[poster]{tcolorbox}
%\allowdisplaybreaks
%\sloppy

\usepackage[many]{tcolorbox}

\xpatchcmd{\proof}{\itshape}{\bfseries\itshape}{}{}

% to set box separation
\setlength{\fboxsep}{0.8em}
\def\breakboxskip{7pt}
\def\breakboxparindent{0em}

\newenvironment{proof}{\begin{breakbox}\textit{Proof.}}{\hfill$\square$\end{breakbox}}
\newenvironment{ans}{\begin{breakbox}\textit{Answer.}}{\end{breakbox}}
\newenvironment{soln}{\begin{breakbox}\textit{Solution.}}{\end{breakbox}}

% \tcolorboxenvironment{proof}{
%     blanker,
%     before skip=\topsep,
%     after skip=\topsep,
%     borderline={0.4pt}{0.4pt}{black},
%     breakable,
%     left=12pt,
%     right=12pt,
%     top=12pt,
%     bottom=12pt,
% }
% 
% \tcolorboxenvironment{ans}{
%     blanker,
%     before skip=\topsep,
%     after skip=\topsep,
%     borderline={0.4pt}{0.4pt}{black},
%     breakable,
%     left=12pt,
%     right=12pt,
% }

\mdfdefinestyle{enclosed}{
    linecolor=black
    ,backgroundcolor=none
    ,apptotikzsetting={\tikzset{mdfbackground/.append style={fill=gray!100,fill opacity=.3}}}
    ,frametitlefont=\sffamily\bfseries\color{black}
    ,splittopskip=.5cm
    ,frametitlebelowskip=.0cm
    ,topline=true
    ,bottomline=true
    ,rightline=true
    ,leftline=true
    ,leftmargin=0.01cm
    ,linewidth=0.02cm
    ,skipabove=0.01cm
    ,innerbottommargin=0.1cm
    ,skipbelow=0.1cm
}

\mdfsetup{%
    middlelinecolor=black,
    middlelinewidth=1pt,
roundcorner=4pt}

\setlength{\parindent}{0pt}

\mdtheorem[style=enclosed]{theorem}{Theorem}
\mdtheorem[style=enclosed]{lemma}{Lemma}[theorem]
\mdtheorem[style=enclosed]{claim}{Claim}[theorem]
\mdtheorem[style=enclosed]{ques}{Question}
\mdtheorem[style=enclosed]{defn}{Definition}
\mdtheorem[style=enclosed]{notn}{Notation}
\mdtheorem[style=enclosed]{obs}{Observation}
\mdtheorem[style=enclosed]{eg}{Example}
\mdtheorem[style=enclosed]{cor}{Corollary}
\mdtheorem[style=enclosed]{note}{Note}

% \let\thetheorem=\relax
% \let\thelemma=\relax
% \let\theclaim=\relax
% \let\theques=\relax
% \let\thedefn=\relax
% \let\thenotn=\relax
% \let\theobs=\relax
% \let\thecor=\relax
% \let\thenote=\relax

% \renewcommand\qedsymbol{$\blacksquare$}
\newcommand{\nl}{\vspace{0.2cm}\\}
\newcommand{\mc}{\mathcal}
\newcommand{\mi}{\mathit}
\newcommand{\mf}{\mathbf}
\newcommand{\mb}{\mathbb}
\renewcommand{\L}{\mc{L}}
\newcommand{\hd}{\hat{\delta}}
\newcommand{\produces}{\implies}
\newcommand{\derives}{\stackrel{*}{\implies}}
\newcommand{\changesto}{\vdash}


\newcommand{\incfig}[1]{%
    \def\svgwidth{\columnwidth}
    \import{./figures/}{#1.pdf_tex}
}
\pdfsuppresswarningpagegroup=1

\title{\textbf{COL352 Lecture 22}}
\date{}

\begin{document}
\maketitle
\tableofcontents

\section{Recap}

Proof that PDA = grammars completed.

\section{Definitions}

\begin{defn}
We define a \underline{simple PDA} $P$ to be a PDA such that
\begin{enumerate}
    \item $\Delta = \Delta_{push} \uplus \Delta_{pop}$, where
        \begin{enumerate}
            \item $\Delta_{push}$ contains transitions $(q, a, \epsilon, q', B)$ where $q, q' \in Q, a \in \Sigma_\epsilon, B \in \Gamma$ (i.e., not allowed to pop, must push), and
            \item $\Delta_{pop}$ contains transitions $(q, a, B, q', \epsilon)$ where $q, q' \in Q, a \in \Sigma_\epsilon, B \in \Gamma$ (i.e., must pop, not allowed to push).
        \end{enumerate}
        \item $|A| = 1$, i.e., unique accepting state.
        \item If $x$ is accepted, then $x$ is accepted with an empty stack, i.e., $(q_{init}, x, \epsilon) \changesto^* (q_{acc}, \epsilon, \alpha)$ for some $\alpha \in \Sigma^*$ iff $q_{init},
            x, \epsilon) \changesto^* (q_{acc}, \epsilon, \epsilon)$.
\end{enumerate}
\end{defn}

\section{Content}

\begin{ques}
    What kinds of closure properties can we think of for context free languages? Union? Intersection? Complementation? Concatenation? $*$?
\end{ques}

\begin{ans}
    Let $G = (N, \Sigma, R, S)$.

    For $*$: $G' = (N \uplus \{T\}, \Sigma, R \cup \{T \to TT, T \to S, T \to \epsilon\})$ generates $L^*$.

    Let $G = (N, \Sigma, R, S)$.
    
    For $\cup: G_1 = (N_1, \Sigma, R_1, S_1)$ generates $L_1$ and $G_2 = (N_2, \Sigma, R_2, S_2)$ generates $L_2$, then
    $(N_1 \uplus N_2 \uplus T, \Sigma, R_1 \cup R_2 \cup \{T \to S_1, T \to S_2\}, T)$ generates $L_1 \cup L_2$.\nl
    $(N_1 \uplus N_2 \uplus T, \Sigma, R_1 \cup R_2 \cup \{T \to S_1 S_2\}, T)$ generates $L_1 L_2$.\nl

    Not closed under intersection (would imply complementation by contradiction and contrapositive).\nl

    Let $L_1 = \{a^nb^nc^* \mid n \in \mb{N} \cup \{0\}\}$, and $L_2 = \{a^*b^nc^n \mid n \in \mb{N} \cup \{0\}\}$. Then $L_1 \cap L_2 = \{a^nb^nc^n \mid n \in \mb{N} \cup \{0\}\}$, which is
    probably not context free. We'll use a version of the pumping lemma for the DFA.
\end{ans}

\begin{note}
    Suppose $L$ is a context free language generated by a grammar $G$. Suppose $w \in L$ is a ``long enough" string. Consider a smallest parse tree $T$ of $w$. Since $w$ is ``long enough", $T$ is ``tall
    enough". Look at the longest root-to-leaf path in $T$, say $p$.\nl
    Since $p$ is long enough, some two nodes on $P$ are labelled by the same non-terminal, say $A \in N$. Let $uw'z$ be $w$ such that the upper $A$ derives $w'$, and let $vxy$ be $w'$. Look at tree
    (very helpful). Break it into $S, A$'s tree (having $u, z$), $A, A$'s tree (having $v, y$) and $A$'s tree (having $x$).\nl
    By copy pasting the second tree into itself again and again, we can get $\forall i : uv^ixy^iz \in L$. This doesn't give us anything if $y = v = \epsilon$, so we enforce smallest tree
    constraints.\nl
    How tall is tall enough? If number of non-terminals is $k$, then we need height $\ge k + 1$.\nl
    How long is long enough? If $d = \max(\{|\alpha| \mid A \to \alpha \in R\})$, then we need $d^{k+1}$.
\end{note}

\begin{theorem}
    \textbf{(Pumping lemma for Context-free languages)}\nl
    For every context-free language $L \subseteq \Sigma^*$, there exists $p \in \mb{N}$ such that for every $w \in L$ with $|w| \ge p$, there exist strings $u, v, x, y, z \in \Sigma^*$ such that
    \begin{enumerate}
        \item $w = uvxyz$
        \item $|vy| > 0$
        \item $|vxy| \le p$
        \item $\forall i \in \mb{N} \cup \{0\}$, we have $uv^ixy^iz \in L$.
    \end{enumerate}
\end{theorem}
\begin{proof}
        $L = \L(G)$ for some grammar $G = (N, \Sigma, R, S)$. Let $k = |N|$, $d = \max(\{|\alpha| \mid A \to \alpha \in R\})$.\nl
        Choose $p = d^{k+1}$.\nl
        For any $w \in L$ with $|w| \ge p$, consider the smallest parse tree $T$ of $w$. Height of $T$ is at least $k + 1$. Let $P$ be a longest root-to-leaf path in $T$, so $P$ has $\ge k + 1$
        non-leaf vertices. Consider the bottom-most $k + 1$ non-leaf vertices of $P$. There exist two vertices $l_1, l_2$ among those, whose label is the same non-terminal, say $A \in N$.\nl
        The yield of $l_1$ in this tree is some substring $w'$ of $w$, i.e., $w = uw'z$ for some $u, z \in \Sigma^*$.\nl
        The yield of $l_2$ in this tree is some substring $x$ of $w'$, i.e., $w' = vxy$ for some $v, y \in \Sigma^*$.\nl
        So, we have $w = uw'z = uvxyz$. $|vy| > 0$ by the minimality of $T$.\nl
        The height of the tree rooted at $l_1$ is at most $\le k + 1$. Therefore, $|vxy| = |w'| \le d^{k+1} = p$.\nl
        Observe that $S \derives uAz$ (chop off the children of $l_1$ in $T$) and $A \derives vAy$ (let $T'$ be the subtree of $T$ rooted at $l_1$, and chop off the children of $l_2$ in $T'$), and
        $A \derives x$ (let $T''$ be the subtree of $T$ rooted at $l_2$).\nl
        $\forall i \in \mb{N} \cup \{0\}$, we have $S \derives uAz \produces uv^iAy^iz$ (by repeated application of the second relation), and using the last relation, we have $S \derives
        uv^ixy^iz$.
\end{proof}

The pumping game for context free languages $L \subseteq \Sigma^*$.

\begin{tabular}{|c|c|}
    Your opponent & You \\
    $p\in \mb{N}$ & \\
     & String $w$ with $|w| \ge p$, and $w \in L$  \\
    $u, v, x, y, z \in \Sigma^*$ such that $w = uvxyz, |vy| > 0, |vxy| \le p$ &  \\
                                                                                & $i \in \mb{N} \cup\{0\}$ \\
\end{tabular}

You win if $uv^ixy^iz \not\in L$, else your opponent wins.

The pumping lemma says that if $L$ is context-free, then your opponent has a winning strategy.

And if you have a winning strategy, then $L$ is not context-free.

\begin{eg}
\begin{tabular}{|c|c|}
    Your opponent & You \\
    $p$ &  \\
     & $a^pb^pc^p$ \\
    $u, v, x, y, z \in \Sigma^*$ such that $w = uvxyz, |vy| > 0, |vxy| \le p$ & so $vxy$ is either a substring of $a^pb^p$ or $b^pc^p$ \\
     & Choose any $i \ne 1$ and reach a contradiction wrt number of $a, c$. \\
\end{tabular}
\end{eg}

\begin{claim}
    $L = \{a^nb^nc^n \mid n \in \mb{N} \cup \{0\}\}$ is not context free.
\end{claim}

\begin{proof}
    As above.
\end{proof}

\begin{cor}
    The class of context-free languages is not closed under intersection.
\end{cor}

\begin{proof}
    Let $L_1 = \{a^mb^mc^n \mid m, n \in \mb{N} \cup \{0\}\}$ and $L_2 = \{a^mb^nc^n \mid m, n \in \mb{N} \cup \{0\}\}$. Both are context-free but $L_1 \cap L_2 = L$ is not context-free.
\end{proof}

\begin{cor}
    The class of context-free languages is not closed under complementation.
\end{cor}

\begin{proof}
    (Indirect): $L_1 \cap L_2 = (L_1^c \cup L_2^c)^c$, and we are done using the previous lemma.\nl
    (Direct): Directly show that $L^c$ is not context free for some language $L$.
\end{proof}

\begin{note}
    Note that the language $\{w \cdot rev(w) \mid w \in \Sigma^*\}$ is context-free (can make a PDA for this).
\end{note}

\begin{claim}
    The language $\{ww \mid w \in \Sigma^*\}$ is not context-free.
\end{claim}

\begin{proof}
 \begin{tabular}{|c|c|}
    Your opponent & You \\
    $p$ &  \\
     & $a^pb^pa^pb^p$ \\
    $u, v, x, y, z \in \Sigma^*$ such that $w = uvxyz, |vy| > 0, |vxy| \le p$ & so $vxy$ is either a substring of $a^pb^p$ or $b^pa^p$ \\
     & Choose any $i \ne 1$ and reach a contradiction. \\
\end{tabular}
   
\end{proof}

\end{document}
