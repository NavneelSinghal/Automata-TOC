\documentclass[a4paper]{article}
\usepackage[english]{babel}
\usepackage[a4paper,top=2cm,bottom=2cm,left=2cm,right=2cm,marginparwidth=1.75cm]{geometry}
\usepackage{amsmath}
\usepackage{amsfonts}
\usepackage{amsthm}
\usepackage{amssymb}
\usepackage{graphicx}
\usepackage[colorinlistoftodos]{todonotes}
\usepackage[colorlinks=true, allcolors=blue]{hyperref}
\usepackage{import}
\usepackage{pdfpages}
\usepackage{transparent}
\usepackage{xcolor}
\usepackage{algorithmicx}
\usepackage{algpseudocode}

\setlength{\parindent}{0pt}

\newtheorem{theorem}{Theorem}[section]
\newtheorem{lemma}{Lemma}
\newtheorem{claim}{Claim}
\newtheorem{ans}{Answer}
\newtheorem{ques}{Question}
\newtheorem{defn}{Definition}
\newtheorem{notn}{Notation}
\newtheorem{obs}{Observation}
\newtheorem{eg}{Example}
\newtheorem{cor}{Corollary}
\newtheorem{note}{Note}

\renewcommand\qedsymbol{$\blacksquare$}
\newcommand{\nl}{\vspace{0.2cm}\\}
\newcommand{\mc}{\mathcal}
\newcommand{\mi}{\mathit}
\newcommand{\mf}{\mathbf}
\newcommand{\mb}{\mathbb}

\newcommand{\incfig}[1]{%
    \def\svgwidth{\columnwidth}
    \import{./figures/}{#1.pdf_tex}
}
\pdfsuppresswarningpagegroup=1

\title{\textbf{COL352 Lecture 5}}
\date{}

\begin{document}
\maketitle
\tableofcontents

\iffalse

\section{Logistics}
\subsection{Lectures}
\begin{enumerate}
    \item Timings - Monday 11AM, Wednesday 11AM, Thursday 12PM
    \item Mode - synchronous, MS Teams
\end{enumerate}
\section{Grading}
\begin{center}
\begin{tabular}{|c|c|}
    \hline
    Class participation     &  5\% \\
    \hline
    Quizzes (best 3 of 4-5) & 30\% \\
    \hline
    HW (best 25 out of 30+) & 25\% \\
    \hline
    Major exam	            & 40\% \\
    \hline
\end{tabular}
\end{center}
\section{The Halting Problem}
\subsection{Statement}
Does there exist a program $H$ that always halts, and given inputs a program $P$ and an input $i$, determine whether $P$ terminates/halts on $i$?
\subsection{Answer}
No, there does not exist such a program.
\subsection{Proof}
Suppose there does exist such a program $H$.\nl
Consider the following program $C$:
\begin{algorithmic}[1]
    \Function{$C$}{$P$}
        \If{\Call{$H$}{$P, P$}}
            \State run an infinite loop
        \Else
            \State \Return
        \EndIf
    \EndFunction
\end{algorithmic}
Now consider what happens when we call $C$ on the input $C$.\nl
Suppose $C(C)$ doesn't halt. Then this means that $H(C, C)$ is false (by the definition of $H$).
Hence, by line 2, $C(C)$ must run forever, and this gives a contradiction.\nl
Hence $C(C)$ must halt. But in that case, we have $H(C, C)$ to be true, and by line $2$, we have a contradiction yet again.\nl
Hence our assumption that such a program $H$ exists is false, and we have proved the claim. $\blacksquare$.

\fi

% start here


\section{Recap}

Recall the informal definition of a run on NFA. We formalize this today.

\section{Content}

\begin{defn}
    Let $N = (Q, \Sigma, \Delta, Q_0, A)$ be an NFA, and $x \in \Sigma^*$ be a string. A \underline{run} of $N$ on $x$ is a sequence $q_0, x_1, q_1, \ldots, x_m, q_m$ such that
    \begin{enumerate}
        \item Each $q_i \in Q$
        \item Each $x_i \in \Sigma \cup \{\epsilon\}$, such that $x = x_1x_2 \ldots x_m$.
        \item $q_0 \in Q_0$
        \item $\forall i : (q_{i - 1}, x_i, q_i) \in \Delta$
    \end{enumerate}
\end{defn}

\begin{defn}
    $q_0, x_1, \ldots, q_{m - 1}, x_m, q_m$ is an \underline{accepting run} if $q_m \in A$, otherwise it is a rejecting run.
\end{defn}

\begin{defn}
    $N$ accepts $x$ if \underline{there exists} an accepting run of $N$ on $x$.
\end{defn}

\begin{defn}
    $\mc{L}(N) = \{x \in \Sigma^* \mid N \text{ accepts } x\}$
\end{defn}

\begin{ques}
    Is it possible to create an NFA for any regular language?
\end{ques}

\begin{ans}
    Yes.
\end{ans}

\begin{claim}
    If $L$ is recognized by a DFA, then $L$ is recognized by some NFA.
\end{claim}
\begin{proof}
    
    Suppose $D = (Q, \Sigma, \delta, q_0, A)$ recognizes the language $L$.
    
    We construct $N = (Q, \Sigma, \Delta, \{q_0\}, A)$, with $\Delta = \{(q, a, q') \in Q \times \Sigma \times Q \mid q' = \delta(q, a)\}$.

    Then $N$ also recognizes $D$.
\end{proof}

\begin{ques}
    Is it possible to create a DFA for any language that is accepted by an NFA?
\end{ques}

\begin{ans}
    Yes.
\end{ans}

\begin{theorem}
    Let $N$ be any NFA. There exists a DFA $D$ such that $\mc{L}(N) = \mc{L}(D)$.
\end{theorem}

\begin{proof}

    We shall break this into two steps.

    \begin{enumerate}
        \item Step 1:
            \begin{theorem}
                Let $N$ be any NFA. There exists an NFA $N'$ without $\epsilon$ transitions such that $\mc{L}(N) = \mc{L}(D)$.
            \end{theorem}
        \item Step 2:
            \begin{theorem}
                Let $N'$ be any NFA without $\epsilon$ transitions. There exists a DFA $D$ such that $\mc{L}(N) = \mc{L}(D)$.
            \end{theorem}
    \end{enumerate}

    First we prove step 1.\\

    $x \in \mc{L}(N)$. Let $q_0 x_1 q_1 x_2 q_2 \ldots x_m q_m$ be an accepting run.\\

    $x = x[1] x[2] \ldots x[n]$, with $x[i] \in \Sigma$.\\

    $\langle x[i] \rangle$ is a subsequence of $\langle x_i \rangle$.\\

    For any $j \not \in \{i_1, i_2, \ldots, i_n\}$, $x_j = \epsilon$, and $x_{i_k} = x[k]$. (basically compression).

    High-level idea - compress $x_{i_{k - 1} + 1} \ldots x_{i_k}$ into a single step.\\

    Now coming to the official proof.\\

    Let $N = (Q, \Sigma, \Delta, Q_0, A)$. Define NFA $N'$ as follows:\\

    $$N' = (Q, \Sigma, \Delta', Q_0, A')$$

    Here, 

    $$\Delta' = \{(q, a, q') \in Q \times \Sigma \times Q \mid \exists q'' \in Q \text{ such that } q''  \text{ is reachable from } q \text{ by } \epsilon \text{ transitions and } (q'', a, q') \in \Delta \}$$

    (note: this corresponds to the compression in the high-level idea)\\

    $$A' = \{q \in Q \mid \exists q' \in A \text{ such that } q' \text{ is reachable from } q \text{ by } \epsilon \text{ transitions of } \Delta\}$$

    (note: this corresponds to compressing the last part of an accepting run)\\

    \begin{claim}
        $\mc{L}(N) \subseteq \mc{L}(N')$
    \end{claim}

    \begin{proof}
        Suppose $x \in \mc{L}(N)$. Then there exists an accepting run $q_0, x_1, q_1, \ldots, x_m, q_m$ of $N$ on $x$.\\

        Let $i_1 < i_2 < \cdots < i_n$ be all the indices such that $x_i \in \Sigma$ (i.e., if $j \not \in \{i_1, \ldots, i_n\}$, then $x_j = \epsilon$).\\

        Let's construct an accepting run of $N'$ on $x$. Consider the sequence $q_0 x[1] q_{i_1} x[2] \ldots x[n] q_{i_n}$.\\

        Now note that:

        \begin{enumerate}
            \item $(q_{i_{j - 1}}, x[j], q_{i_j}) \in \Delta'$
                \begin{proof}
                    This follows directly from the definition of $\Delta'$; indeed, in the original sequence, we have $q_{i_{j-1}} \epsilon q_{i_{j-1}+1} \epsilon \ldots \epsilon q){i_j - 1} x[j]
                    q_{i_j}$, so using $q = q_{i_{j-1}}, q'' = q_{i_j - 1}, q' = q_{i_j}$ in the definition, we are done.
                \end{proof}
            \item $q_{i_n} \in A'$
                \begin{proof}
                    Note that all transitions after this are $\epsilon$ transitions, and hence the accepting state $q_m$ is reachable from $q_{i_n}$ using $\epsilon$ transitions, and hence $q_{i_n} \in A'$.
                \end{proof}
        \end{enumerate}

        Hence we get an accepting run of $N'$ on $x$, whence we are done.
    \end{proof}

    \begin{claim}
        $\mc{L}(N') \subseteq \mc{L}(N)$
    \end{claim}

    \begin{proof}
        Exercise.\\

        Suppose $x \in \mc{L}(N')$. Then there exists an acepting run $q_0, x[1]', q_2, \ldots, x[n']', q_{n'}$ of $N'$ on $x$. Note that since there are no $\epsilon$ transitions, we have $x[i]'
        = x[i]$, and $n = n'$.\\

        Now consider any $(q_{i - 1}, x[i], q_i)$. Then since this is in $D'$, we must have some $q''$ such that $q''$ is reachable from $q_{i - 1}$ by $\epsilon$ transitions and $(q'', x[i], q_i) \in
        D$.\\
        
        Consider the corresponding path from $q_{i - 1}$ to $q''$, say $q_{i - 1} \epsilon q_{i - 1, 1}' \epsilon \ldots \epsilon q'' = q_{i - 1, l_{i - 1}}'$.\\

        Also note that the last state $q_n$ is an accepting state in $N'$, hence there exists a path starting from $q_n$ to some accepting state $q''$ in $A$ consisting solely of $\epsilon$
        transitions, say $q_n \epsilon q_{n, 1}' \epsilon \ldots \epsilon q_{n, l_n}' = q''$.\\

        Consider the path

        $$q_0 \epsilon q_{0, 1}' \epsilon \ldots \epsilon q_{0, l_0} x[1] q_1 \epsilon q_{1, 1} \epsilon \ldots \epsilon q_{1, l_1} x[2] q_2 \ldots q_n \epsilon q_{n, 1}' \epsilon \ldots \epsilon
        q_{n, l_n}'$$

        Then the last state $q_{n, l_n}'$ is accepting as discussed above, and all transitions are in $\Delta$ (by definition of $\Delta'$), and the string corresponding to this run is
        $\epsilon^{l_0} x[1] \epsilon^{l_1} x[2] \ldots \epsilon^{l_{n-1}} x[n] \epsilon^{l_n} = x$, as needed.\\
        
        Hence we get an accepting run of $N$ on $x$, whence we are done.

    \end{proof}

    Now these two claims together show that $\mc{L}(N') = \mc{L}(N)$, whence we have completed the proof of theorem 2.2.
\end{proof}

\end{document}
