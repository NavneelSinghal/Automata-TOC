\documentclass[a4paper]{article}
\usepackage[english]{babel}
\usepackage[a4paper,top=2cm,bottom=2cm,left=2cm,right=2cm,marginparwidth=1.75cm]{geometry}
\usepackage{amsmath}
\usepackage{amsfonts}
% \usepackage{amsthm}
\usepackage{amssymb}
\usepackage{graphicx}
\usepackage[colorinlistoftodos]{todonotes}
\usepackage[colorlinks=true, allcolors=blue]{hyperref}
\usepackage{import}
\usepackage{pdfpages}
\usepackage{transparent}
\usepackage{xcolor}
\usepackage{algorithmicx}
\usepackage{algpseudocode}

\usepackage{thmtools}
\usepackage{enumitem}
\usepackage[framemethod=TikZ]{mdframed}

\usepackage{xpatch}

\usepackage{boites}
\makeatletter
\xpatchcmd{\endmdframed}
{\aftergroup\endmdf@trivlist\color@endgroup}
{\endmdf@trivlist\color@endgroup\@doendpe}
{}{}
\makeatother

%\usepackage[poster]{tcolorbox}
%\allowdisplaybreaks
%\sloppy

\usepackage[many]{tcolorbox}

\xpatchcmd{\proof}{\itshape}{\bfseries\itshape}{}{}

% to set box separation
\setlength{\fboxsep}{0.8em}
\def\breakboxskip{7pt}
\def\breakboxparindent{0em}

\newenvironment{proof}{\begin{breakbox}\textit{Proof.}}{\hfill$\square$\end{breakbox}}
\newenvironment{ans}{\begin{breakbox}\textit{Answer.}}{\end{breakbox}}
\newenvironment{soln}{\begin{breakbox}\textit{Solution.}}{\end{breakbox}}

% \tcolorboxenvironment{proof}{
%     blanker,
%     before skip=\topsep,
%     after skip=\topsep,
%     borderline={0.4pt}{0.4pt}{black},
%     breakable,
%     left=12pt,
%     right=12pt,
%     top=12pt,
%     bottom=12pt,
% }
% 
% \tcolorboxenvironment{ans}{
%     blanker,
%     before skip=\topsep,
%     after skip=\topsep,
%     borderline={0.4pt}{0.4pt}{black},
%     breakable,
%     left=12pt,
%     right=12pt,
% }

\mdfdefinestyle{enclosed}{
    linecolor=black
    ,backgroundcolor=none
    ,apptotikzsetting={\tikzset{mdfbackground/.append style={fill=gray!100,fill opacity=.3}}}
    ,frametitlefont=\sffamily\bfseries\color{black}
    ,splittopskip=.5cm
    ,frametitlebelowskip=.0cm
    ,topline=true
    ,bottomline=true
    ,rightline=true
    ,leftline=true
    ,leftmargin=0.01cm
    ,linewidth=0.02cm
    ,skipabove=0.01cm
    ,innerbottommargin=0.1cm
    ,skipbelow=0.1cm
}

\mdfsetup{%
    middlelinecolor=black,
    middlelinewidth=1pt,
roundcorner=4pt}

\setlength{\parindent}{0pt}

\mdtheorem[style=enclosed]{theorem}{Theorem}
\mdtheorem[style=enclosed]{lemma}{Lemma}[theorem]
\mdtheorem[style=enclosed]{claim}{Claim}[theorem]
\mdtheorem[style=enclosed]{ques}{Question}
\mdtheorem[style=enclosed]{defn}{Definition}
\mdtheorem[style=enclosed]{notn}{Notation}
\mdtheorem[style=enclosed]{obs}{Observation}
\mdtheorem[style=enclosed]{eg}{Example}
\mdtheorem[style=enclosed]{cor}{Corollary}
\mdtheorem[style=enclosed]{note}{Note}

% \let\thetheorem=\relax
% \let\thelemma=\relax
% \let\theclaim=\relax
% \let\theques=\relax
% \let\thedefn=\relax
% \let\thenotn=\relax
% \let\theobs=\relax
% \let\thecor=\relax
% \let\thenote=\relax

% \renewcommand\qedsymbol{$\blacksquare$}
\newcommand{\nl}{\vspace{0.2cm}\\}
\newcommand{\mc}{\mathcal}
\newcommand{\mi}{\mathit}
\newcommand{\mf}{\mathbf}
\newcommand{\mb}{\mathbb}
\renewcommand{\L}{\mc{L}}
\newcommand{\hd}{\hat{\delta}}

\newcommand{\incfig}[1]{%
    \def\svgwidth{\columnwidth}
    \import{./figures/}{#1.pdf_tex}
}
\pdfsuppresswarningpagegroup=1

\title{\textbf{COL352 Lecture 10}}
\date{}

\begin{document}
\maketitle
\tableofcontents

\section{Recap}

Discussion in previous class about Myhill-Nerode theorem.

\section{Definitions}

\section{Content}

Given DFA $D = (Q, \Sigma, \delta, q_0, A)$ without unreachable states.

We want DFA $D^* = (Q^*, \Sigma, \delta^*, q_0^*, A^*)$ with min number of states that recognizes $\L(D)$.

We know $\sim_D$ refines $=_{\L(D)}$, and $\sim_{D^*}$ is identical to $=_{\L(D)}$.

\begin{defn}
$C_q = \{x \in \Sigma^* \mid \hd(q_0, x) = q\}$.
\end{defn}

Note that none of the $C_q$'s are empty since $D$ has no unreachable states.

We then know that these are equivalence classes of $\sim_D$.

Let $x_q$ be an arbitrary string in $C_q$, i.e., $\hd(q_0, x_q) = q$.

\begin{defn}
    $\equiv$ is an equivalence relation on $Q$ defined as $q \equiv q'$ if $C_q, C_{q'}$ are in the sam eequivalence class of $=_{\L(D)}$.
\end{defn}

\begin{claim}
    $q \equiv q' \iff x_q =_{\L(D)} x_q'$
\end{claim}

\begin{proof}
    Forward direction: Obvious by the definition of $\equiv$ and $x_q \in C_q, x_{q'} \in C_{q'}$\nl
    Backward direction: Follows from the fact that $\sim_D$ refines $=_{\L(D)}$ and the equivalence class of $x_q$ wrt $\sim_D$ is $C_q$, and similarly for $C_{q'}$.
\end{proof}

\begin{claim}
    If $q \equiv q'$, then $\forall a \in \Sigma, \delta(q, a) \equiv \delta(q', a)$.
\end{claim}

\begin{proof}
    $q \equiv q' \iff x_q =_L x_{q'}$ from the previous claim.\nl
    $x_q =_L x_{q'} \implies x_q a =_L x_{q'} a$ (as done in last class) $\implies \hd(q_0, x_qa) \equiv \hd(q_0, x_{q'}a) \implies \delta(\hd(q_0, x_q), a) \equiv \delta(\hd(q_0, x_{q'}), a)
    \implies \delta(q, a) \equiv \delta(q', a)$.
\end{proof}

\begin{claim}
    $q \equiv q'$ iff $\forall z \in \Sigma^*, \hd(q, z), \hd(q', z)$ are both in $A$ or both not in $A$.
\end{claim}

\begin{proof}
    $q \equiv q' \equiv x_q =_L x_{q'} \iff \forall z$ either both $x_q z, x_{q'}z$ are in $L$ or both not in $L$ $\iff \forall z, \hd(q_0, x_qz), \hd(q_0, x_{q'}z)$ both in $A$ or both not in $A$
        iff ... iff $\hd(q, z), \hd(q', z)$ both in $A$ or both not in $A$.
\end{proof}

Having determined $\equiv$, $D^*$ can be constructed as:

\begin{enumerate}
    \item $Q^*:$ one state from each equivalence class of $\equiv$.
    \item rep $: Q \to Q'$ rep$(q)$ = the representative fo equivalence class of $q$ under $\equiv$.
    \item $q_0^* = $ rep$(q_0)$.
    \item $\delta^*(r, a) = $ rep$(\delta(r, a))$.
    \item $A^* = \{\mathrm{rep}(q) \mid q \in A\}$.
\end{enumerate}

The invariant is $\hd^*(q_0^*, x) = \mathrm{rep}(\hd(q, x))$.

\begin{ques}
    How do we construct this equivalence relation $\equiv$?
\end{ques}

\begin{ans}
    Recall that $q \equiv q'$ iff $\forall z \in \Sigma^*, \hd(q, z), \hd(q', z)$ are both in $A$ or both not in $A$.
\end{ans}

\begin{defn}
    $q \equiv_n q'$ if for all $z \in \Sigma^*$ of length $\le n$, $\forall z \in \Sigma^*, \hd(q, z), \hd(q', z)$ are both in $A$ or both not in $A$.
\end{defn}

\begin{note}
    Note that each $\equiv_n$ refines $\equiv_{n-1}$, and $\equiv$ refines $\equiv_n$. Also, $q \equiv q \iff \forall n, q \equiv_n q'$.\nl
    Also note that $\equiv_0$ is easy to find: $q \equiv_0 q'$ iff either both $q, q'$ are in $A$ or both are not in $A$, so we can use $\equiv_0$ as an equivalent statement for either both $q, q'$
    being in $A$ or not being in $A$.
\end{note}

\begin{claim}
    $q \equiv_n q'$ iff $q \equiv_{n-1} q' \land \forall a \in \Sigma, \delta(q, a) \equiv_{n-1} \delta(q', a)$.
\end{claim}

\begin{proof}
    Forward direction:\nl
    \begin{align*}    
        q \equiv_n q' &\implies (\forall z, |z| \le n \implies \hd(q, z) \equiv_0 \hd(q', z))\\
    \end{align*}
    Specialising to $|z| \le n - 1$, we have $q \equiv_{n-1} q'$.\nl
    Now continuing, we have for $z'$ of length $\le n - 1$, $\hd(q, az') = \hd(q', az') \iff \hd(\delta(q, a), z') \equiv_0 \hd(\delta(q', a), z')$, so $\delta(q, a) \equiv_{n-1} \delta(q',
        a)$.
    Backward direction:\nl
    If $|z| \le n - 1$, use $q \equiv_{n-1} q'$ to get $\hd(q, z) \equiv_0 \hd(q', z)$.
    If $|z| = n$, then $z = az'$ for some $a \in \Sigma$, $z' \in \Sigma^*$, and $|z'| \le n - 1$. So we have $\hd(q, z) = \hd(q, az') = \hd(\delta(q, a), z') \equiv_0 \hd(\delta(q', a), z')
\text{(either both are or are not in $A$)} = \hd(q', z)$, so $q \equiv_n q'$.
\end{proof}

\begin{ques}
    Can we make an algorithm to find the equivalence classes?
\end{ques}

\begin{ans}
\begin{algorithmic}
    \Function{Partition}{$D = (Q, \Sigma, \delta, q_0, A)$}
        \State $n \gets 0$
        \State $\equiv_0 = \{(q, q') \in Q \times Q \mid \text{ both } q, q' \in A \text{ or both }q, q' \not\in A\}$
        \While{True}
            \State $n \gets n + 1$
            \State $\equiv_n \gets \{(q, q') \in \equiv_{n-1} \land \forall a \in \Sigma, (\delta(q, a), \delta(q', a)) \in \equiv_{n-1}\}$.
            \If{$\equiv_n = \equiv_{n-1}$}
                \State break
            \EndIf
        \EndWhile
        \State \Return $\equiv_n$
    \EndFunction
\end{algorithmic}
\end{ans}

\begin{claim}
    If $\equiv_n$ is identical to $\equiv_{n-1}$, then they are identical to $\equiv$.
\end{claim}

\begin{proof}
    Next class.
\end{proof}

\end{document}
