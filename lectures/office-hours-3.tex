\documentclass[a4paper]{article}
\usepackage[english]{babel}
\usepackage[a4paper,top=2cm,bottom=2cm,left=2cm,right=2cm,marginparwidth=1.75cm]{geometry}
\usepackage{amsmath}
\usepackage{amsfonts}
% \usepackage{amsthm}
\usepackage{amssymb}
\usepackage{graphicx}
\usepackage[colorinlistoftodos]{todonotes}
\usepackage[colorlinks=true, allcolors=blue]{hyperref}
\usepackage{import}
\usepackage{pdfpages}
\usepackage{transparent}
\usepackage{xcolor}
\usepackage{algorithmicx}
\usepackage{algpseudocode}

\usepackage{thmtools}
\usepackage{enumitem}
\usepackage[framemethod=TikZ]{mdframed}

\usepackage{xpatch}

\usepackage{boites}
\makeatletter
\xpatchcmd{\endmdframed}
{\aftergroup\endmdf@trivlist\color@endgroup}
{\endmdf@trivlist\color@endgroup\@doendpe}
{}{}
\makeatother

%\usepackage[poster]{tcolorbox}
%\allowdisplaybreaks
%\sloppy

\usepackage[many]{tcolorbox}

\xpatchcmd{\proof}{\itshape}{\bfseries\itshape}{}{}

% to set box separation
\setlength{\fboxsep}{0.8em}
\def\breakboxskip{7pt}
\def\breakboxparindent{0em}

\newenvironment{proof}{\begin{breakbox}\textit{Proof.}}{\hfill$\square$\end{breakbox}}
\newenvironment{ans}{\begin{breakbox}\textit{Answer.}}{\end{breakbox}}
\newenvironment{soln}{\begin{breakbox}\textit{Solution.}}{\end{breakbox}}

% \tcolorboxenvironment{proof}{
%     blanker,
%     before skip=\topsep,
%     after skip=\topsep,
%     borderline={0.4pt}{0.4pt}{black},
%     breakable,
%     left=12pt,
%     right=12pt,
%     top=12pt,
%     bottom=12pt,
% }
% 
% \tcolorboxenvironment{ans}{
%     blanker,
%     before skip=\topsep,
%     after skip=\topsep,
%     borderline={0.4pt}{0.4pt}{black},
%     breakable,
%     left=12pt,
%     right=12pt,
% }

\mdfdefinestyle{enclosed}{
    linecolor=black
    ,backgroundcolor=none
    ,apptotikzsetting={\tikzset{mdfbackground/.append style={fill=gray!100,fill opacity=.3}}}
    ,frametitlefont=\sffamily\bfseries\color{black}
    ,splittopskip=.5cm
    ,frametitlebelowskip=.0cm
    ,topline=true
    ,bottomline=true
    ,rightline=true
    ,leftline=true
    ,leftmargin=0.01cm
    ,linewidth=0.02cm
    ,skipabove=0.01cm
    ,innerbottommargin=0.1cm
    ,skipbelow=0.1cm
}

\mdfsetup{%
    middlelinecolor=black,
    middlelinewidth=1pt,
roundcorner=4pt}

\setlength{\parindent}{0pt}

\mdtheorem[style=enclosed]{theorem}{Theorem}
\mdtheorem[style=enclosed]{lemma}{Lemma}[theorem]
\mdtheorem[style=enclosed]{claim}{Claim}[theorem]
\mdtheorem[style=enclosed]{ques}{Question}
\mdtheorem[style=enclosed]{defn}{Definition}
\mdtheorem[style=enclosed]{notn}{Notation}
\mdtheorem[style=enclosed]{obs}{Observation}
\mdtheorem[style=enclosed]{eg}{Example}
\mdtheorem[style=enclosed]{cor}{Corollary}
\mdtheorem[style=enclosed]{note}{Note}

% \let\thetheorem=\relax
% \let\thelemma=\relax
% \let\theclaim=\relax
% \let\theques=\relax
% \let\thedefn=\relax
% \let\thenotn=\relax
% \let\theobs=\relax
% \let\thecor=\relax
% \let\thenote=\relax

% \renewcommand\qedsymbol{$\blacksquare$}
\newcommand{\nl}{\vspace{0.2cm}\\}
\newcommand{\mc}{\mathcal}
\newcommand{\mi}{\mathit}
\newcommand{\mf}{\mathbf}
\newcommand{\mb}{\mathbb}
\renewcommand{\L}{\mc{L}}
\newcommand{\hd}{\hat{\delta}}

\newcommand{\incfig}[1]{%
    \def\svgwidth{\columnwidth}
    \import{./figures/}{#1.pdf_tex}
}
\pdfsuppresswarningpagegroup=1

\title{\textbf{COL352 Office Hours 2}}
\date{}

\begin{document}
\maketitle

\begin{ques}
    How to solve question 1 of quiz inductively? Using Myhill-Nerode theorem?
\end{ques}

\begin{ans}
    The empty string is trivially in $L$. Consider a string $x \in L$. The same partition works for $x0$, and for $x1$, shifting the partition location by $1$ to the right works. So $x$ is in $L$, and
    we are done.\nl
    Using Myhill-Nerode theorem looks slightly weird here, since we need to show that there is a finite number of equivalence classes. However, it leads to the above proof by the following argument.
    To build the set of all equivalence classes, we can try all strings one by one. In a way, we can traverse the infinite complete trie on $\{0, 1\}$. So let's see what happens when we add in $x0$
    for some $x$ into the equivalence classes, and similarly for $x1$. How is the equivalence class of these related to that of $x$? The path to answering that question itself leads to the fact
    $\{0, 1\}^*$ is in $L$.
\end{ans}

\begin{ques}
    Is it possible to find an encoding of DFAs on a fixed alphabet (or on any alphabet) that forms a regular language? (motivation - relation between encodings, passing DFAs into DFAs). What about an encoding of DFAs, and inputs, such that
    there is a DFA that recognizes $enc(D)\#x$ where $x$ is an input accepted by $D$? 
    \todo[inline]{Can we even do the second part?}
\end{ques}

\begin{ans}
    For simplicity consider $\Sigma = \{0, 1\}$. For the second question, need to extend $\Sigma$ a bit but we'll keep that for later on.\nl
    The answer to the fixed alphabet case is the same as that of the any alphabet case -- just prepend the number of 1s equal to the size of the alphabets in the usual encoding of DFAs.\nl
    Now using the natural order of such DFA representations, we can find a bijection between this and $\mb{N}$, and similarly we can find one between $\mb{N}$ and $\{0, 1\}^*$.\nl
    Hence there is a bijection between the set of all DFAs (in this encoding) and $\{0, 1\}^*$.\nl
    On a later search on Google, this was actually a question
    \href{https://cs.stackexchange.com/questions/2682/is-the-set-of-codes-of-deterministic-finite-state-automata-a-regular-language}{here}.
\end{ans}

\begin{ques}
    Consider a string $x$ which has an equal number of 0s and 1s, and starting and ending with the same bit. Show that there exist non-empty strings $x_1, x_2$ such that $x = x_1x_2$, and number of 0s
    and 1s in $x_1$ are the same, and the same holds for $x_2$.
\end{ques}

\begin{ans}
    Use the same argument as the quiz - in the first position, the difference is not 0, and just before the last position, the difference is not 0 either, so there exists a place somewhere in
    between, so done.
\end{ans}

\end{document}
