\documentclass[a4paper]{article}
\usepackage[english]{babel}
\usepackage[a4paper,top=2cm,bottom=2cm,left=2cm,right=2cm,marginparwidth=1.75cm]{geometry}
\usepackage{amsmath}
\usepackage{amsfonts}
% \usepackage{amsthm}
\usepackage{amssymb}
\usepackage{graphicx}
\usepackage[colorinlistoftodos]{todonotes}
\usepackage[colorlinks=true, allcolors=blue]{hyperref}
\usepackage{import}
\usepackage{pdfpages}
\usepackage{transparent}
\usepackage{xcolor}
\usepackage{algorithmicx}
\usepackage{algpseudocode}

\usepackage{thmtools}
\usepackage{enumitem}
\usepackage[framemethod=TikZ]{mdframed}

\usepackage{xpatch}

\usepackage{boites}
\makeatletter
\xpatchcmd{\endmdframed}
{\aftergroup\endmdf@trivlist\color@endgroup}
{\endmdf@trivlist\color@endgroup\@doendpe}
{}{}
\makeatother

%\usepackage[poster]{tcolorbox}
%\allowdisplaybreaks
%\sloppy

\usepackage[many]{tcolorbox}

\xpatchcmd{\proof}{\itshape}{\bfseries\itshape}{}{}

% to set box separation
\setlength{\fboxsep}{0.8em}
\def\breakboxskip{7pt}
\def\breakboxparindent{0em}

\newenvironment{proof}{\begin{breakbox}\textit{Proof.}}{\hfill$\square$\end{breakbox}}
\newenvironment{ans}{\begin{breakbox}\textit{Answer.}}{\end{breakbox}}
\newenvironment{soln}{\begin{breakbox}\textit{Solution.}}{\end{breakbox}}

% \tcolorboxenvironment{proof}{
%     blanker,
%     before skip=\topsep,
%     after skip=\topsep,
%     borderline={0.4pt}{0.4pt}{black},
%     breakable,
%     left=12pt,
%     right=12pt,
%     top=12pt,
%     bottom=12pt,
% }
% 
% \tcolorboxenvironment{ans}{
%     blanker,
%     before skip=\topsep,
%     after skip=\topsep,
%     borderline={0.4pt}{0.4pt}{black},
%     breakable,
%     left=12pt,
%     right=12pt,
% }

\mdfdefinestyle{enclosed}{
    linecolor=black
    ,backgroundcolor=none
    ,apptotikzsetting={\tikzset{mdfbackground/.append style={fill=gray!100,fill opacity=.3}}}
    ,frametitlefont=\sffamily\bfseries\color{black}
    ,splittopskip=.5cm
    ,frametitlebelowskip=.0cm
    ,topline=true
    ,bottomline=true
    ,rightline=true
    ,leftline=true
    ,leftmargin=0.01cm
    ,linewidth=0.02cm
    ,skipabove=0.01cm
    ,innerbottommargin=0.1cm
    ,skipbelow=0.1cm
}

\mdfsetup{%
    middlelinecolor=black,
    middlelinewidth=1pt,
roundcorner=4pt}

\setlength{\parindent}{0pt}

\mdtheorem[style=enclosed]{theorem}{Theorem}
\mdtheorem[style=enclosed]{lemma}{Lemma}[theorem]
\mdtheorem[style=enclosed]{claim}{Claim}[theorem]
\mdtheorem[style=enclosed]{ques}{Question}
\mdtheorem[style=enclosed]{defn}{Definition}
\mdtheorem[style=enclosed]{notn}{Notation}
\mdtheorem[style=enclosed]{obs}{Observation}
\mdtheorem[style=enclosed]{eg}{Example}
\mdtheorem[style=enclosed]{cor}{Corollary}
\mdtheorem[style=enclosed]{note}{Note}

% \let\thetheorem=\relax
% \let\thelemma=\relax
% \let\theclaim=\relax
% \let\theques=\relax
% \let\thedefn=\relax
% \let\thenotn=\relax
% \let\theobs=\relax
% \let\thecor=\relax
% \let\thenote=\relax

% \renewcommand\qedsymbol{$\blacksquare$}
\newcommand{\nl}{\vspace{0.2cm}\\}
\newcommand{\mc}{\mathcal}
\newcommand{\mi}{\mathit}
\newcommand{\mf}{\mathbf}
\newcommand{\mb}{\mathbb}
\renewcommand{\L}{\mc{L}}
\newcommand{\hd}{\hat{\delta}}

\newcommand{\incfig}[1]{%
    \def\svgwidth{\columnwidth}
    \import{./figures/}{#1.pdf_tex}
}
\pdfsuppresswarningpagegroup=1

\title{\textbf{COL352 Lecture 10}}
\date{}

\begin{document}
\maketitle
\tableofcontents

\section{Recap}
Proof of the fact that $\{0^n 1^n \mid n \in \mathbb{N}\}$ is not regular.

We constructed a long enough string depending on $|Q|$, and used pigeonhole principle to claim some state was visited twice in the first $n$ transitions. Then we pumped using the loop, and found a
$k$ such that pumping $k$ times results in a string not in $L$, which gives a contradiction by the definition of the DFA $D$.

\section{Definitions}


\section{Content}

\begin{note}
    Plan: Abstract out ideas from the last proof to get a necessary condition for $L$ to be regular, or equivalently, a sufficient condition for $L$ to be not regular.
\end{note}

Suppose $L \subseteq \Sigma^*$ is a regular language, and $D = (Q, \Sigma, \delta, q_0, A)$ be a DFA recognizing $L$.\nl
Let $p = |Q|$. Consider $s \in L$ with $|s| = n \ge p$. Let $q_i = \hd(q_0, s[1]s[2] \ldots s[i])$.\nl
By the pigeonhole principle, $\exists i, j$ such that $i < j \le p$ and $q_i = q_j$.\nl
$\hd(q_i, s[i+1]\ldots s[j]) = q_j = q_i$.\nl
Let $x = s[1] \ldots s[i], y = s[i+1] \ldots s[j], x = s[j+1] \ldots s[n]$. So we have $s = xyz$, $|y| > 0, |xy| \le p$.\nl
$\hd(q_i, y) = q_i, \hd(q_0, x) = q_i, \hd(q_i, z) = q_n \in A$.\nl
$\hd(q_0, xz) = \hd(\hd(q_0, x), z) = \hd(q_i, z) = q_n \in A$. So $xz \in L$.\nl
$\hd(q_0, xy^kz) = q_n \in A$.\nl
Thus $\forall k \in \mb{N} \cup \{0\}$, we have $xy^kz \in L$.\nl

\begin{lemma}
    \textbf{Pumping Lemma for Regular Languages}\nl
    Let $L \in \Sigma^*$ be a regular language. Then $\exists p \in \mb{N}$ such that $\forall s \in L$ with $|s| \ge p$, $\exists x, y, z \in \Sigma^*$ such that
    \begin{enumerate}
        \item $s = xyz$
        \item $|y| > 0$
        \item $|xy| < p$
        \item $\forall i \in \mb{N} \cup \{0\}$, $xy^iz \in L$
    \end{enumerate}
\end{lemma}
\begin{proof}
    Let $D$ be a DFA for $L$. Set $p = |Q|$ and proceed as in the previous paragraph.
\end{proof}
\begin{note}
    This holds even when $|L|$ is finite, just take $p > $ the length of the longest string in $L$. Then this statement is vacuously true.
\end{note}
\begin{lemma}
    \textbf{Contrapositive of the Pumping Lemma for Regular languages}\nl
    Let $L \in \Sigma^*$ be any regular language. If $\forall p \in \mb{N}, \exists s \in L$ with $|s| \ge p$, such that $\forall x, y, z \in \Sigma^*$,
    $$
    (s = xyz) \land (|y| > 0) \land (|xy| \le p) \implies \exists i \in \mb{N} \cup \{0\} \text{ such that } xy^iz \not\in L
    $$
    Then $L$ is not regular.
\end{lemma}
\begin{note}
    Guideline: Look at it like this sort of a game.\nl
    $\forall$ corresponds to moves by opponent. $\exists$ corresponds to moves by you.\nl
    \begin{center}
    \begin{tabular}{|c|c|}
        \hline
        Opponent & You  \\
        \hline
        Gives $p \in \mb{N}$ &  \\
        & $s : s \in L, |s| \ge p$ \\
        $x, y, z$ such that $s = xyz, y \ne \epsilon, |xy| \le p$ & \\
        & $i \in \mb{N} \cup \{0\}$\\
        \hline
    \end{tabular}
    \end{center}

    You win if $xy^i \not \in L$, opponent otherwise.\nl

    The pumping lemma says that if $L$ is regular, then the opponent has a winning strategy, i.e., if you have a winning strategy, then $L$ is not regular.
\end{note}

\begin{eg}
    Consider $\Sigma = \{0, 1\}, L = \{s \in \Sigma^* \mid \# 0 = \# 1 \text{ in }s\}$.
\end{eg}

\begin{ans}
    \begin{center}
        \begin{tabular}{|c|c|}
            \hline
            Opponent  & You \\
            \hline
            $p$  &  \\
                 & $0^p1^p$ \\
            $0^p 1^p = xyz, y \ne \epsilon, |xy| \le p$ &  \\
                                                        & $i \ne 1$\\
                                                        \hline
        \end{tabular}
    \end{center}
    $|xy| = q \le p$, so $xy = 0^q$. Then done.
\end{ans}

\begin{eg}
    Consider $\Sigma = \{0, 1\}, L = \{ww \in \Sigma^* \mid w \in \Sigma^*\}$.
\end{eg}

\begin{ans}
    \begin{center}
        \begin{tabular}{|c|c|}
            \hline
            Opponent & You \\
            \hline
            $p$  &  \\
                 & $0^p1^p0^p1^p$ \\
            $x, y, z$...  &  \\
                          & any $i \ne 1$ \\
                          \hline
        \end{tabular}

    \end{center}
\end{ans}

\begin{eg}
    Consider $\Sigma = \{0, 1\}, L = \{0^i1^j \in \Sigma^* \mid i > j\}$.
\end{eg}
\begin{ans}
    \begin{center}
        \begin{tabular}{|c|c|}
            \hline
            Opponent & You \\
            \hline
            $p$  &  \\
                 & $0^{p+1}1^p$ \\
            $x, y, z$...  &  \\
                          & $i = 0$ \\
                          \hline
        \end{tabular}
    \end{center}
    Note that when we play this, by a similar argument as above, we have $|y| > 0$, so the length of first prefix of $0$ is $\le p$, which gives a contradiction.
\end{ans}

\begin{eg}
    Consider $\Sigma = \{a, b, c\}, L = \{a^{n_1}b^{n_2}c^{n_3} \in \Sigma^* \mid n_1 = 1 \implies n_2 = n_3\}$.
\end{eg}
\begin{ans}
    The opponent always has a winning strategy. If we give $ab^nc^n$, then the opponent gives us $x = \epsilon, y = a, z = b^nc^n$.
    If we give $a^m b^n c^p$, the opponent gives us $x = \epsilon, y = a^m, z = b^n c^p$. However this language is not regular (apply pumping lemma on the reverse of this language, or show that the
    language of strings $s$ such that  $as$ is in $L$ is regular if $L$ is regular, and use the pumping lemma on that).
\end{ans}

\end{document}
