\documentclass[a4paper]{article}
\usepackage[english]{babel}
\usepackage[a4paper,top=2cm,bottom=2cm,left=2cm,right=2cm,marginparwidth=1.75cm]{geometry}
\usepackage{amsmath}
\usepackage{amsfonts}
\usepackage{amsthm}
\usepackage{amssymb}
\usepackage{graphicx}
\usepackage[colorinlistoftodos]{todonotes}
\usepackage[colorlinks=true, allcolors=blue]{hyperref}
\usepackage{import}
\usepackage{pdfpages}
\usepackage{transparent}
\usepackage{xcolor}
\usepackage{algorithmicx}
\usepackage{algpseudocode}

\setlength{\parindent}{0pt}

\newtheorem{theorem}{Theorem}[section]
\newtheorem{lemma}{Lemma}
\newtheorem{claim}{Claim}
\newtheorem{defn}{Definition}
\newtheorem{notn}{Notation}
\newtheorem{obs}{Observation}
\newtheorem{eg}{Example}
\newtheorem{cor}{Corollary}
\newtheorem{note}{Note}

\renewcommand\qedsymbol{$\blacksquare$}
\newcommand{\nl}{\vspace{0.2cm}\\}
\newcommand{\mc}{\mathcal}
\newcommand{\mi}{\mathit}
\newcommand{\mf}{\mathbf}
\newcommand{\mb}{\mathbb}

\newcommand{\incfig}[1]{%
    \def\svgwidth{\columnwidth}
    \import{./figures/}{#1.pdf_tex}
}
\pdfsuppresswarningpagegroup=1

\title{\textbf{COL352 Lecture 4}}
\date{}

\begin{document}
\maketitle
\tableofcontents

\iffalse

\section{Logistics}
\subsection{Lectures}
\begin{enumerate}
    \item Timings - Monday 11AM, Wednesday 11AM, Thursday 12PM
    \item Mode - synchronous, MS Teams
\end{enumerate}
\section{Grading}
\begin{center}
\begin{tabular}{|c|c|}
    \hline
    Class participation     &  5\% \\
    \hline
    Quizzes (best 3 of 4-5) & 30\% \\
    \hline
    HW (best 25 out of 30+) & 25\% \\
    \hline
    Major exam	            & 40\% \\
    \hline
\end{tabular}
\end{center}
\section{The Halting Problem}
\subsection{Statement}
Does there exist a program $H$ that always halts, and given inputs a program $P$ and an input $i$, determine whether $P$ terminates/halts on $i$?
\subsection{Answer}
No, there does not exist such a program.
\subsection{Proof}
Suppose there does exist such a program $H$.\nl
Consider the following program $C$:
\begin{algorithmic}[1]
    \Function{$C$}{$P$}
        \If{\Call{$H$}{$P, P$}}
            \State run an infinite loop
        \Else
            \State \Return
        \EndIf
    \EndFunction
\end{algorithmic}
Now consider what happens when we call $C$ on the input $C$.\nl
Suppose $C(C)$ doesn't halt. Then this means that $H(C, C)$ is false (by the definition of $H$).
Hence, by line 2, $C(C)$ must run forever, and this gives a contradiction.\nl
Hence $C(C)$ must halt. But in that case, we have $H(C, C)$ to be true, and by line $2$, we have a contradiction yet again.\nl
Hence our assumption that such a program $H$ exists is false, and we have proved the claim. $\blacksquare$.

\fi

% start here


\section{Recap}

Recall that regular languages are closed under complementation, $\cap$ and $\cup$.

We also defined $L_1 L_2 = \{x_1 x_2 \mid x_1 \in L_1, x_2 \in L_2\}$.

\begin{claim}
    The class of regular languages is closed under concatenation.
\end{claim}

\begin{notn}
    Let $rev(x)$ denote the reverse of a string $x \in \Sigma^*$.
\end{notn}

\begin{notn}
    Let $rev(L)$ denote $\{rev(x) \mid x \i L\}$.
\end{notn}

\begin{claim}
    The class of regular languages is closed under reversal.
\end{claim}

\begin{figure}[ht]
    \centering
    \incfig{example}
    \caption{An example of thing that doesn't work}
    \label{fig:example}
\end{figure}

Consideration: $\epsilon$ transition? Gives NFA, not DFA.\\

So let's consider the following language:\\

$\Sigma = \{0, 1\}$, $L_1 = \{x \in \Sigma^* \mid x \text{ begins with } 01\}$.\\

$rev(\Sigma) = \{x \in \Sigma^* | x \text{ ends with } 10\}$.\\

Strategy: reverse all transitions of the DFA, swap accepting and starting states.
Easy to make DFA for both, but using the reversal strategy, $\delta$ is not a function, and hence not a DFA.\\

Consider \\

$\Sigma = \{0, 1\}$, $L_1 = \Sigma^*$, $L_2 = \{1w \mid |w| = 3\}$.
Easy to make DFA of both.\\

Using $\epsilon$ transitions, we can create an NFA for the concatenation of these languages.\\
\begin{figure}[ht]
    \centering
    \incfig{lec4-second-example}
    \caption{NFA example for $L_1 L_2$}
    \label{fig:lec4-second-example}
\end{figure}


\begin{defn}
    A non-deterministic finite automaton (NFA) is a 5-tuple $(Q, \Sigma, \Delta, Q_0, A)$, where
    \begin{enumerate}
        \item $Q$ is a \underline{finite non-empty set of states}.
        \item $\Sigma$ is the input alphabet (finite set).
        \item $\Delta \subseteq Q \times \left(\Sigma \cup \{\epsilon\}\right) \times Q$ is the \underline{set of transitions}. 
        \item $Q_0 \subseteq Q$ (possibly empty).
        \item $A \subseteq Q$ is a set of accepting states.
    \end{enumerate}
\end{defn}

Use the first reversal example here.

\begin{note}
    NFA allows you to take/not take a transition.
\end{note}

\begin{note}
    Note that point 3 is equivalent to $\Delta : Q \times \left(\Sigma \cup \{\epsilon\}\right) \to 2^Q$; this equivalence arises by considering the set of $z$ such that $(x, y, z) \in
    \Delta$, and mapping $(x, y)$ to this set, which is a subset of $Q$.
\end{note}

\begin{note}
    There might be (possibly 0) many runs of an NFA on an input.
\end{note}

\begin{note}
    It might be possible that one run ends in an accepting state and another ends in a rejecting state.
\end{note}

The corresponding NFA for the reversal problem is as follows:

\begin{figure}[ht]
    \centering
    \incfig{example-beginning-with-01}
    \caption{example beginning with 01}
    \label{fig:example-beginning-with-01}
\end{figure}

\begin{figure}[ht]
    \centering
    \incfig{example-ending-with-10}
    \caption{example ending with 10}
    \label{fig:example-ending-with-10}
\end{figure}

$Q = \{q_0, q_1, q_2, q_e\}$.

$\Sigma = \{0, 1\}$.

$Q_0 = \{q_2\}, A = \{q_0\}$.

$\Delta = \{(q_2, 0, q_2), (q_2, 1, q_2), (q_2, 1, q_1), (q_1, 0, q_0), (q_e, 0, q_e), (q_e, 1, q_e), (q_e, 0, q_1), (q_e, 1, q_e)\}$.\\

Now let's see what happens on a run of this NFA with the input $1010$.\\

Current state $(q_2, 1010)$.\\

Can go to $(q_1, 010)$ or $(q_2, 010)$.\\

From $(q_1, 010)$, forced to go to $(q_0, 10)$, and from $q_2, 010$, forced to stay in $(q_2, 10)$.\\

The first run dies off, since no other possibilities.\\

From $(q_2, 10)$ we can go to $(q_1, 0)$ or $(q_2, 0)$, and from the first, we get to $(q_0, \epsilon)$, and from the second, we get to $(q_2, \epsilon)$.\\

Hence there are 3 possible runs, first rejecting, second accepting, and third rejecting.\\

For the second example, which has an $\epsilon$ transition, let's see how it works as well.\\

Consider the input $x = 00100$.\\

First step\\
$(A, 00100) \to (A, 0100), (B, 00100)$.\\

Second step\\
$(B, 00100) \to (error, 0100) \to \cdots$ (stays forever)\\
$(A, 0100) \to (A, 100), (B, 0100)$\\

Third step\\
$(A, 100) \to (B, 100) \to (C, 00) \to (D, 0) \to (E, \epsilon)$\\
$(A, 100) \to (A, 00) \to (B, 00) \to (error, 0) \to (error, \epsilon)$\\
$(A, 100) \to (A, 00) \to (A, 0)$\\

Fourth step\\
$(A, 0) \to (A, \epsilon) \to (B, \epsilon)$\\
$(A, 0) \to (B, 0) \to (error, \epsilon)$\\

So in this case, all runs of the NFA reject $x$.

We say that the NFA accepts a string $x$, if there is at least one accepting run of the NFA, and it rejects $x$ otherwise.

\begin{notn}
    Informally, we say NFA $N$ accepts string $x$ is $\exists$ an accepting run of $N$ on $x$; $N$ rejects $x$ if all runs of $N$ on $x$ reject.
\end{notn}

\end{document}
