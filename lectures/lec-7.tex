\documentclass[a4paper]{article}
\usepackage[english]{babel}
\usepackage[a4paper,top=2cm,bottom=2cm,left=2cm,right=2cm,marginparwidth=1.75cm]{geometry}
\usepackage{amsmath}
\usepackage{amsfonts}
\usepackage{amsthm}
\usepackage{amssymb}
\usepackage{graphicx}
\usepackage[colorinlistoftodos]{todonotes}
\usepackage[colorlinks=true, allcolors=blue]{hyperref}
\usepackage{import}
\usepackage{pdfpages}
\usepackage{transparent}
\usepackage{xcolor}
\usepackage{algorithmicx}
\usepackage{algpseudocode}

\usepackage{thmtools}
\usepackage{enumitem}
\usepackage[framemethod=TikZ]{mdframed}

\usepackage{xpatch}

\makeatletter
\xpatchcmd{\endmdframed}
{\aftergroup\endmdf@trivlist\color@endgroup}
{\endmdf@trivlist\color@endgroup\@doendpe}
{}{}
\makeatother

%\usepackage[poster]{tcolorbox}
%\allowdisplaybreaks
%\sloppy

\usepackage[many]{tcolorbox}

\xpatchcmd{\proof}{\itshape}{\bfseries\itshape}{}{}

\newenvironment{ans} { 
    \vspace{12pt}
    \begin{mdframed}[skipabove=0pt,skipbelow=0pt,innertopmargin=0pt,innerbottommargin=0pt,leftline=false,bottomline=false,topline=false,rightline=false]%
    \noindent \textit{Answer.}  
}{%
    \end{mdframed}
    \vspace{12pt}
}

\tcolorboxenvironment{proof}{
    blanker,
    before skip=\topsep,
    after skip=\topsep,
    borderline={0.4pt}{0.4pt}{black},
    breakable,
    left=12pt,
    right=12pt,
    top=12pt,
    bottom=12pt,
}

\tcolorboxenvironment{ans}{
    blanker,
    before skip=\topsep,
    after skip=\topsep,
    borderline={0.4pt}{0.4pt}{black},
    breakable,
    left=12pt,
    right=12pt,
}

\mdfdefinestyle{enclosed}{
    linecolor=black
    ,backgroundcolor=none
    ,apptotikzsetting={\tikzset{mdfbackground/.append style={fill=gray!100,fill opacity=.3}}}
    ,frametitlefont=\sffamily\bfseries\color{black}
    ,splittopskip=.5cm
    ,frametitlebelowskip=.0cm
    ,topline=true
    ,bottomline=true
    ,rightline=true
    ,leftline=true
    ,leftmargin=0.01cm
    ,linewidth=0.02cm
    ,skipabove=0.01cm
    ,innerbottommargin=0.1cm
    ,skipbelow=0.1cm
}

\mdfsetup{%
    middlelinecolor=black,
    middlelinewidth=1pt,
roundcorner=4pt}

\setlength{\parindent}{0pt}

\mdtheorem[style=enclosed]{theorem}{Theorem}
\mdtheorem[style=enclosed]{lemma}{Lemma}[theorem]
\mdtheorem[style=enclosed]{claim}{Claim}[theorem]
\mdtheorem[style=enclosed]{ques}{Question}
\mdtheorem[style=enclosed]{defn}{Definition}
\mdtheorem[style=enclosed]{notn}{Notation}
\mdtheorem[style=enclosed]{obs}{Observation}
\mdtheorem[style=enclosed]{eg}{Example}
\mdtheorem[style=enclosed]{cor}{Corollary}
\mdtheorem[style=enclosed]{note}{Note}

% \let\thetheorem=\relax
% \let\thelemma=\relax
% \let\theclaim=\relax
% \let\theques=\relax
% \let\thedefn=\relax
% \let\thenotn=\relax
% \let\theobs=\relax
% \let\thecor=\relax
% \let\thenote=\relax

\renewcommand\qedsymbol{$\blacksquare$}
\newcommand{\nl}{\vspace{0.2cm}\\}
\newcommand{\mc}{\mathcal}
\newcommand{\mi}{\mathit}
\newcommand{\mf}{\mathbf}
\newcommand{\mb}{\mathbb}
\renewcommand{\L}{\mc{L}}

\newcommand{\incfig}[1]{%
    \def\svgwidth{\columnwidth}
    \import{./figures/}{#1.pdf_tex}
}
\pdfsuppresswarningpagegroup=1

\title{\textbf{COL352 Lecture 7}}
\date{}

\begin{document}
\maketitle
\tableofcontents

\section{Recap}

Completion of proof of the second part of the proof of the class of regular languages being closed under concatenation.

\section{Definitions}

\begin{defn}
    Let $L \in \Sigma^*$ be any language. We define the language $L^*$ as follows:
    \[
        L^* = L_0 \cup L_1 \cup \cdots = \bigcup_{n=0}^\infty L^n
    \]
    where $L^0 = \{\epsilon\}, L_1 = L, L^n = L\cdot L \cdots L$ where there are $n$ instances of $L$.
\end{defn}

\begin{defn}
    Let $\Sigma$ be a finite alphabet. A \underline{regular expression} over $\Sigma$ is any expression that is in one of the following forms:
    \begin{enumerate}
        \item $\emptyset$.
        \item $\epsilon$.
        \item $a$, where $a \in \Sigma$.
        \item $(R_1 \cup R_2)$ where $R_1, R_2$ are regular expressions over $\Sigma$.
        \item $(R_1R_2)$ or $(R_1 \cdot R_2)$ where $R_1, R_2$ are regular expressions over $\Sigma$.
        \item $(R)^*$ where $R$ is a regular expression over $\Sigma$.
    \end{enumerate}
\end{defn}

\begin{defn}
    The \underline{language of a regular expression} $R$, denoted by $\mc{L}(R)$ is defined as follows:
    \begin{enumerate}
        \item $\mc{L}(\emptyset) = \emptyset$.
        \item $\mc{L}(\epsilon) = \{\epsilon\}$.
        \item $\mc{L}(a) = \{a\}$.
        \item $\mc{L}(R_1 \cup R_2) = \mc{L}(R_1) \cup \mc{L}(R_2)$.
        \item $\mc{L}(R_1R_2) = \mc{L}(R_1) \cdot \L(R_2)$.
        \item $\mc{L}\left((R)^*\right) = \left(\mc{L}(R)\right)^*$
    \end{enumerate}
\end{defn}

\section{Content}

\subsection{Closure under *}

\begin{theorem}
    The class of regular languages is closed under $*$.
\end{theorem}

\begin{proof}
    Let $L$ be regular. Let $D$ be a DFA recognizing $L$. Construct an NFA recognizing $L^*$.\\

    For the DFA $D = (Q, \Sigma, \delta, q_0, A)$, consider the NFA $N = (Q \uplus q^*, \Sigma, \Delta, \{q^*\}, \{q^*\})$.\\

    where $$\Delta = \{(q, a, q') \mid \delta(q, a) = q' \land q, q' \in Q \land a \in A\} \cup \{(q_a, \epsilon, q^*) \mid q_a \in A\} \cup (q^*, \epsilon, q_0)$$

    
    {
        \centering
        \incfig{lec-7-nfs}
    }

    \begin{claim}
        If $N$ accepts $x$, then $x \in L^*$.
    \end{claim}

    \begin{proof}
        Consider an accepting run of $N$ on $x$ Suppose it visits $q^*$ $n + 1$ times. Then we claim $x \in L^n$.
        \begin{proof}
            Induction on $n$.
            $n = 0 \implies$ empty string.

            Now suppose $n > 0$. Consider the second-last time the run visits $q^*$. After that state, it transitions to $q_0$, and that is a run of $D$ on the substring corresponding to whatever
            is left in $x$ after that state, and the part of the run before that is an accepting run of $N$ on the prefix of $x$ that was deleted, and it visits $q^*$ exactly $n$ times. Hence $x = x'x''$
            where $x' \in L^{n-1}$ by the inductive hypothesis and $x'' \in L$ as $D$ accepts $x''$. Hence we have $x \in L^n$, as required.
        \end{proof}
        Hence we are done.
    \end{proof}

    \begin{claim}
        If $x \in L^*$, then $N$ accepts $x$.
    \end{claim}

    \begin{proof}
        $x \in L^* \implies \exists n$ such that $x \in L^n$. Let $n$ be the least integer such that $x \in L^n$. Now we proceed with a proof by induction on $n$.

        Base case: $n = 0$ means empty string, so we are done.

        Now consider $n > 0$. Then $x \in L^n \implies x = x_1 \cdot x_2$ such that $x_1 \in L_{n-1}$ and $x_2 \in L$. By induction hypothesis, $N$ accepts $x_1$. Construct an accepting run
        of $N$ on $x$ as follows:

        [run of $N$ on $x_1$ starting and ending with $q^*$] $\epsilon$ [accepting run of $D$ on $x_2$ starting at $q_0$ and ending at $q \in A$] $\epsilon$ $q^*$.
    \end{proof}

\end{proof}

\begin{note}
    In general, to prove closure of the class of regular languages under some binary operation:
    \begin{enumerate}
        \item Assume $L_1, L_2$ are some arbitrary regular languages. Let $D_1, D_2$ be the DFAs recognizing $L_1, L_2$ respectively.
        \item Construct an \underline{NFA} $N$ recognizing $L_1 \mathrm{op} L_2$.
        \item Argue that $x \in L_1 \mathrm{op} L_2$ iff $N$ accepts $x$.
    \end{enumerate}
\end{note}

\subsection{Regular expressions}

\begin{theorem}
    Suppose $L \in \Sigma^*$ is a language generated by some regular expression over $\Sigma$. Then $L$ is a regular language.
\end{theorem}

\end{document}
