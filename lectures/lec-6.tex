\documentclass[a4paper]{article}
\usepackage[english]{babel}
\usepackage[a4paper,top=2cm,bottom=2cm,left=2cm,right=2cm,marginparwidth=1.75cm]{geometry}
\usepackage{amsmath}
\usepackage{amsfonts}
\usepackage{amsthm}
\usepackage{amssymb}
\usepackage{graphicx}
\usepackage[colorinlistoftodos]{todonotes}
\usepackage[colorlinks=true, allcolors=blue]{hyperref}
\usepackage{import}
\usepackage{pdfpages}
\usepackage{transparent}
\usepackage{xcolor}
\usepackage{algorithmicx}
\usepackage{algpseudocode}

\usepackage{thmtools}
\usepackage{enumitem}
\usepackage[framemethod=TikZ]{mdframed}

\usepackage{xpatch}

\makeatletter
\xpatchcmd{\endmdframed}
{\aftergroup\endmdf@trivlist\color@endgroup}
{\endmdf@trivlist\color@endgroup\@doendpe}
{}{}
\makeatother

%\usepackage[poster]{tcolorbox}
%\allowdisplaybreaks
%\sloppy

\usepackage[many]{tcolorbox}

\xpatchcmd{\proof}{\itshape}{\bfseries\itshape}{}{}

\newenvironment{ans} { 
    \vspace{12pt}
    \begin{mdframed}[skipabove=0pt,skipbelow=0pt,innertopmargin=0pt,innerbottommargin=0pt,leftline=false,bottomline=false,topline=false,rightline=false]%
    \noindent \textit{Answer.}  
}{%
    \end{mdframed}
    \vspace{12pt}
}

\tcolorboxenvironment{proof}{
    blanker,
    before skip=\topsep,
    after skip=\topsep,
    borderline={0.4pt}{0.4pt}{black},
    breakable,
    left=12pt,
    right=12pt,
    top=12pt,
    bottom=12pt,
}

\tcolorboxenvironment{ans}{
    blanker,
    before skip=\topsep,
    after skip=\topsep,
    borderline={0.4pt}{0.4pt}{black},
    breakable,
    left=12pt,
    right=12pt,
}

\mdfdefinestyle{enclosed}{
    linecolor=black
    ,backgroundcolor=none
    ,apptotikzsetting={\tikzset{mdfbackground/.append style={fill=gray!100,fill opacity=.3}}}
    ,frametitlefont=\sffamily\bfseries\color{black}
    ,splittopskip=.5cm
    ,frametitlebelowskip=.0cm
    ,topline=true
    ,bottomline=true
    ,rightline=true
    ,leftline=true
    ,leftmargin=0.01cm
    ,linewidth=0.02cm
    ,skipabove=0.01cm
    ,innerbottommargin=0.1cm
    ,skipbelow=0.1cm
}

\mdfsetup{%
    middlelinecolor=black,
    middlelinewidth=1pt,
roundcorner=4pt}

\setlength{\parindent}{0pt}

\mdtheorem[style=enclosed]{theorem}{Theorem}
\mdtheorem[style=enclosed]{lemma}{Lemma}[theorem]
\mdtheorem[style=enclosed]{claim}{Claim}[theorem]
\mdtheorem[style=enclosed]{ques}{Question}
\mdtheorem[style=enclosed]{defn}{Definition}
\mdtheorem[style=enclosed]{notn}{Notation}
\mdtheorem[style=enclosed]{obs}{Observation}
\mdtheorem[style=enclosed]{eg}{Example}
\mdtheorem[style=enclosed]{cor}{Corollary}
\mdtheorem[style=enclosed]{note}{Note}

% \let\thetheorem=\relax
% \let\thelemma=\relax
% \let\theclaim=\relax
% \let\theques=\relax
% \let\thedefn=\relax
% \let\thenotn=\relax
% \let\theobs=\relax
% \let\thecor=\relax
% \let\thenote=\relax

\renewcommand\qedsymbol{$\blacksquare$}
\newcommand{\nl}{\vspace{0.2cm}\\}
\newcommand{\mc}{\mathcal}
\newcommand{\mi}{\mathit}
\newcommand{\mf}{\mathbf}
\newcommand{\mb}{\mathbb}

\newcommand{\incfig}[1]{%
    \def\svgwidth{\columnwidth}
    \import{./figures/}{#1.pdf_tex}
}
\pdfsuppresswarningpagegroup=1

\title{\textbf{COL352 Lecture 6}}
\date{}

\begin{document}
\maketitle
\tableofcontents

\section{Recap}

Recall the informal definition of a run on NFA. We formalize this today.

\section{Definitions}

\begin{defn}
    Let $N = (Q, \Sigma, \Delta, Q_0, A)$ be an NFA, and $x \in \Sigma^*$ be a string. A \underline{run} of $N$ on $x$ is a sequence $q_0, x_1, q_1, \ldots, x_m, q_m$ such that
    \begin{enumerate}
        \item Each $q_i \in Q$
        \item Each $x_i \in \Sigma \cup \{\epsilon\}$, such that $x = x_1x_2 \ldots x_m$.
        \item $q_0 \in Q_0$
        \item $\forall i : (q_{i - 1}, x_i, q_i) \in \Delta$
    \end{enumerate}
\end{defn}

\begin{defn}
    $q_0, x_1, \ldots, q_{m - 1}, x_m, q_m$ is an \underline{accepting run} if $q_m \in A$, otherwise it is a rejecting run.
\end{defn}

\begin{defn}
    $N$ accepts $x$ if \underline{there exists} an accepting run of $N$ on $x$.
\end{defn}

\begin{defn}
    $\mc{L}(N) = \{x \in \Sigma^* \mid N \text{ accepts } x\}$
\end{defn}

\section{DFA v/s NFA}

\begin{ques}
    Is it possible to create an NFA for any regular language?
\end{ques}

\begin{ans}
    Yes. Consider the following claim.
\end{ans}

\begin{claim}
    If $L$ is recognized by a DFA, then $L$ is recognized by some NFA.
\end{claim}
\begin{proof}

    Suppose $D = (Q, \Sigma, \delta, q_0, A)$ recognizes the language $L$.

    We construct $N = (Q, \Sigma, \Delta, \{q_0\}, A)$, with $\Delta = \{(q, a, q') \in Q \times \Sigma \times Q \mid q' = \delta(q, a)\}$. Then $N$ also recognizes $D$.
\end{proof}

\begin{ques}
    Is it possible to create a DFA for any language that is accepted by an NFA?
\end{ques}

\begin{ans}
    Yes. Consider the following theorem.
\end{ans}

\begin{theorem}
    Let $N$ be any NFA. There exists a DFA $D$ such that $\mc{L}(N) = \mc{L}(D)$.
\end{theorem}



\begin{proof}
    We shall break this into two steps.
    \begin{theorem}
        Let $N$ be any NFA. There exists an NFA $N'$ without $\epsilon$ transitions such that $\mc{L}(N) = \mc{L}(D)$.
    \end{theorem}


    \textit{Intuition:} 
    $x \in \mc{L}(N)$. Let $q_0 x_1 q_1 x_2 q_2 \ldots x_m q_m$ be an accepting run.\\
    $x = x[1] x[2] \ldots x[n]$, with $x[i] \in \Sigma$.
    $\langle x[i] \rangle$ is a subsequence of $\langle x_i \rangle$.\\
    For any $j \not \in \{i_1, i_2, \ldots, i_n\}$, $x_j = \epsilon$, and $x_{i_k} = x[k]$. (basically compression).
    High-level idea - compress $x_{i_{k - 1} + 1} \ldots x_{i_k}$ into a single step.



    Let $N = (Q, \Sigma, \Delta, Q_0, A)$. Define NFA $N'$ as follows:\\

    $$N' = (Q, \Sigma, \Delta', Q_0, A')$$

    Here, 

    \[
        \Delta' = \{(q, a, q') \in Q \times \Sigma \times Q \mid \exists q'' \in Q \text{ such that } q''  \text{ is reachable from } q \text{ by } \epsilon \text{ transitions and } (q'', a, q') \in \Delta \}
    \]

    (note: this corresponds to the compression in the high-level idea)\\

    $$A' = \{q \in Q \mid \exists q' \in A \text{ such that } q' \text{ is reachable from } q \text{ by } \epsilon \text{ transitions of } \Delta\}$$

    (note: this corresponds to compressing the last part of an accepting run)\\

    \begin{claim}
        $\mc{L}(N) \subseteq \mc{L}(N')$
    \end{claim}


    \begin{proof}
        Suppose $x \in \mc{L}(N)$. Then there exists an accepting run $q_0, x_1, q_1, \ldots, x_m, q_m$ of $N$ on $x$.\\
        Let $i_1 < i_2 < \cdots < i_n$ be all the indices such that $x_i \in \Sigma$ (i.e., if $j \not \in \{i_1, \ldots, i_n\}$, then $x_j = \epsilon$).\\
        Let's construct an accepting run of $N'$ on $x$. Consider the sequence $q_0 x[1] q_{i_1} x[2] \ldots x[n] q_{i_n}$.\\
        Now note that:
        \begin{enumerate}
            \item $(q_{i_{j - 1}}, x[j], q_{i_j}) \in \Delta'$

                This follows directly from the definition of $\Delta'$; indeed, in the original sequence, we have $q_{i_{j-1}} \epsilon q_{i_{j-1}+1} \epsilon \ldots \epsilon q){i_j - 1} x[j]
                q_{i_j}$, so using $q = q_{i_{j-1}}, q'' = q_{i_j - 1}, q' = q_{i_j}$ in the definition, we are done.

            \item $q_{i_n} \in A'$

                Note that all transitions after this are $\epsilon$ transitions, and hence the accepting state $q_m$ is reachable from $q_{i_n}$ using $\epsilon$ transitions, and hence $q_{i_n} \in A'$.

        \end{enumerate}
        Hence we get an accepting run of $N'$ on $x$, whence we are done.
    \end{proof}


    \begin{claim}
        $\mc{L}(N') \subseteq \mc{L}(N)$
    \end{claim}


    \begin{proof}

        Suppose $x \in \mc{L}(N')$. Then there exists an acepting run $q_0, x[1]', q_2, \ldots, x[n']', q_{n'}$ of $N'$ on $x$. Note that since there are no $\epsilon$ transitions, we have $x[i]'
        = x[i]$, and $n = n'$.\\

        Now consider any $(q_{i - 1}, x[i], q_i)$. Then since this is in $D'$, we must have some $q''$ such that $q''$ is reachable from $q_{i - 1}$ by $\epsilon$ transitions and $(q'', x[i], q_i) \in
        D$.\\

        Consider the corresponding path from $q_{i - 1}$ to $q''$, say $q_{i - 1} \epsilon q_{i - 1, 1}' \epsilon \ldots \epsilon q'' = q_{i - 1, l_{i - 1}}'$.\\

        Also note that the last state $q_n$ is an accepting state in $N'$, hence there exists a path starting from $q_n$ to some accepting state $q''$ in $A$ consisting solely of $\epsilon$
        transitions, say $q_n \epsilon q_{n, 1}' \epsilon \ldots \epsilon q_{n, l_n}' = q''$.\\

        Consider the path

        $$q_0 \epsilon q_{0, 1}' \epsilon \ldots \epsilon q_{0, l_0} x[1] q_1 \epsilon q_{1, 1} \epsilon \ldots \epsilon q_{1, l_1} x[2] q_2 \ldots q_n \epsilon q_{n, 1}' \epsilon \ldots \epsilon
        q_{n, l_n}'$$

        Then the last state $q_{n, l_n}'$ is accepting as discussed above, and all transitions are in $\Delta$ (by definition of $\Delta'$), and the string corresponding to this run is
        $\epsilon^{l_0} x[1] \epsilon^{l_1} x[2] \ldots \epsilon^{l_{n-1}} x[n] \epsilon^{l_n} = x$, as needed.\\

        Hence we get an accepting run of $N$ on $x$, whence we are done.

    \end{proof}


    Now these two claims together show that $\mc{L}(N') = \mc{L}(N)$, whence we have completed the proof of theorem 2.\\



    \begin{theorem}
        Let $N'$ be any NFA without $\epsilon$ transitions. There exists a DFA $D$ such that $\mc{L}(N) = \mc{L}(D)$.
    \end{theorem}


    \begin{proof}
        Let $N' = (Q, \Sigma, \Delta, Q_0, A)$.\\

        Construct DFA $D = (2^Q, \Sigma, \delta, Q_0, A)$, where

        \begin{enumerate}
            \item $\delta : 2^Q \times \Sigma \to 2^Q  \text{is defined as}$\\
                $$\delta(R, a) = \{q \in Q \mid \exists r \in R : (r, a, q) \in \Delta\} \quad R \in 2^Q \text{, i.e., }R \subseteq Q, a \in \Sigma$$

            \item $Q_0$ is a subset of $2^Q$, and is hence one state.

            \item $\mc{A} = \{R \in 2^Q \mid R \cap A \ne \emptyset\}$
        \end{enumerate}

        \begin{claim}
            $\mc{L}(N') \subseteq \mc{L}(D)$
        \end{claim}


        \begin{proof}
            Suppose $x \in \mc{L}(N')$. Then there exists an accepting run $e = q_0 x[1] q_1 \ldots q_{n-1} x[n]$ of $N'$ on $x$.\\

            Let $Q_0 x[1] Q_1 \ldots Q_{n-1} x[n] Q_n$ be \underline{the} run of $D$ on $x$.\\

            We show the following inductive claim:
            \begin{claim}
                $q_i \in Q_i \forall i$
            \end{claim}

            \begin{proof}
                Base case: for $i = 0$, this is true since the set of starting states is $Q_0$.\\
                Now suppose $i > 0$, and that the inductive hypothesis holds. So we have $q_{i-1} \in Q_{i-1}$. Since $q_{i-1} \in Q_{i-1}$, and $(q_{i-1}, x[i], q_i) \in \Delta$, we have $q_i \in
                \delta(Q_{i-1}, x[i])$, by the definition of $\delta$. Now since $Q_i = \delta(Q_{i-1}, x[i])$ in the run, our inductive step is complete.
            \end{proof}

            Now using the above claim, we get $q_n \in Q_n$. Since $q_n \in A$ (as $e$ is an accepting run), $Q_n \cap A \ne \emptyset$, whence it follows that $D$ accepts $x$.
        \end{proof}


        \begin{claim}
            $\mc{L}(N') \supseteq \mc{L}(D)$
        \end{claim}



        \begin{proof}
            Suppose $x \in \mc{L}(D)$. Let $Q_0 x[1] \ldots Q_{n-1} x[n] Q_n$ be the run of $D$ on $x$, where $Q_n \cap A \ne \emptyset$ (which follows from the fact that this run is an
            accepting run).
            \begin{claim}
                $\forall i$: $q_i \in Q_i \implies$ there exists a run of $N'$ on $x_1, \ldots, x_i$ which ends in state $q_i$.
            \end{claim}

            \begin{proof}
                Base case: $i = 0$: Empty string's run on $q_0$ is an accepting run.\\
                Now assume $i > 0$. $q_i \in Q_i = \delta(Q_{i-1}, x[i]) \implies \exists q_{i-1} \in Q_{i-1}$ such that $(q_{i-1}, x[i], q_i) \in \Delta$. By the inductibe hypothesis, there
                exists a run $q_0 x[1] q_1 \ldots q_{i-2} x[i-1] q_{i-1}$ of $N'$ on $x[1] \ldots x[i-1]$. Consider the run $q_0 x[1] \ldots q_{i-2} x[i-1] q_{i-1} x[i] q_i$. It is a run of $N$ on $x[1]
                \ldots x[i]$, whence we are done.
            \end{proof}


            Using the assumption that $Q_n \cap A \ne \emptyset$, we have a state $q_n \in Q_n \cap A$, and thus there is a run of $N'$ on $x$ ending in state $q_n$, and this is an accepting run on
            $N'$.
        \end{proof}


        Hence we have shown that theorem 3 is true.
    \end{proof}


    Now that we have shown these two claims, we are done.

\end{proof}

\begin{theorem}
    The class of regular languages is closed under concatenation
\end{theorem}

\begin{proof}
    Let $L_1, L_2$ be regular languages, and let $D_1 = (Q_1, \Sigma, \delta_1, q_1, A_1)$ and $D_2 = (Q_2, \Sigma, \delta_2, q_2, A_2)$ be DFAs recognizing $L_1, L_2$ respectively. Then the
    following NFA $N = (Q, \Sigma, \Delta, Q_0, A)$ recognizes $L_1 \cdot L_2$. (Assume wlog that $Q_1 \cap Q_2 = \emptyset$).

    \begin{enumerate}
        \item $Q = Q_1 \cup Q_2$.
        \item $Q_0 = \{q_1\}$.
        \item $A = A_2$.
        \item $\Delta = \{(r_1, a, r_2) \in Q_1 \times \Sigma \times Q_1 \mid \delta_1(r_1, a) = r_2\} \cup \{(q, \epsilon, q_2) \mid q \in A_1\} \cup \{(r_1, a, r_2) \in Q_2 \times \Sigma
            \times Q_2 \mid \delta_2(r_1, a) = r_2\} = \Delta_1 \cup \Delta_2 \cup \Delta_3$
    \end{enumerate}

    \begin{claim}
        If $x \in L_1 \cdot L_2$ then $N$ accepts $x$.
    \end{claim}


    \begin{proof}
        $\exists k$ such that $x[1]\ldots x[k] \in L_1$, $x[k+1]\ldots x[n]\in L_2$ where $n = |x|$.\\

        Let $r_i = \hat{\delta}_1(q_1, x[1] \ldots x[i]), s_j = \hat{\delta}_2(q_2, x[k + 1] \ldots x[k + j])$.\\

        Look at the run
        $r_0 x[1] r_1 \ldots r_{k-1} x[k] r_k \epsilon s_0 x[k + 1] s_1 \ldots s_{n-k}$.\\

        Note that the first $k$ transitions are in $\Delta_1$, the last $n - k$ transitions are in $\Delta_3$, and the middle transition is in $\Delta_2$, so this is indeed a run.

        Now that $s_{n-k}$ is in $A_2$ since $s_{n-k}$ is the end of an accepting run of $x[k+1]\ldots x[n]$ on $D_2$, so $s_{n-k}$ is an accepting state of $N$ as well.

    \end{proof}


    \begin{claim}
        If $N$ accepts $x$, then $x \in L_1 \cdot L_2$.
    \end{claim}


    \begin{proof}
        Since $N$ accepts $x$, there must be a path from $q_1$ to a state in $A_2$, corresponding to a run in $N$.

        Suppose the run is $a_0 x_1 a_1 \ldots a_{m-1} x_m$, where $x = x_1x_2\ldots x_m$. Consider the largest $i$ such that $a_i \in Q_1$. Clearly, since the last state is in $A_2$ which is
        disjoint with $Q_1$, we can't have $i = m$. Hence by the maximality of $i$, we have $a_{i+1} \in Q_2$.\\

        Suppose there is a $j < i$ such that $a_j \in Q_2$. Consider the largest such $j$. Then we have $a_{j+1} \in Q_1$ since if $j = i - 1$ then we know that $a_i \in Q_1$ and otherwise we can
        invoke the maximality of $j$. This implies there is a transition of the form $(q, a, q')$ where $q \in Q_2$ and $q' \in Q_1$, which is impossible. Hence, for all $j \le i$, we have $a_j
        \in
        Q_1$. By a similar argument, we have for all $j > i$, $a_j \in Q_2$.\\

        Using this fact, we know that $x_{i+1} = \epsilon$, and all the other $x_j$s are in $\Sigma$, by the definition of $\Delta$.\\

        Now consider the strings $S_1 = x_1 x_1 \ldots x_i$ and $S_2 = x_{i+2} \ldots x_m$. Since $x_{i+1} = \epsilon$, we have $x = x_1 x_1 \ldots x_i x_{i+2} \ldots x_m = S_1 \cdot S_2$.\\

        Now we claim that $S_1$ is accepted by $D_1$ and $S_2$ is accepted by $D_2$.

        \begin{enumerate}
            \item $S_1$ is accepted by $D_1$
                \begin{proof}

                    We prove this by induction.

                    \begin{claim}
                        $\hat{\delta}_1(a_0, x_1\ldots x_j) = a_j \quad \forall j \le i$
                    \end{claim}

                    \begin{proof}
                        Base case is when $j = 0$, which is true since the empty string ends up in $a_0$ by the definition of $\hat{\delta}_1$. Now suppose $j > 0$. Then we have, by the
                        inductive hypothesis, that $\hat{\delta}_1(a_0, x_1\ldots x_{j-1}) = a_{j-1}$. Now note that $\hat{\delta}_1(a_0, x_1 \ldots x_j) = \delta(\hat{\delta}(a_0,
                        x_1\ldots x_{j-1}), x_j) = \delta(a_{j-1}, x_j)$.
                        This is clearly $a_j$, since $(a_{j-1}, x_j, a_j) \in \Delta_1$. Hence we have completed the inductive step and are done.
                    \end{proof}

                    Using the result for $j = i$, we can see that $\hat{\delta}_1(a_0, S_1) = a_i$. Now note that since $(a_i, x_{i+1}, a_{i+1}) \in \Delta$, with $a_i \in Q_1, a_{i+1} \in Q_2$,
                    we must have $(a_i, x_{i+1}, a_{i+1}) \in \Delta_2$, so $a_i \in A_1$, so $D_1$ accepts $S_1$.

                \end{proof}
            \item $S_2$ is accepted by $D_2$
                \begin{proof}
                    Note that from the last line of the previous proof, we have that $a_{i+1} = q_2$ as well. Hence $a_{i+1}$ is the starting state of $D_2$.
                    \begin{claim}
                        $\hat{\delta}_2(q_2, x_{i+2}\ldots x_{i+k+1}) = a_{i+k+2} \quad \forall i + 2 \le i + k + 2 \le m$
                    \end{claim}
                    \begin{proof}
                        The proof is analogous to the proof of the claim in the previous point.
                    \end{proof}
                    Using the result for $k = m - i - 2$, we have $\hat{\delta}_2(q_2, S_2) = a_m$. Since the run of $x$ considered above is accepting on $N$, $a_m \in A = A_2$, so $D_2$ accepts
                    $S_2$ as well.
                \end{proof}
        \end{enumerate}

        Since $S_1$ is accepted by $D_1$, $S_1 \in L_1$. Similarly, $S_2 \in L_2$. Hence $x = S_1 \cdot S_2 \in L_1 \cdot L_2$.

    \end{proof}


    Now that we have shown these claims, we are done.
\end{proof}


\end{document}
