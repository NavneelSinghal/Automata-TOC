\documentclass[a4paper]{article}
\usepackage[english]{babel}
\usepackage[a4paper,top=2cm,bottom=2cm,left=2cm,right=2cm,marginparwidth=1.75cm]{geometry}
\usepackage{amsmath}
\usepackage{amsfonts}
% \usepackage{amsthm}
\usepackage{amssymb}
\usepackage{graphicx}
\usepackage[colorinlistoftodos]{todonotes}
\usepackage[colorlinks=true, allcolors=blue]{hyperref}
\usepackage{import}
\usepackage{pdfpages}
\usepackage{transparent}
\usepackage{xcolor}
\usepackage{algorithmicx}
\usepackage{algpseudocode}

\usepackage{thmtools}
\usepackage{enumitem}
\usepackage[framemethod=TikZ]{mdframed}

\usepackage{xpatch}

\usepackage{boites}
\makeatletter
\xpatchcmd{\endmdframed}
{\aftergroup\endmdf@trivlist\color@endgroup}
{\endmdf@trivlist\color@endgroup\@doendpe}
{}{}
\makeatother

%\usepackage[poster]{tcolorbox}
%\allowdisplaybreaks
%\sloppy

\usepackage[many]{tcolorbox}

\xpatchcmd{\proof}{\itshape}{\bfseries\itshape}{}{}

% to set box separation
\setlength{\fboxsep}{0.8em}
\def\breakboxskip{7pt}
\def\breakboxparindent{0em}

\newenvironment{proof}{\begin{breakbox}\textit{Proof.}}{\hfill$\square$\end{breakbox}}
\newenvironment{ans}{\begin{breakbox}\textit{Answer.}}{\end{breakbox}}
\newenvironment{soln}{\begin{breakbox}\textit{Solution.}}{\end{breakbox}}

% \tcolorboxenvironment{proof}{
%     blanker,
%     before skip=\topsep,
%     after skip=\topsep,
%     borderline={0.4pt}{0.4pt}{black},
%     breakable,
%     left=12pt,
%     right=12pt,
%     top=12pt,
%     bottom=12pt,
% }
% 
% \tcolorboxenvironment{ans}{
%     blanker,
%     before skip=\topsep,
%     after skip=\topsep,
%     borderline={0.4pt}{0.4pt}{black},
%     breakable,
%     left=12pt,
%     right=12pt,
% }

\mdfdefinestyle{enclosed}{
    linecolor=black
    ,backgroundcolor=none
    ,apptotikzsetting={\tikzset{mdfbackground/.append style={fill=gray!100,fill opacity=.3}}}
    ,frametitlefont=\sffamily\bfseries\color{black}
    ,splittopskip=.5cm
    ,frametitlebelowskip=.0cm
    ,topline=true
    ,bottomline=true
    ,rightline=true
    ,leftline=true
    ,leftmargin=0.01cm
    ,linewidth=0.02cm
    ,skipabove=0.01cm
    ,innerbottommargin=0.1cm
    ,skipbelow=0.1cm
}

\mdfsetup{%
    middlelinecolor=black,
    middlelinewidth=1pt,
roundcorner=4pt}

\setlength{\parindent}{0pt}

\mdtheorem[style=enclosed]{theorem}{Theorem}
\mdtheorem[style=enclosed]{lemma}{Lemma}[theorem]
\mdtheorem[style=enclosed]{claim}{Claim}[theorem]
\mdtheorem[style=enclosed]{ques}{Question}
\mdtheorem[style=enclosed]{defn}{Definition}
\mdtheorem[style=enclosed]{notn}{Notation}
\mdtheorem[style=enclosed]{obs}{Observation}
\mdtheorem[style=enclosed]{eg}{Example}
\mdtheorem[style=enclosed]{cor}{Corollary}
\mdtheorem[style=enclosed]{note}{Note}

% \let\thetheorem=\relax
% \let\thelemma=\relax
% \let\theclaim=\relax
% \let\theques=\relax
% \let\thedefn=\relax
% \let\thenotn=\relax
% \let\theobs=\relax
% \let\thecor=\relax
% \let\thenote=\relax

% \renewcommand\qedsymbol{$\blacksquare$}
\newcommand{\nl}{\vspace{0.2cm}\\}
\newcommand{\mc}{\mathcal}
\newcommand{\mi}{\mathit}
\newcommand{\mf}{\mathbf}
\newcommand{\mb}{\mathbb}
\renewcommand{\L}{\mc{L}}
\newcommand{\hd}{\hat{\delta}}

\newcommand{\incfig}[1]{%
    \def\svgwidth{\columnwidth}
    \import{./figures/}{#1.pdf_tex}
}
\pdfsuppresswarningpagegroup=1

\title{\textbf{COL352 Lecture 8}}
\date{}

\begin{document}
\maketitle
\tableofcontents

\section{Recap}

Definitions from last class. See below.

\section{Definitions}

\begin{defn}
	Let $L \in \Sigma^*$ be any language. We define the language $L^*$ as follows:
	\[
		L^* = L_0 \cup L_1 \cup \cdots = \bigcup_{n=0}^\infty L^n
	\]
	where $L^0 = \{\epsilon\}, L_1 = L, L^n = L\cdot L \cdots L$ where there are $n$ instances of $L$.
\end{defn}

\begin{defn}
	Let $\Sigma$ be a finite alphabet. A \underline{regular expression} over $\Sigma$ is any expression that is in one of the following forms:
	\begin{enumerate}
		\item $\emptyset$.
		\item $\epsilon$.
		\item $a$, where $a \in \Sigma$.
		\item $(R_1 \cup R_2)$ where $R_1, R_2$ are regular expressions over $\Sigma$.
		\item $(R_1R_2)$ or $(R_1 \cdot R_2)$ where $R_1, R_2$ are regular expressions over $\Sigma$.
		\item $(R^*)$ where $R$ is a regular expression over $\Sigma$.
	\end{enumerate}
\end{defn}

\begin{defn}
	The \underline{language of a regular expression} $R$, denoted by $\mc{L}(R)$ is defined as follows:
	\begin{enumerate}
		\item $\mc{L}(\emptyset) = \emptyset$.
		\item $\mc{L}(\epsilon) = \{\epsilon\}$.
		\item $\mc{L}(a) = \{a\}$.
		\item $\mc{L}((R_1 \cup R_2)) = \mc{L}(R_1) \cup \mc{L}(R_2)$.
		\item $\mc{L}((R_1R_2)) = \mc{L}(R_1) \cdot \L(R_2)$.
		\item $\mc{L}\left((R^*)\right) = \left(\mc{L}(R)\right)^*$
	\end{enumerate}
\end{defn}

\section{Content}

\begin{theorem}
	Suppose $L \subseteq \Sigma^*$ is the language of some regular expression over $\Sigma$. Then $L$ is a regular language.
\end{theorem}

\begin{note}
	\textbf{Sketch of proof by structural induction}\\

	Obvious for cases 1, 2, 3 in the definition of regular expressions. For 4, 5, assume true for $R_1, R_2$, prove the claim for $R_1 \cup R_2$ and $R_1 \cdot R_2$. For 6, assume true for $R$, prove
	for $(R)^*$.\\

	\textbf{Sketch of proof by normal induction}\\

	Associate a size with every regular expression.
	Base case: Prove the claim for regular expressions of size 0.
	Inductive case: Assume claim is true for all regular expressions of size $< n$, and prove the claim for regular expressions of size $= n$.\\
\end{note}

\begin{proof}
	Define $\mathit{size}$ of a regular expression as follows:
	$\mathit{size}(\emptyset) = \mathit{size}(\epsilon) = \mathit{size}(a) = 0 \forall a \in \Sigma$, and $\mathit{size}((R_1 \cup R_2)) = \mathit{size}((R_1 \cdot R_2)) =
		\mathit{size}(R_1) + \mathit{size}(R_2) + 1$, and $\mathit{size}((R^*)) = \mathit{size}(R) + 1$.
	Then induct on $\mathit{size}$.
\end{proof}

\begin{ques}
	Is the converse true?
\end{ques}

\begin{ans}
	Yes, the converse is true.
\end{ans}

\begin{theorem}
	Let $L \subseteq \Sigma^*$ be a regular language. Then there exists a regular expression $R$ over $\Sigma$ such that $L = \L(R)$.
\end{theorem}

\begin{note}
	\textbf{Sketch of proof:}

	\begin{enumerate}
		\item Suppose $D = (Q, \Sigma, \delta, q_0, A)$ recognizes $L_0$. Let $m = |Q|$. Wlog, assume $Q = \{1, \ldots, m\}$.
		\item For now, ignore $q_0, A$, and focus on $\delta$.
		\item For $i, j \in \{1, \ldots, m\}$ and $k \in \{0, \ldots, m\}$, define the language $L_{i, j, k} \subseteq \Sigma^*$ as follows:
		      \[
			      L_{ijk} = \{x \mid \text{ run of } D \text{ on } x \text{ starts, ends at } i, j \text{ respectively, and each intermediate state is} \le k \}
		      \]
		      It is allowed to have $i, j > k$.
		      Note that this is a regular language for all $i, j, k$ (for states $> k$, make all outgoing edges end on that state, set starting state to $i$, and accepting states = $\{j\}$).
		\item Then we shall design a regular expression $R_{ijk}$ for each $L_{ijk}$.
		\item
		      From $\{R_{ijk} \mid i, j \in \{1, \ldots, m\} \land k \in \{0, \ldots, m\}\}$, construct a regular expression for $L$.

		      We can show that $L = \bigcup_{a \in A} L_{q_0am}$ (exercise).

		      Then we have $R = \left(\bigcup_{a \in A} R_{q_0am}\right), L = \L(R)$.

	\end{enumerate}
	It suffices to do step 4, since the rest has been done above already.

	Recall the Floyd-Warshall algorithm for finding shortest walks between every pair of vertices. Define
	\begin{align*}
		D[i, j, k] = \text{length of the shortest } i \to j \text{ walk which passes through } \{1 \ldots k\}.
	\end{align*}
	Then we have the following recurrence:

	\begin{align*}
		D[i, j, k] = \min(D[i, j, k - 1], D[i, k, k - 1] + D[k, j, k - 1])
	\end{align*}

	Here we break the analysis into two cases: the shortest walk either passes through $k$ or doesn't pass through $k$.

	\begin{algorithmic}[1]
		\Function{Floyd-Warshall}{$G[m \times m]$} \Comment{$G[i, j]$ contains the weight of an edge $(i, j)$ if it exists, and if it doesn't, it is $\infty$ except when $i = j$, where $G[i, i] = 0$}
		\State $D := G$
		\For{$k = 1 \ldots m$}
		\For{$i = 1 \ldots m$}
		\For{$j = 1 \ldots m$}
		\State $D[i, j, k] := \min(D[i, j, k - 1], D[i, k, k - 1] + D[k, j, k - 1])$
		\EndFor
		\EndFor
		\EndFor
		\EndFunction
	\end{algorithmic}

	\vspace{1em}

	We can construct $R_{ijk}$ from $R_{ij(k-1)}$, $R_{ik(k-1)}$, $R_{kj(k-1)}$, $R_{kk(k-1)}$ as follows, in a similar fashion.\\


\end{note}

\begin{proof}

	Let $D = (Q, \Sigma, \delta, q_0, A)$ be a DFA recognizing $L$. We shall use this DFA throughout the proof.\\

	We start off with some definitions.\\

	For $i, j \in \{1, \ldots, m\}$ and $k \in \{0, \ldots, m\}$, define the language $L_{i, j, k} \subseteq \Sigma^*$ as follows:
	\[
		L_{ijk} = \{x \mid \text{ run of } (Q, \Sigma, \delta, i, \{j\}) \text{ on } x \text{ starts, ends at } i, j \text{ respectively, and each intermediate state is} \le k \}
	\]
	Note that it is allowed to have $i, j > k$.\\

	\begin{claim}
		$L_{ijk}$ is a regular language.
	\end{claim}

	\begin{proof}
		Consider the DFA $D_{ijk} = (Q \cup \{s_e\}, \Sigma, \delta', i, \{j\})$, where $\delta'$ is defined as:
		\[
			\delta'(s, a) = \begin{cases}
				\delta(s, a) & \text{ if } s = i \lor s \le k \\
				s_e          & \text{ otherwise}
			\end{cases}
		\]

		We shall call this DFA \underline{the} DFA corresponding to $L_{ijk}$.\\

		Now we show the following two subclaims:\\

		\begin{claim}
			$L_{ijk} \subseteq \L(D_{ijk})$
		\end{claim}

		\begin{proof}
			Consider any string $s$ in $L_{ijk}$. Then the result directly follows, since all transitions on the states involved apart from the last state in the run agree with the transitions on the
			run on $D_{ijk}$, because of the definition of $\delta'$.
		\end{proof}

		\begin{claim}
			$\L(D_{ijk}) \subseteq L_{ijk}$
		\end{claim}

		\begin{proof}
			Consider any string accepted by $D_{ijk}$. Suppose it has an intermediate state $> k$. Then the next state will be $s_e$, and since $\delta(s_e, a) = s_e$ for all $a$, the last state of
			the run will be $s_e$, which is not an accepting state of $D_{ijk}$. This implies that the assumption is false, and thus the string is in $L_{ijk}$, whence we are done.
		\end{proof}

		Using these two claims, it follows that $L_{ijk} = \L(D_{ijk})$, and thus $L$ is regular by definition of a regular language.
	\end{proof}
	\begin{note}
		In the case $k = m$, $s_e$ is an isolated state and $\delta'$ is semantically the extension of $\delta$ on the extra state $s_e$.
	\end{note}

	Consider the following algorithm that takes as input a DFA $D$ and returns a regular expression.\\

	\begin{algorithmic}[1]
		\Function{GenerateRegularExpression}{$D = (Q, \Sigma, \delta, q_0, A)$}
		\State let $R[m \times m \times (m + 1)]$ be a table initialized by $\emptyset$.
		\For{$i = 1 \ldots m$}
		\For{$j = 1 \ldots m$}
		\If{$i = j$}
		\State $\displaystyle R[i, i, 0] := \left(\epsilon \cup \bigcup_{\delta(i, a) = i} a\right)$ \Comment{Use \texttt{foldl} to formalize this and fix parentheses}
		\Else
		\State $\displaystyle R[i, j, 0] := \left(\bigcup_{\delta(i, a) = j} a\right)$
		\EndIf
		\EndFor
		\EndFor
		\For{$k = 1 \ldots m$}
		\For{$i = 1 \ldots m$}
		\For{$j = 1 \ldots m$}
		\State $R[i, j, k] := \left(R[i, j, k - 1] \cup \left(R[i, k, k - 1] \cdot (R[k, k, k - 1]^*) \cdot R[k, j, k - 1]\right)\right)$
		\EndFor
		\EndFor
		\EndFor
		\State \Return $\displaystyle \left(\bigcup_{a \in A} R[q_0, a, m]\right)$
		\EndFunction
	\end{algorithmic}

	\vspace{1em}
	Now we show the following claim:\\

	\begin{claim}
		$L_{ijk} = \L(R[i, j, k])$
	\end{claim}

	\begin{proof}

		The proof shall proceed via induction on $k$.

		\begin{enumerate}

			\item Base case: $k = 0$. In this case, the run consists of at most 2 states. In the case when $i = j$, there can be runs with one state and two states, and they are precisely those
			      which correspond to strings in $\L(R[i, i, 0])$ by the definition of the language of a regular expression. The case for $i \ne j$ is similar, except that there need to be at
			      least two states, and hence all possible strings in $L_{ij0}$ have length $1$, and correspond to precisely the strings in $\L(R[i, j, 0])$.

			\item Inductive step: Suppose $k > 0$.\\

			      Fix $i, j$. We have $\L(R[i, j, k]) = \L(R[i, j, k - 1]) \cup (\L(R[i, k, k - 1]) \cdot \L(R[k, k, k - 1])^* \cdot \L(R[k, j, k - 1]))$, which, by the inductive hypothesis, gives us
			      $$\L(R[i, j, k]) = L_{ij(k-1)} \cup (L_{ik(k-1)} \cdot L_{kk(k-1)}^* \cdot L_{kj(k-1)})$$

			      Then we claim the following:\\

			      \begin{claim}
				      $L_{ijk} \subseteq \L(R[i, j, k])$
			      \end{claim}
			      \begin{proof}
				      Consider any string $x = x[1] x[2] \ldots x[n]$ in $L_{ijk}$. Then any state in the run of the corresponding DFA on $x$ is at most $k$.\\

				      If the state $k$ is never reached in the run, then the string is in $L_{ij(k-1)}$, which is a subset of $\L(R[i, j, k])$ due to the expression above.\\

				      Suppose the state $k$ is reached. Suppose $e = i x[1] i_1 x[2] \ldots i_{n-1} x[n] j$ is the run corresponding to $x$,
				      and let $j_1 < \cdots < j_r$ be such that $i_w = k \iff w \in \{j_1, \ldots, j_r\}$.\\

				      Then consider the runs

				      \begin{align*}
					      i x[1] i_1 \ldots x[j_1] i_{j_1},        \\
					      i_{j_1} x[j_1 + 1] \ldots i_{j_2},       \\
					      \vdots                                   \\
					      i_{j_{r-1}} x[j_{r-1}+1] \ldots i_{j_r}, \\
					      i_{j_r} x[j_r + 1] \ldots j              \\
				      \end{align*}

				      The first and the last correspond to runs of the DFAs corresponding to the languages $L_{ik(k-1)}$ and $L_{kj(k-1)}$, and all intermediate runs correspond to the runs of the DFAs
				      corresponding to the language $L_{kk(k-1)}$.\\

				      Hence, the string $x$ can be decomposed into $r + 2$ possibly empty substrings $x_1 \cdot x_2 \cdots x_{r+2}$, such that $x_1 \in L_{ik(k-1)}, x_{r+2} \in L_{kj(k-1)}$ and
				      $x_w \in L_{kk(k-1)}$, where $2 \le w \le r + 1$.\\

				      Hence we have $x_2 \cdots x_w \in L_{kk(k-1)}^*$. So we have $x = x_1 \cdot (x_2 \ldots x_{r+1}) \cdot x_{r+2} \in L_{ik(k-1)} \cdot L_{kk(k-1)}^* \cdot L_{kj(k-1)}
					      \subseteq \L(R[i,
						      j, k])$.\\

				      Since $x$ was arbitrary, we have shown $L_{ijk} \subseteq \L(R[i, j, k])$, as needed.

			      \end{proof}
			      \begin{claim}
				      $L_{ijk} \supseteq \L(R[i, j, k])$
			      \end{claim}
			      \begin{proof}
				      Consider any string $x$ in $\L(R[i, j, k]) = L_{ij(k-1)} \cup (L_{ik(k-1)} \cdot L_{kk(k-1)}^* \cdot L_{kj(k-1))}$.\\

				      Either it is in $L_{ij(k-1)}$, in which case it is already in $L_{ijk}$ since all intermediate states are $\le k - 1 < k$, or it is in $L_{ik(k-1)} \cdot L_{kk(k-1)}^* \cdot
					      L_{kj(k-1)}$.\\

				      In the second case, it is the result of concatenation of a string in $L_{ik(k-1)}$, some strings in $L_{kk(k-1)}$ and a final string in $L_{kj(k-1)}$. Suppose there are
				      exactly $r$ strings from $L_{kk(k-1)}$.\\

				      Consider the runs of each of these strings:\\

				      For the first string, let the run be $i x[1] s_{11} x[2] s_{12} \ldots x[l_1] s_{1l_1} = k$.\\

				      For the $d^{th}$ string ($2 \le d \le r + 1$), let the run be $k x[l_{d-1} + 1] s_{d1} \ldots x[l_d] s_{dl_d} = k$.\\

				      For the last string, let the run be $k x[l_{k+1}+1] s_{(r+2)1} \ldots x[l_{k+2}] s_{(r+2)l_{r+2}} = j$.\\

				      Consider the following sequence:

				      $i x[1] s_{11} \ldots x[l_1] k x[l_1+1] \ldots x[l_2] k \ldots k x[l_{k+1}+1] s_{(r+2)1} \ldots x[l_{k+2}] j$.\\

				      It suffices to show that this is a (in fact \textit{the}) valid run for the DFA corresponding to the language $L_{ijk}$ (if it is a valid run, it must be accepting since it ends at $j$).\\

				      Consider any intermediate state in this run; it suffices to show that all such intermediate states are $\le k$ (since all transitions are valid due to them being
				      accepting runs, and all states are in the same state-space). Note that the states in this sequence are of the following kinds:

				      \begin{enumerate}
					      \item Intermediate states in the runs of substrings mentioned above.
					            In this case, it is trivially true that the states are $\le k - 1$.
					      \item Start states of runs corresponding to the strings $2$ through $k+2$. In this case, it's true since all such states are $k$.
					      \item End states of runs corresponding to the strings $1$ through $k+1$. In this state, it's true since all such states are $k$.
				      \end{enumerate}

				      From here, we get the fact that $x$ is accepted by the DFA corresponding to $L_{ijk}$, so $x \in L_{ijk}$. Since $x$ was arbitrary, we get the fact that $\L(R[i, j, k])
					      \subseteq L_{ijk}$, as needed.
			      \end{proof}
			      From these two claims, it follows that $L_{ijk} = \L(R[i, j, k])$ for these particular values of $i, j$. Since $i, j$ were arbitrary, this completes the inductive step.
		\end{enumerate}
		Thus, we are done by induction on $k$.
	\end{proof}

	We shall now claim the following:\\

	\begin{claim}
		$$L = \bigcup_{a\in A} L_{q_0am}$$
	\end{claim}

	\begin{proof}
		This is fairly obvious; for showing that $\displaystyle L \subseteq \bigcup_{a\in A} L_{q_0am}$, consider any string in $L$, and consider the run corresponding to it. The run ends at an
		accepting state (say $a$), and $x$ is in $L_{q_0am}$ by the definition of $L_{ijk}$, since all states are $\le m$ and the start and end states are $q_0, a$ respectively.

		For showing the other direction, consider any string $x$ in the RHS. Then there exists an $a \in A$ such that there is an accepting run of the DFA corresponding to $L_{q_0am}$ on $x$. It is a
		valid and accepting run of the DFA of $L$ on $x$ as well, since the run starts on $q_0$, ends at $a$ (an accepting state), and all transitions are the same in $L_{ijm}$ by the construction of the DFA
		corresponding to it.
	\end{proof}

	Now moving on to the main proof, note that

	\begin{align*}
		L & = \bigcup_{a\in A} L_{q_0am}                                 \\
		  & = \bigcup_{a \in A} \L(R[q_0, a, m])                         \\
		  & = \L\left(\left(\bigcup_{a \in A} R[q_0, a, m]\right)\right)
	\end{align*}

	From here, we get that $\displaystyle \left(\bigcup_{a \in A} R[q_0, a, m]\right)$ is a regular expression corresponding to $L$, whence we are done.

\end{proof}

\end{document}
