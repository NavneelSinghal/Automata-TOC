\documentclass[a4paper]{article}
\usepackage[english]{babel}
\usepackage[a4paper,top=2cm,bottom=2cm,left=2cm,right=2cm,marginparwidth=1.75cm]{geometry}
\usepackage{amsmath}
\usepackage{amsfonts}
% \usepackage{amsthm}
\usepackage{amssymb}
\usepackage{graphicx}
\usepackage[colorinlistoftodos]{todonotes}
\usepackage[colorlinks=true, allcolors=blue]{hyperref}
\usepackage{import}
\usepackage{pdfpages}
\usepackage{transparent}
\usepackage{xcolor}
\usepackage{algorithmicx}
\usepackage{algpseudocode}

\usepackage{thmtools}
\usepackage{enumitem}
\usepackage[framemethod=TikZ]{mdframed}

\usepackage{xpatch}

\usepackage{boites}
\makeatletter
\xpatchcmd{\endmdframed}
{\aftergroup\endmdf@trivlist\color@endgroup}
{\endmdf@trivlist\color@endgroup\@doendpe}
{}{}
\makeatother

%\usepackage[poster]{tcolorbox}
%\allowdisplaybreaks
%\sloppy

\usepackage[many]{tcolorbox}

\xpatchcmd{\proof}{\itshape}{\bfseries\itshape}{}{}

% to set box separation
\setlength{\fboxsep}{0.8em}
\def\breakboxskip{7pt}
\def\breakboxparindent{0em}

\newenvironment{proof}{\begin{breakbox}\textit{Proof.}}{\hfill$\square$\end{breakbox}}
\newenvironment{ans}{\begin{breakbox}\textit{Answer.}}{\end{breakbox}}
\newenvironment{soln}{\begin{breakbox}\textit{Solution.}}{\end{breakbox}}

% \tcolorboxenvironment{proof}{
%     blanker,
%     before skip=\topsep,
%     after skip=\topsep,
%     borderline={0.4pt}{0.4pt}{black},
%     breakable,
%     left=12pt,
%     right=12pt,
%     top=12pt,
%     bottom=12pt,
% }
%
% \tcolorboxenvironment{ans}{
%     blanker,
%     before skip=\topsep,
%     after skip=\topsep,
%     borderline={0.4pt}{0.4pt}{black},
%     breakable,
%     left=12pt,
%     right=12pt,
% }

\mdfdefinestyle{enclosed}{
    linecolor=black
    ,backgroundcolor=none
    ,apptotikzsetting={\tikzset{mdfbackground/.append style={fill=gray!100,fill opacity=.3}}}
    ,frametitlefont=\sffamily\bfseries\color{black}
    ,splittopskip=.5cm
    ,frametitlebelowskip=.0cm
    ,topline=true
    ,bottomline=true
    ,rightline=true
    ,leftline=true
    ,leftmargin=0.01cm
    ,linewidth=0.02cm
    ,skipabove=0.01cm
    ,innerbottommargin=0.1cm
    ,skipbelow=0.1cm
}

\mdfsetup{%
    middlelinecolor=black,
    middlelinewidth=1pt,
roundcorner=4pt}

\setlength{\parindent}{0pt}

\mdtheorem[style=enclosed]{theorem}{Theorem}
\mdtheorem[style=enclosed]{lemma}{Lemma}[theorem]
\mdtheorem[style=enclosed]{claim}{Claim}[theorem]
\mdtheorem[style=enclosed]{ques}{Question}
\mdtheorem[style=enclosed]{defn}{Definition}
\mdtheorem[style=enclosed]{notn}{Notation}
\mdtheorem[style=enclosed]{obs}{Observation}
\mdtheorem[style=enclosed]{eg}{Example}
\mdtheorem[style=enclosed]{cor}{Corollary}
\mdtheorem[style=enclosed]{note}{Note}

% \let\thetheorem=\relax
% \let\thelemma=\relax
% \let\theclaim=\relax
% \let\theques=\relax
% \let\thedefn=\relax
% \let\thenotn=\relax
% \let\theobs=\relax
% \let\thecor=\relax
% \let\thenote=\relax

% \renewcommand\qedsymbol{$\blacksquare$}
\newcommand{\nl}{\vspace{0.2cm}\\}
\newcommand{\mc}{\mathcal}
\newcommand{\mi}{\mathit}
\newcommand{\mf}{\mathbf}
\newcommand{\mb}{\mathbb}
\renewcommand{\L}{\mc{L}}
\newcommand{\hd}{\hat{\delta}}
\newcommand{\produces}{\implies}
\newcommand{\derives}{\stackrel{*}{\implies}}
\newcommand{\changesto}{\vdash}
\newcommand\Vtextvisiblespace[1][.3em]{%
    \mbox{\kern.06em\vrule height.3ex}%
    \vbox{\hrule width#1}%
    \hbox{\vrule height.3ex}
}
\newcommand{\blank}{{\Vtextvisiblespace[0.7em]}}
\newcommand{\leftend}{\triangleright}
\newcommand{\comp}{\overline}

\newcommand{\incfig}[1]{%
    \def\svgwidth{\columnwidth}
    \import{./figures/}{#1.pdf_tex}
}
\pdfsuppresswarningpagegroup=1

\title{\textbf{COL352 Lecture 29}}
\date{}

\begin{document}
\maketitle
\tableofcontents

\section{Recap}

\section{Definitions}

\begin{defn}
    A language $L$ is said to be co-Turing recognizable if its complement is Turing recognizable.
\end{defn}

\begin{defn}
    The membership language of Turing machines is defined as
    \[
        A_{TM} = \{(\langle M \rangle, x) \mid \text{Turing machine } M \text{ accepts string }x\}
    \]
\end{defn}

\begin{defn}
    A DTM recognizing $A_{TM}$ is called a universal TM.
\end{defn}

\begin{defn}
    $\mf{HALT} = \{(\langle M \rangle, x) \mid M \text{ halts on } x\}$.
\end{defn}

\begin{defn}
    \textbf{Computable function}: Let $f : \Sigma_1^* \to \Sigma_2^*$ be a function. We say $f$ is computable if there exists a Turing machine $C$ which does the following:\nl
    $\forall x \in \Sigma_1^*$, $C$ on input $x$
    \begin{enumerate}
        \item halts,
        \item leaves $f(x)$ on the tape,
        \item and at the end of execution leaves the read/write head on the leftmost cell of the tape.
    \end{enumerate}
\end{defn}

\begin{defn}
    Let $L_1 \subseteq \Sigma_1^*, L_2 \subseteq \Sigma_2^*$ be two languages. A function $f : \Sigma_1^* \to \Sigma_2^*$ is called a \underline{mapping reduction} \underline{from $L_1$ to $L_2$}
    if:
    \begin{enumerate}
        \item $f$ is computable.
        \item $\forall x \in \Sigma_1^*, x \in L_1 \iff f(x) \in L_2$.
    \end{enumerate}
    We say that $L_1$ is mapping reducible to $L_2$, and write $L_1 \le_m L_2$ if such a mapping reduction exists.
\end{defn}

\section{Content}

\begin{theorem}
    $\mf{HALT}$ is undecidable.
\end{theorem}
\begin{proof}
    Assume $\mf{HALT}$ is decidable. Then construct a decider for $A_{TM}$. This gives a contradiction. Details follow:\nl
    Recall from the last class that $A_{TM} = \{(\langle M \rangle, x) \mid M \text{ accepts } x\}$.\nl
    Let $H$ be a decider for $\mf{HALT}$. Do the following: if $H$ halts on $(\langle M \rangle, x)$, then reject, else run a universal Turing machine on the input.\nl
    So, consider the Turing machine $D$ which, on input $(\langle M \rangle, x)$ does the following:
    \begin{enumerate}
        \item Use $H$ to decide whether $M$ halts on $x$. If $H$ rejects $(\langle M \rangle, x)$, then reject.
        \item (Now we know $H$ accepts). Run a universal Turing machine $U$ on $(\langle M \rangle, x)$ and output its answer.
    \end{enumerate}
    Proof of correctness:\nl
    $(\langle M \rangle, x) \in A_{TM} \iff M \text{ accepts } x \iff H, U \text{ accept } (\langle M \rangle, x) \iff D \text{ accepts } (\langle M \rangle, x)$.
    If $D$ accepts, we are done. Now note that for $D$ to run forever on the input, since $H$ is a decider, only $U$ can run forever on the input. However, for $U$ to run on the input, $H$ has to
    accept it, so $M$ halts on $x$, and hence $U$ halts on $(\langle M \rangle, x)$, which is a contradiction. So $D$ always halts, and is hence a decider.
\end{proof}

Now coming to computability, check that basic arithmetic operations are computable. Check that $f(G) = $ breadth first traversal of $G$ is also computable.\nl

\begin{ques}
    Show that the class of computable functions from $\Sigma^*$ to $\Sigma^*$ is closed under composition.
\end{ques}

Consider the following situation: we have a computable function $f$ from $\Sigma_1^*$ to $\Sigma_2^*$, and there are languages $L_1$ and $L_2$ such that $f(L_1) \subseteq L_2$ and $f(\comp{L_1}) \subseteq
\comp{L_2}$.\nl

Now suppose TM $R$ computes $f$, and $L_2$ is decidable. Then $x \in L_1 \iff f(x) \in L_2 \iff$ decider for $L_2$ accepts $f(x)$.\nl

Example of a mapping reduction: $\comp{\mf{DIAG}} \le_m \mf{A_{TM}}$, $f(w) = (w, w)$.\nl

\begin{theorem}
    Suppose $L_1 \le_m L_2$. Then the following hold.
    \begin{enumerate}
        \item If $L_2$ is decidable, then $L_1$ is decidable.
        \item If $L_2$ is recognizable, then $L_1$ is recognizable.
        \item If $L_2$ is co-recognizable, then $L_2$ is also co-recognizable.
    \end{enumerate}
\end{theorem}

\begin{proof}
    \begin{enumerate}
        \item $L_1 \le_m L_2$, so there exists a computable function $f$ such that $\forall x$, $x \in L_1 \equiv f(x) \in L_2$. Suppose TM $R$ computes $f$. Let $M_2$ be a decider for $L_2$. Now
            consider the Turing machine $M_1$ which on input $x$ does the following:
            \begin{enumerate}
                \item Use $R$ to compute $f(x)$.
                \item Run $M_2$ on $f(x)$ and return its answer.
            \end{enumerate}
            Since $R, M_2$ halt on every input, $M_1$ also halts on every input.
            Now $x \in L_1 \iff f(x) \in L_2 \iff M_2$ accepts $f(x) \iff M_1$ accepts $x$.
        \item The same construction works (except that $M_2$ need not halt but $R$ always halts).
    \end{enumerate}
\end{proof}

\begin{theorem}
    Equivalently,\nl
    Suppose $L_1 \le_m L_2$. Then the following hold.
    \begin{enumerate}
        \item If $L_1$ is undecidable, then $L_2$ is undecidable.
        \item If $L_1$ is unrecognizable, then $L_2$ is unrecognizable.
        \item If $L_1$ is unco-recognizable, then $L_2$ is also unco-recognizable.
    \end{enumerate}
\end{theorem}

Goal: To prove $L_2$ is undecidable.\nl

Recipe: Take a known undecidable language $L_1$ (for instance, $L_1 = A_{TM}$), and show that $L_1 \le_m L_2$.\nl

\begin{eg}
    Let $E_{TM} = \{w \mid M_w \text{ recognizes the empty language}\}$.
\end{eg}

\begin{claim}
    $\mf{DIAG} \le_m E_{TM}$
\end{claim}

\begin{proof}
    Define $f(w)$ as follows. $f(w)$ is the description of a TM that does the following:
    On input $x$,
    \begin{enumerate}
        \item Erase $x$ and write $w$ on the tape.
        \item Simulate $M_w$ on $w$ and return the answer.
    \end{enumerate}
    Observe that $f$ is computable. Also, $w \in \mf{DIAG} \iff M_w$ doesn't accept $w \iff M_{f(w)}$ doesn't accept any $x \iff f(w) \in E_{TM}$.
\end{proof}

\end{document}
