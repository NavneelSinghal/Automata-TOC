\documentclass[a4paper]{article}
\usepackage[english]{babel}
\usepackage[a4paper,top=2cm,bottom=2cm,left=2cm,right=2cm,marginparwidth=1.75cm]{geometry}
\usepackage{amsmath}
\usepackage{amsfonts}
% \usepackage{amsthm}
\usepackage{amssymb}
\usepackage{graphicx}
\usepackage[colorinlistoftodos]{todonotes}
\usepackage[colorlinks=true, allcolors=blue]{hyperref}
\usepackage{import}
\usepackage{pdfpages}
\usepackage{transparent}
\usepackage{xcolor}
\usepackage{algorithmicx}
\usepackage{algpseudocode}

\usepackage{thmtools}
\usepackage{enumitem}
\usepackage[framemethod=TikZ]{mdframed}

\usepackage{xpatch}

\usepackage{boites}
\makeatletter
\xpatchcmd{\endmdframed}
{\aftergroup\endmdf@trivlist\color@endgroup}
{\endmdf@trivlist\color@endgroup\@doendpe}
{}{}
\makeatother

%\usepackage[poster]{tcolorbox}
%\allowdisplaybreaks
%\sloppy

\usepackage[many]{tcolorbox}

\xpatchcmd{\proof}{\itshape}{\bfseries\itshape}{}{}

% to set box separation
\setlength{\fboxsep}{0.8em}
\def\breakboxskip{7pt}
\def\breakboxparindent{0em}

\newenvironment{proof}{\begin{breakbox}\textit{Proof.}}{\hfill$\square$\end{breakbox}}
\newenvironment{ans}{\begin{breakbox}\textit{Answer.}}{\end{breakbox}}
\newenvironment{soln}{\begin{breakbox}\textit{Solution.}}{\end{breakbox}}

% \tcolorboxenvironment{proof}{
%     blanker,
%     before skip=\topsep,
%     after skip=\topsep,
%     borderline={0.4pt}{0.4pt}{black},
%     breakable,
%     left=12pt,
%     right=12pt,
%     top=12pt,
%     bottom=12pt,
% }
% 
% \tcolorboxenvironment{ans}{
%     blanker,
%     before skip=\topsep,
%     after skip=\topsep,
%     borderline={0.4pt}{0.4pt}{black},
%     breakable,
%     left=12pt,
%     right=12pt,
% }

\mdfdefinestyle{enclosed}{
    linecolor=black
    ,backgroundcolor=none
    ,apptotikzsetting={\tikzset{mdfbackground/.append style={fill=gray!100,fill opacity=.3}}}
    ,frametitlefont=\sffamily\bfseries\color{black}
    ,splittopskip=.5cm
    ,frametitlebelowskip=.0cm
    ,topline=true
    ,bottomline=true
    ,rightline=true
    ,leftline=true
    ,leftmargin=0.01cm
    ,linewidth=0.02cm
    ,skipabove=0.01cm
    ,innerbottommargin=0.1cm
    ,skipbelow=0.1cm
}

\mdfsetup{%
    middlelinecolor=black,
    middlelinewidth=1pt,
roundcorner=4pt}

\setlength{\parindent}{0pt}

\mdtheorem[style=enclosed]{theorem}{Theorem}
\mdtheorem[style=enclosed]{lemma}{Lemma}[theorem]
\mdtheorem[style=enclosed]{claim}{Claim}[theorem]
\mdtheorem[style=enclosed]{ques}{Question}
\mdtheorem[style=enclosed]{defn}{Definition}
\mdtheorem[style=enclosed]{notn}{Notation}
\mdtheorem[style=enclosed]{obs}{Observation}
\mdtheorem[style=enclosed]{eg}{Example}
\mdtheorem[style=enclosed]{cor}{Corollary}
\mdtheorem[style=enclosed]{note}{Note}

% \let\thetheorem=\relax
% \let\thelemma=\relax
% \let\theclaim=\relax
% \let\theques=\relax
% \let\thedefn=\relax
% \let\thenotn=\relax
% \let\theobs=\relax
% \let\thecor=\relax
% \let\thenote=\relax

% \renewcommand\qedsymbol{$\blacksquare$}
\newcommand{\nl}{\vspace{0.2cm}\\}
\newcommand{\mc}{\mathcal}
\newcommand{\mi}{\mathit}
\newcommand{\mf}{\mathbf}
\newcommand{\mb}{\mathbb}
\renewcommand{\L}{\mc{L}}
\newcommand{\hd}{\hat{\delta}}
\newcommand{\produces}{\implies}
\newcommand{\derives}{\stackrel{*}{\implies}}
\newcommand{\changesto}{\vdash}


\newcommand{\incfig}[1]{%
    \def\svgwidth{\columnwidth}
    \import{./figures/}{#1.pdf_tex}
}
\pdfsuppresswarningpagegroup=1

\title{\textbf{COL352 Lecture 17}}
\date{}

\begin{document}
\maketitle
\tableofcontents

\section{Recap}

Grammar $\implies$ PDA.

\section{Definitions}

\begin{defn}
    A (non-deterministic) pushdown automaton ((N)PDA) is a 6-tuple $(Q, \Sigma, \Gamma, \Delta, q_0, A)$ where
    \begin{enumerate}
        \item $Q$ -- finite nonempty set of states
        \item $\Sigma$ -- finite nonempty input alphabet
        \item $\Gamma$ -- finite stack alphabet
        \item $q_0 \in Q$ -- initial state
        \item $A$ -- set of accepting states
        \item $\Delta \subseteq Q \times \Sigma_\epsilon \times \Gamma_\epsilon \times Q \times \Gamma_\epsilon$, where $X_\epsilon$ is defined as $X \cup \{\epsilon\}$
    \end{enumerate}
\end{defn}

Note that in an NFA, $\Delta \subseteq Q \times (\Sigma \cup \{\epsilon\}) \times Q$.

\begin{defn}
    Let $P = (Q, \Sigma, \Gamma, \Delta, q_0, A)$ be a PDA. An instantaneous description (i.d.) of $P$ is a tuple $(q, x, \alpha)$ where $q \in Q$, $x \in \Sigma^*$, $\alpha \in \Gamma^*$. (The set
    of instantaneous descriptions is $Q \times \Sigma^* \times \Gamma^*$).
\end{defn}

Informally, it consists of the current state, the string left to be read, and the description of the current stack.\nl

\begin{defn}
    Let $P = (Q, \Sigma, \Gamma, \Delta, q_0, A)$ be a PDA. The relation $\changesto_P$ (read as ``changes to") is defined on the set of i.d.s as follows:\nl
    If $(q, a, B, q', B') \in \Delta$, then $(q, ax, B\alpha) \changesto_P (q', x, B'\alpha)$, and no other pairs of i.d.s are related.\nl
    In other words:
    \begin{align*}
        (q, y, \beta) \changesto_P (q', y', \beta') &\iff\\ &\exists a \in \Sigma_\epsilon, B \in \Gamma_\epsilon, \alpha \in \Gamma^*, B' \in \Gamma_\epsilon \text{ such that } y = ay', \beta =
    B\alpha, \beta' = B'\alpha, (q, a, B, q', B') \in \Delta
    \end{align*}
\end{defn}

\begin{defn}
    $\changesto^*_P$ is defined as the reflexive transitive closure of $\changesto$ (read as ``changes to in finitely many steps").
\end{defn}

\begin{defn}
    $x \in \Sigma^*$ is said to be accepted by PDA $P = (Q, \Sigma^*, \Gamma, \Delta, q_0, A)$ iff
    $$(q_0, x, \epsilon) \changesto^*_P (q, \epsilon, \alpha)$$
    for some $q \in A$ and some $\alpha \in \Gamma^*$.
\end{defn}

\todo[inline]{New in lecture 17}

\begin{defn}
    The language recognized by PDA $P$ denoted by $\L(P)$ is $\{x \in \Sigma^* \mid P \text{ accepts }x\}$.
\end{defn}

\section{Content}

Plan for the next 4 lectures is to show that $L$ is generated by the grammar iff $L$ is recognized by a PDA.\nl

The forward implication will be done in the next two lectures.

Recall that $\Delta \subseteq Q \times \Sigma_\epsilon \times \Gamma_\epsilon \times Q \times \Gamma_\epsilon$, $(q, a, B, q', B') \in \Delta$. $q$ stands for the current state, $a$ for current
input, $B$ for the current stack top to be popped, $q'$ for the next state, and $B'$ for the element to be pushed onto stack.\nl

Suppose that instead, we do $\Delta \in Q \times \Sigma_\epsilon \times \Gamma_\epsilon \times Q \times \Gamma^*$ (i.e., we can push strings onto the stack). Call such a PDA a \underline{fast
PDA}.

\begin{theorem}
    $L$ is recognized by a PDA $\iff$ $L$ is recognized by a fast PDA.
\end{theorem}

\begin{proof}
    The forward direction is obvious.\nl
    Now we look at a sketch of the proof of the non-obvious reverse direction.\nl
    $L$ is recognized by a fast $PDA$, say $P$. A state transition looks like the following:
    \begin{center}
        
    \incfig{pda-state-transition}
    \end{center}
    We do the following:
    \begin{center}
    \incfig{transformed-pda-state-transition}
    \end{center}
    Check that $P$ accepts $x$ iff $P'$ accepts $x$. Now since $\Delta$ was finite, $\Delta'$ is finite too.
\end{proof}

\begin{cor}
    In order to show that $L$ generated by grammar $\implies$ $L$ recognized by a PDA, it is sufficient to show that $L$ generated by grammar $\implies$ $L$ recognized by a fast PDA.
\end{cor}

\begin{theorem}
    Every language which is generated by a grammar is also recognized by a fast PDA.
\end{theorem}

We begin by an example.\nl

\begin{eg}
    Consider $N = \{A, B\}$, $\Sigma = \{0, 1\}$, $R = \{A \to 0BB, A \to 1, B \to A1A, B \to 0\}$, with initial nonterminal as $A$.\nl

    We construct the PDA as follows.\nl

    $Q = \{q_i, q, q_a\}, A = \{q_a\},$ initial state $= q_i$, $\Sigma = \{0, 1\}, \Gamma = N \cup \Gamma \cup \{\bot\}$.\nl

    $\Delta = \Delta_{special} \cup \Delta_{match} \cup \Delta_{produce}$.\nl

    $\Delta_{match} = \{(q, 0, 0, q, \epsilon), (q, 1, 1, q, \epsilon)\}$. This corresponds to removing $a$ from the stack, where $a \in \Sigma$.\nl

    $\Delta_{produce} = \{(q, \epsilon, A, q, BB0), (q, \epsilon, A, q, 1), (q, \epsilon, B, q, A1A), (q, \epsilon, B, q, 0)\}$. This corresponds to the production rules.\nl

    $\Delta_{special} = \{(q_i, \epsilon, \epsilon, q, \bot A), (q, \epsilon, \bot, q_a, \epsilon)\}$. This corresponds to initialization and cleanup.\nl

    Consider the following leftmost derivation of $01110$: $A \produces 0BB \produces 0A1AB \produces 011AB \produces 0111B \produces 01110$.\nl

    Then we have
    \begin{align*}
        (q_i, 01110, \epsilon) &\changesto (q, 01110, A\bot)\\
                               &\changesto (q, 01110, A\bot)\\
                               &\changesto (q, 01110, 0BB\bot)\\
                               &\changesto (q, 1110, BB\bot)\\
                               &\changesto (q, 1110, A1AB\bot)\\
                               &\changesto (q, 1110, 11AB\bot)\\
                               &\changesto (q, 110, 1AB\bot)\\
                               &\changesto (q, 10, AB\bot)\\
                               &\changesto (q, 10, 1B\bot)\\
                               &\changesto (q, 0, B\bot)\\
                               &\changesto (q, 0, 0\bot)\\
                               &\changesto (q, \epsilon, \bot)\\
                               &\changesto (q_a, \epsilon, \epsilon)
    \end{align*}
\end{eg}

\begin{proof}
    \todo[inline]{Continuation from lecture 17.}
    We proceed to formalize the proof.

    \paragraph{Fast PDA construction}
    We construct the fast PDA $P$ from grammar $G = (N, \Sigma, R, S)$ as follows:\nl
    $P = (\{q_{init}, q, q_{accept}\}, \Sigma, \Gamma, \Delta, q_{init}, \{q_{accept}\})$, where
    \begin{enumerate}
        \item $\Gamma = N \cup \Sigma \cup \{\bot\}$
        \item $\Delta = \Delta_{match} \cup \Delta_{produce} \cup \Delta_{special}$ where
            \begin{enumerate}
                \item $\Delta_{match} = \{(q, a, a, q, \epsilon) \mid a \in \Sigma\}$
                \item $\Delta_{produce} = \{(q, \epsilon, A, q, rev(\alpha)) \mid (A \to \alpha) \in R\}$ (this is finite since $R$ is finite)
                \item $\Delta_{special} = \{(q_{init}, \epsilon, \epsilon, q, \bot S), (q, \epsilon, \bot, q_{accept}, \epsilon)\}$
            \end{enumerate}
    \end{enumerate}

    We need to prove that $\forall x, S \derives x \iff $ $P$ accepts $x$ (actually $(q_{init}, x, \epsilon) \changesto^* (q_{accept}, \epsilon, \epsilon)$).\nl
        Observe that $(q_{init}, x, \epsilon) \changesto^* (q_{accept}, \epsilon, \epsilon)$ iff $(q, x, S\bot) \changesto^* (q, \epsilon, \bot)$.\nl
        Need to show that $\forall x : S \derives x \iff (q, x, S\bot) \changesto^* (q, \epsilon, \bot)$.\nl
    \begin{claim}
        $\forall \alpha \in (N \cup \Sigma)^*, \forall x \in \Sigma^* : \alpha \derives x \iff (q, x, \alpha\bot) \changesto^* (q, \epsilon, \bot)$
    \end{claim}
    To show this we break this into two claims.\nl
    \begin{claim}
        $\forall \alpha \in (N \cup \Sigma)^*, \forall x \in \Sigma^* : \alpha \derives x \impliedby (q, x, \alpha\bot) \changesto^* (q, \epsilon, \bot)$
    \end{claim}
    \begin{proof}
        By induction on the number of transitions in the (shortest) run of $P$ from $(q, x, \alpha\bot)$ to $(q, \epsilon, \bot)$.\nl
        Base case: $0$ transitions, then $\alpha = \epsilon, x = \epsilon$, then $\alpha \derives x$.\nl
        Inductive case: $n > 0$ transitions. Then $\exists x', \alpha'$ such that $(q, x, \alpha\bot) \changesto (q, x', \alpha'\bot) \changesto^* (q, \epsilon, \bot)$, where the second
        part takes $n - 1$ transitions.
        \begin{enumerate}
            \item Case 1: If $T$ is a match transition, $x = ax'$ and $\alpha = a \alpha'$ for some $a \in \Sigma$. By inductive hypothesis, $\alpha' \derives x'$. But $\alpha = a \alpha' \derives
                ax' = x$, so $\alpha \derives x$.
            \item Case 2: If $T$ is a produce transition, $x = x'$, $\exists A \in N, \beta, \gamma \in (N \cup \Sigma)^*$ such that $\alpha = A\beta, \alpha' = \gamma \beta$, and $(A \to \gamma)
                \in R$. (basically pop off a non-terminal from the stack, and push the reverse of the production rule RHS on the stack). By the inductive hypothesis, $\alpha' \derives x' = x$,
                i.e., $\gamma \beta \derives x$. But $\alpha = A \beta \implies \gamma \beta$ (because $(A \to \gamma) \in R$), so $\alpha \derives x$.
        \end{enumerate}
    \end{proof}
    \begin{claim}
        $\forall \alpha \in (N \cup \Sigma)^*, \forall x \in \Sigma^* : \alpha \derives x \implies (q, x, \alpha\bot) \changesto^* (q, \epsilon, \bot)$
    \end{claim}
    \begin{proof}
        By induction on the number of productions, say $n$, in the (shortest) (leftmost) derivation of $x$ from $\alpha$, and then on $|x|$. We have the following two inductive hypotheses:
        \begin{enumerate}
            \item $\forall \alpha \in (N \cup \Sigma)^*, \forall x \in \Sigma^*$ : if $\alpha' \derives x'$ in $< n$ productions, then $(q, x', \alpha\bot) \changesto^* (q, \epsilon, \bot)$.
            \item $\forall \alpha \in (N \cup \Sigma)^*, \forall x \in \Sigma^*$ : if $\alpha' \derives x'$ in $n$ productions, and $|x'| < |x|$, then $(q, x', \alpha'\bot) \changesto^* (q,
                \epsilon, \bot)$.
        \end{enumerate}
        This is a standard double induction, similar to nested loops.\nl
        Note that $|\alpha| > 0, |x| > 0$ because we are in the inductive case.
        \begin{enumerate}
            \item Case 1: $\alpha = a \alpha'$ for some $a \in \Sigma$.\nl
                In this case, $x = a x'$ for some $x' \in \Sigma^*$, and $\alpha' \derives x'$ in $n$ productions.\nl
                Using a match transition, we have $(q, x, \alpha \bot) = (q, ax', a\alpha'\bot) \changesto (q, x', \alpha'\bot)$. By inductive hypothesis 2, $(q, x', \alpha'\bot) \changesto^* (q,
                \epsilon, \bot)$, and hence, $(q, x, \alpha\bot) \changesto^* (q, \epsilon, \bot)$.
            \item Case 2: $\alpha = A\alpha'$ for some $A \in N$.\nl
                So $\exists \gamma \in (N \cup \Sigma)^*$ such that $(A \to \gamma) \in R$ and $\gamma \alpha' \derives x$ in $n - 1$ productions.\nl
                Using a produce transition, we get $(q, x, \alpha\bot) = (q, x, A\alpha'\bot) \changesto (q, x, \gamma\alpha'\bot)$. By inductive hypothesis 1, $(q, x, \gamma\alpha'\bot)
                \changesto^* (q, \epsilon, \bot)$. Hence, $(q, x, \alpha\bot) \changesto^* (q, \epsilon, \bot)$.
        \end{enumerate}
        Now for the base case for inductive hypothesis 1, $\alpha \derives x$ in $0$ steps, so $\alpha = x$. If $\alpha = x = \epsilon$ then it's trivial. Else, go to case 1.\nl
        For the base case for inductive hypothesis 2, if $\alpha \derives x$ in $n$ steps and $x = \epsilon$, either $\alpha = \epsilon$ or it isn't. If it is, then it's trivial, else $\alpha$ is a string of non-terminals
        only, so go to case 2.
    \end{proof}

\end{proof}

\begin{cor}
    Every language which is generated by a grammar is also recognized by a PDA.
\end{cor}

\end{document}
