\documentclass[a4paper]{article}
\usepackage[english]{babel}
\usepackage[a4paper,top=2cm,bottom=2cm,left=2cm,right=2cm,marginparwidth=1.75cm]{geometry}
\usepackage{amsmath}
\usepackage{amsfonts}
% \usepackage{amsthm}
\usepackage{amssymb}
\usepackage{graphicx}
\usepackage[colorinlistoftodos]{todonotes}
\usepackage[colorlinks=true, allcolors=blue]{hyperref}
\usepackage{import}
\usepackage{pdfpages}
\usepackage{transparent}
\usepackage{xcolor}
\usepackage{algorithmicx}
\usepackage{algpseudocode}

\usepackage{thmtools}
\usepackage{enumitem}
\usepackage[framemethod=TikZ]{mdframed}

\usepackage{xpatch}

\usepackage{boites}
\makeatletter
\xpatchcmd{\endmdframed}
{\aftergroup\endmdf@trivlist\color@endgroup}
{\endmdf@trivlist\color@endgroup\@doendpe}
{}{}
\makeatother

%\usepackage[poster]{tcolorbox}
%\allowdisplaybreaks
%\sloppy

\usepackage[many]{tcolorbox}

\xpatchcmd{\proof}{\itshape}{\bfseries\itshape}{}{}

% to set box separation
\setlength{\fboxsep}{0.8em}
\def\breakboxskip{7pt}
\def\breakboxparindent{0em}

\newenvironment{proof}{\begin{breakbox}\textit{Proof.}}{\hfill$\square$\end{breakbox}}
\newenvironment{ans}{\begin{breakbox}\textit{Answer.}}{\end{breakbox}}
\newenvironment{soln}{\begin{breakbox}\textit{Solution.}}{\end{breakbox}}

% \tcolorboxenvironment{proof}{
%     blanker,
%     before skip=\topsep,
%     after skip=\topsep,
%     borderline={0.4pt}{0.4pt}{black},
%     breakable,
%     left=12pt,
%     right=12pt,
%     top=12pt,
%     bottom=12pt,
% }
% 
% \tcolorboxenvironment{ans}{
%     blanker,
%     before skip=\topsep,
%     after skip=\topsep,
%     borderline={0.4pt}{0.4pt}{black},
%     breakable,
%     left=12pt,
%     right=12pt,
% }

\mdfdefinestyle{enclosed}{
    linecolor=black
    ,backgroundcolor=none
    ,apptotikzsetting={\tikzset{mdfbackground/.append style={fill=gray!100,fill opacity=.3}}}
    ,frametitlefont=\sffamily\bfseries\color{black}
    ,splittopskip=.5cm
    ,frametitlebelowskip=.0cm
    ,topline=true
    ,bottomline=true
    ,rightline=true
    ,leftline=true
    ,leftmargin=0.01cm
    ,linewidth=0.02cm
    ,skipabove=0.01cm
    ,innerbottommargin=0.1cm
    ,skipbelow=0.1cm
}

\mdfsetup{%
    middlelinecolor=black,
    middlelinewidth=1pt,
roundcorner=4pt}

\setlength{\parindent}{0pt}

\mdtheorem[style=enclosed]{theorem}{Theorem}
\mdtheorem[style=enclosed]{lemma}{Lemma}[theorem]
\mdtheorem[style=enclosed]{claim}{Claim}[theorem]
\mdtheorem[style=enclosed]{ques}{Question}
\mdtheorem[style=enclosed]{defn}{Definition}
\mdtheorem[style=enclosed]{notn}{Notation}
\mdtheorem[style=enclosed]{obs}{Observation}
\mdtheorem[style=enclosed]{eg}{Example}
\mdtheorem[style=enclosed]{cor}{Corollary}
\mdtheorem[style=enclosed]{note}{Note}

% \let\thetheorem=\relax
% \let\thelemma=\relax
% \let\theclaim=\relax
% \let\theques=\relax
% \let\thedefn=\relax
% \let\thenotn=\relax
% \let\theobs=\relax
% \let\thecor=\relax
% \let\thenote=\relax

% \renewcommand\qedsymbol{$\blacksquare$}
\newcommand{\nl}{\vspace{0.2cm}\\}
\newcommand{\mc}{\mathcal}
\newcommand{\mi}{\mathit}
\newcommand{\mf}{\mathbf}
\newcommand{\mb}{\mathbb}
\renewcommand{\L}{\mc{L}}
\newcommand{\hd}{\hat{\delta}}

\newcommand{\incfig}[1]{%
    \def\svgwidth{\columnwidth}
    \import{./figures/}{#1.pdf_tex}
}
\pdfsuppresswarningpagegroup=1

\title{\textbf{COL352 Lecture 10}}
\date{}

\begin{document}
\maketitle
\tableofcontents

\section{Recap}

Discussion in previous class.

\section{Definitions}

\section{Content}

\begin{theorem}
    \textbf{Myhill-Nerode Theorem (Part 1)}\nl
    If $L$ is regular, then $=_L$ has finitely many equivalence classes.
\end{theorem}

\begin{proof}
    $L$ is regular, so there exists DFA $D$ that recognizes $L$. So the number of states of $D$ is at least the number of equivalence classes of $D$, which is at least the number of
    equivalence classes of $=_L$. Hence, this claim follows since there are finitely many states in $D$.
\end{proof}

\begin{theorem}
    \textbf{Myhill-Nerode Theorem (Part 2)}\nl
    Suppose $L \subseteq \Sigma^*$ is a language such that $=_L$ has finitely many equivalence classes, say $k$. Then $L$ is recognized by a DFA $D(L)$ with $k$ states. 
    (Note that this would imply that $D(L)$ would be a minimum-state DFA for $L$).
\end{theorem}

\begin{proof}
    Consider $Q = \{w \in \Sigma^* \mid w \text{ is the lexicographically smallest string in its equivalence class of }=_L\}$.\nl
    $=_L$ has finitely many equivalence classes, so $Q$ is finite, and $|Q| = k$.\nl
    Define $rep : \Sigma^* \to Q$ as $rep(x) = $ the lexicographically smallest string in the equivalence class of $x$ under $=_L$.\nl
    Initial state of $D(L) = \epsilon$ (since $\epsilon \in Q$, as it is the lexicographically smallest string in its equivalence class).\nl
    Accepting states of $D(L)$, $A = \{w \in Q \mid w \in L\}$.\nl
    Transition function of $D(L) : \forall w \in Q, a \in \Sigma : \delta(w, a) = rep(wa)$.\nl

    \begin{claim}
        $\hd(\epsilon, x) = rep(x)$
    \end{claim}

    \begin{proof}
        By induction on $|x|$. If $|x| = 0$, then $x = \epsilon$, so since $\hd(\epsilon, \epsilon) = \epsilon$, the base case is done.
        Now suppose $x = x'a$ for $x' \in \Sigma^*, a \in \Sigma$. Then we have $\hd(\epsilon, x) = \delta(\hd(\epsilon, x'), a) = \delta(rep(x'), a) = rep(rep(x')a) = rep(x'a) = rep(x)$. The
        second last equality follows from $=_L$, since by definition, $x' =_L rep(x') \implies x'a =_L rep(x')a \implies x =_L rep(x')a \implies rep(x) = rep(rep(x')a)$.
        First equality from definition of $\hd$, second from induction hypothesis and the third from the definition of $\delta$.
    \end{proof}

    \begin{claim}
        $D(L)$ recognizes $L$.
    \end{claim}

    \begin{proof}
        Let $x$ be an arbitrary string in $\Sigma^*$. Then we have
        \begin{align*}
            x \in L &\iff rep(x) \in L\\
                    &\iff rep(x) \in A\\
                    &\iff \hd(\epsilon, x) \in A\\
                    &\iff x \text{ is accepted by } D(L)
        \end{align*}
        The first follows using $\epsilon$ in the definition of $=_L$, the second from the definition of $A$ and of $rep$, the third from the previous claim, and the last from the
        definition of a DFA.
    \end{proof}

    Observe that this proves Myhill-Nerode theorem part 2.

\end{proof}

\begin{ques}
    How does $\sim_{D(L)}$ look like?
\end{ques}

\begin{ans}
    The number of equivalence classes of $\sim_{D(L)}$ is equal to the number of equivalence classes of $=_L$. Since $D(L)$ recognizes $L$, $\sim_{D(L)}$ refines $=_L$.
\end{ans}

\begin{ques}
    How do we show that all DFAs with $k$ states that recognize $L$ are isomorphic?
\end{ques}

\begin{ans}
    Suppose DFA $D$ recognizes $L$ and has exactly as many states as $D(L)$. So $\sim_D$ is the same as $\sim_{D(L)}$, which is the same as $=_L$.
    Each equivalence class of $\sim_D$ represents a state of $D$, and each equivalence class of $\sim_{D(L)}$ represents a state of $D(L)$, and this gives the required isomorphism. Fill in the
    details -- exercise.
\end{ans}

\begin{ques}
    Given a DFA $D = (Q, \Sigma, \delta, q_0, A)$, recognizing some language $L \subseteq \Sigma^*$, construct $D(L)$.
\end{ques}

\begin{ans}
    Idea: We know the thing about how $\sim_D$ refines $=_L$.\nl
    For $q \in Q$, define $C_q = \{x \in \Sigma^* \mid \hd(q_0, x) = q\}$. If it isn't the null set, it is an equivalence class of $\sim_D$.\nl
    We do the following steps.
    \begin{enumerate}
        \item Remove unreachable states, i.e., $q$ such that $C_q = \emptyset$.
        \item Idea: Figure out, for each $q, q'$ in $Q$, whether $C_q, C_q'$ are contained in the same equialence class of $=_L$ and if so, merge them somehow.\nl
            Define relation $\equiv$ on $Q$ as $q \equiv q'$ if $C_q, C_q'$ are contained in the same equivalence class of $=_L$. Then we would merge $q, q'$ if $q \equiv q'$.
    \end{enumerate}
\end{ans}

\end{document}
