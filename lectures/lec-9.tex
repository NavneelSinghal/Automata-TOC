\documentclass[a4paper]{article}
\usepackage[english]{babel}
\usepackage[a4paper,top=2cm,bottom=2cm,left=2cm,right=2cm,marginparwidth=1.75cm]{geometry}
\usepackage{amsmath}
\usepackage{amsfonts}
% \usepackage{amsthm}
\usepackage{amssymb}
\usepackage{graphicx}
\usepackage[colorinlistoftodos]{todonotes}
\usepackage[colorlinks=true, allcolors=blue]{hyperref}
\usepackage{import}
\usepackage{pdfpages}
\usepackage{transparent}
\usepackage{xcolor}
\usepackage{algorithmicx}
\usepackage{algpseudocode}

\usepackage{thmtools}
\usepackage{enumitem}
\usepackage[framemethod=TikZ]{mdframed}

\usepackage{xpatch}

\usepackage{boites}
\makeatletter
\xpatchcmd{\endmdframed}
{\aftergroup\endmdf@trivlist\color@endgroup}
{\endmdf@trivlist\color@endgroup\@doendpe}
{}{}
\makeatother

%\usepackage[poster]{tcolorbox}
%\allowdisplaybreaks
%\sloppy

\usepackage[many]{tcolorbox}

\xpatchcmd{\proof}{\itshape}{\bfseries\itshape}{}{}

% to set box separation
\setlength{\fboxsep}{0.8em}
\def\breakboxskip{7pt}
\def\breakboxparindent{0em}

\newenvironment{proof}{\begin{breakbox}\textit{Proof.}}{\hfill$\square$\end{breakbox}}
\newenvironment{ans}{\begin{breakbox}\textit{Answer.}}{\end{breakbox}}
\newenvironment{soln}{\begin{breakbox}\textit{Solution.}}{\end{breakbox}}

% \tcolorboxenvironment{proof}{
%     blanker,
%     before skip=\topsep,
%     after skip=\topsep,
%     borderline={0.4pt}{0.4pt}{black},
%     breakable,
%     left=12pt,
%     right=12pt,
%     top=12pt,
%     bottom=12pt,
% }
% 
% \tcolorboxenvironment{ans}{
%     blanker,
%     before skip=\topsep,
%     after skip=\topsep,
%     borderline={0.4pt}{0.4pt}{black},
%     breakable,
%     left=12pt,
%     right=12pt,
% }

\mdfdefinestyle{enclosed}{
    linecolor=black
    ,backgroundcolor=none
    ,apptotikzsetting={\tikzset{mdfbackground/.append style={fill=gray!100,fill opacity=.3}}}
    ,frametitlefont=\sffamily\bfseries\color{black}
    ,splittopskip=.5cm
    ,frametitlebelowskip=.0cm
    ,topline=true
    ,bottomline=true
    ,rightline=true
    ,leftline=true
    ,leftmargin=0.01cm
    ,linewidth=0.02cm
    ,skipabove=0.01cm
    ,innerbottommargin=0.1cm
    ,skipbelow=0.1cm
}

\mdfsetup{%
    middlelinecolor=black,
    middlelinewidth=1pt,
roundcorner=4pt}

\setlength{\parindent}{0pt}

\mdtheorem[style=enclosed]{theorem}{Theorem}
\mdtheorem[style=enclosed]{lemma}{Lemma}[theorem]
\mdtheorem[style=enclosed]{claim}{Claim}[theorem]
\mdtheorem[style=enclosed]{ques}{Question}
\mdtheorem[style=enclosed]{defn}{Definition}
\mdtheorem[style=enclosed]{notn}{Notation}
\mdtheorem[style=enclosed]{obs}{Observation}
\mdtheorem[style=enclosed]{eg}{Example}
\mdtheorem[style=enclosed]{cor}{Corollary}
\mdtheorem[style=enclosed]{note}{Note}

% \let\thetheorem=\relax
% \let\thelemma=\relax
% \let\theclaim=\relax
% \let\theques=\relax
% \let\thedefn=\relax
% \let\thenotn=\relax
% \let\theobs=\relax
% \let\thecor=\relax
% \let\thenote=\relax

% \renewcommand\qedsymbol{$\blacksquare$}
\newcommand{\nl}{\vspace{0.2cm}\\}
\newcommand{\mc}{\mathcal}
\newcommand{\mi}{\mathit}
\newcommand{\mf}{\mathbf}
\newcommand{\mb}{\mathbb}
\renewcommand{\L}{\mc{L}}
\newcommand{\hd}{\hat{\delta}}

\newcommand{\incfig}[1]{%
    \def\svgwidth{\columnwidth}
    \import{./figures/}{#1.pdf_tex}
}
\pdfsuppresswarningpagegroup=1

\title{\textbf{COL352 Lecture 9}}
\date{}

\begin{document}
\maketitle
\tableofcontents

\section{Recap}
Last class stuff about the summary. Also look at the second reference for a proof via GNFAs.
\section{Definitions}
Same as last class.
\section{Content}
We shall do the proof slightly differently.\nl
\begin{claim}
    $L_{ijk} = L_{ij(k-1)} \cup L'$, where $L' = \left(L_{ik(k-1)}\cup (L_{kk(k-1)}^*) L_{kj(k-1)}\right)$
\end{claim}
\begin{proof}
    Two parts.\nl
    \begin{enumerate}
        \item $L_{ij(k-1)} \cup L' \subseteq L_{ijk}$:
            \begin{proof}
                \begin{enumerate}
                    \item $x \in L_{ij(k-1)} \subseteq L_{ijk} \implies x \in L_{ijk}$
                    \item $x \in L' \implies x = x_0 x_1 \cdots x_{N-1} x_N$ where $x_0 \in L_{ik(k-1)}$. $\forall l = 1 \ldots N - 1, x_l \in L_{kk(k-1)}$. Now check that $x \in L$.
                \end{enumerate}
            \end{proof}
        \item Suppose $x \in L_{ijk}$. If the run of $D$ on $x$ starting from $i$ never visits $k$ then $x \in L_{ij(k-1)}$. If the run visits $k$ $N$ times, then $x = x_0 \ldots x_{N-1} x_N$,
            where $\forall l = 0 \ldots N - 1, \hd(i, x_0, \ldots, x_l) = k$. Then $x_0 \in L_{ik(k-1)}$, and $\forall l = 1 \ldots N - 1, x_l \in L_{kk(k-1)}$, and $x_N \in L_{kj(k-1)}$. Hence $x
            \in L'$.
    \end{enumerate}
\end{proof}

\begin{claim}
    The set of all regular languages is countably infinite.
\end{claim}
\begin{note}
    Proof using regular expressions - easy to see using lexicographical ordering. We shall prove using definition of DFA.
\end{note}
\begin{proof}
    Wlog, let $\Sigma = \{0, \ldots, m - 1\}$, and $Q = \{1, \ldots, n\}$.
    We can encode a DFA with a finite length binary string, as follows:
    $n$ ones, one zero, $|\Sigma|$ ones, one zero, $\lceil\log n\rceil$ bits denoting initial state, then $n m \lceil \log n \rceil$ bits for $\delta$ as a table ($Q \times \Sigma$ table, with
    each entry with size $\lceil \log n \rceil$), and the set of accepting states in $n$
    bits.\nl
    Now consider the first DFA in this total ordering that accepts a language $L$.
\end{proof}

\begin{cor}
    The set of regular languages over $\Sigma$ is uncountable.
\end{cor}

\begin{eg}
    $\{\langle P, x \rangle \mid P \text{ halts on } x\}$: not regular else we'll get a contradiction to the halting problem.
\end{eg}

\begin{eg}
    $\{x \in \{0,1\}^* \mid \# 0 = \# 1 \text{ in } x\}$
\end{eg}

\begin{eg}
    Set of all palindromes over $\Sigma$ with at least $2$ characters.
\end{eg}

\begin{eg}
    $\{0^n 1^n \mid n \in \mb{N} \cup \{0\}\}$
\end{eg}

Now we show that the last example is indeed not a regular language.

\begin{proof}
    Suppose $L = \{0^n 1^n \mid n \in \mb{N} \cup \{0\}\}$ is regular. Let $D = (Q, \{0, 1\}, \delta, q_0, A)$ be a DFA recognizing $L$. Let $n = |Q|$.\nl
    We know $0^n 1^n \in L$. Look at the run of $D$ on $0^n 1^n$ -- say $q_0, q_1, \ldots, q_n, q_{n+1}, \ldots, q_{2n}$.\nl
    Some two states of $q_0, \ldots, q_n$ are equal. Suppose $q_i = q_j, 0 \le i < j \le n$.\nl
    \begin{claim}
        $D$ also accepts $0^{n+k(j-i)} 1^n$ for $k \ge -1$.
    \end{claim}
    \begin{proof}
        We can construct this inductively.
    \end{proof}
    Since this constructed string doesn't belong to $L$ for $k \ne 0$.
\end{proof}

\end{document}
