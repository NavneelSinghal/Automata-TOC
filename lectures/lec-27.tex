\documentclass[a4paper]{article}
\usepackage[english]{babel}
\usepackage[a4paper,top=2cm,bottom=2cm,left=2cm,right=2cm,marginparwidth=1.75cm]{geometry}
\usepackage{amsmath}
\usepackage{amsfonts}
% \usepackage{amsthm}
\usepackage{amssymb}
\usepackage{graphicx}
\usepackage[colorinlistoftodos]{todonotes}
\usepackage[colorlinks=true, allcolors=blue]{hyperref}
\usepackage{import}
\usepackage{pdfpages}
\usepackage{transparent}
\usepackage{xcolor}
\usepackage{algorithmicx}
\usepackage{algpseudocode}

\usepackage{thmtools}
\usepackage{enumitem}
\usepackage[framemethod=TikZ]{mdframed}

\usepackage{xpatch}

\usepackage{boites}
\makeatletter
\xpatchcmd{\endmdframed}
{\aftergroup\endmdf@trivlist\color@endgroup}
{\endmdf@trivlist\color@endgroup\@doendpe}
{}{}
\makeatother

%\usepackage[poster]{tcolorbox}
%\allowdisplaybreaks
%\sloppy

\usepackage[many]{tcolorbox}

\xpatchcmd{\proof}{\itshape}{\bfseries\itshape}{}{}

% to set box separation
\setlength{\fboxsep}{0.8em}
\def\breakboxskip{7pt}
\def\breakboxparindent{0em}

\newenvironment{proof}{\begin{breakbox}\textit{Proof.}}{\hfill$\square$\end{breakbox}}
\newenvironment{ans}{\begin{breakbox}\textit{Answer.}}{\end{breakbox}}
\newenvironment{soln}{\begin{breakbox}\textit{Solution.}}{\end{breakbox}}

% \tcolorboxenvironment{proof}{
%     blanker,
%     before skip=\topsep,
%     after skip=\topsep,
%     borderline={0.4pt}{0.4pt}{black},
%     breakable,
%     left=12pt,
%     right=12pt,
%     top=12pt,
%     bottom=12pt,
% }
%
% \tcolorboxenvironment{ans}{
%     blanker,
%     before skip=\topsep,
%     after skip=\topsep,
%     borderline={0.4pt}{0.4pt}{black},
%     breakable,
%     left=12pt,
%     right=12pt,
% }

\mdfdefinestyle{enclosed}{
    linecolor=black
    ,backgroundcolor=none
    ,apptotikzsetting={\tikzset{mdfbackground/.append style={fill=gray!100,fill opacity=.3}}}
    ,frametitlefont=\sffamily\bfseries\color{black}
    ,splittopskip=.5cm
    ,frametitlebelowskip=.0cm
    ,topline=true
    ,bottomline=true
    ,rightline=true
    ,leftline=true
    ,leftmargin=0.01cm
    ,linewidth=0.02cm
    ,skipabove=0.01cm
    ,innerbottommargin=0.1cm
    ,skipbelow=0.1cm
}

\mdfsetup{%
    middlelinecolor=black,
    middlelinewidth=1pt,
roundcorner=4pt}

\setlength{\parindent}{0pt}

\mdtheorem[style=enclosed]{theorem}{Theorem}
\mdtheorem[style=enclosed]{lemma}{Lemma}[theorem]
\mdtheorem[style=enclosed]{claim}{Claim}[theorem]
\mdtheorem[style=enclosed]{ques}{Question}
\mdtheorem[style=enclosed]{defn}{Definition}
\mdtheorem[style=enclosed]{notn}{Notation}
\mdtheorem[style=enclosed]{obs}{Observation}
\mdtheorem[style=enclosed]{eg}{Example}
\mdtheorem[style=enclosed]{cor}{Corollary}
\mdtheorem[style=enclosed]{note}{Note}

% \let\thetheorem=\relax
% \let\thelemma=\relax
% \let\theclaim=\relax
% \let\theques=\relax
% \let\thedefn=\relax
% \let\thenotn=\relax
% \let\theobs=\relax
% \let\thecor=\relax
% \let\thenote=\relax

% \renewcommand\qedsymbol{$\blacksquare$}
\newcommand{\nl}{\vspace{0.2cm}\\}
\newcommand{\mc}{\mathcal}
\newcommand{\mi}{\mathit}
\newcommand{\mf}{\mathbf}
\newcommand{\mb}{\mathbb}
\renewcommand{\L}{\mc{L}}
\newcommand{\hd}{\hat{\delta}}
\newcommand{\produces}{\implies}
\newcommand{\derives}{\stackrel{*}{\implies}}
\newcommand{\changesto}{\vdash}
\newcommand\Vtextvisiblespace[1][.3em]{%
    \mbox{\kern.06em\vrule height.3ex}%
    \vbox{\hrule width#1}%
    \hbox{\vrule height.3ex}
}
\newcommand{\blank}{{\Vtextvisiblespace[0.7em]}}
\newcommand{\leftend}{\triangleright}

\newcommand{\incfig}[1]{%
    \def\svgwidth{\columnwidth}
    \import{./figures/}{#1.pdf_tex}
}
\pdfsuppresswarningpagegroup=1

\title{\textbf{COL352 Lecture 27}}
\date{}

\begin{document}
\maketitle
\tableofcontents

\section{Recap}

Started closure properties last time.

\section{Definitions}

\begin{defn}
    A language $L$ is said to be co-Turing recognizable if its complement is Turing recognizable.
\end{defn}

\section{Content}

\section{Closure properties}
\paragraph{Decidable languages} The class of decidable languages is closed under $\cap, \cup, \cdot, *$ and complementation.
        \begin{enumerate}
            \item $\cap$ -- copy $x$ to tape $2$, simulate $M_1$ on tape $1$, $M_2$ on tape $2$, and if both accept $x$, accept $x$ else reject $x$.
            \item $\cup$ -- similar.
            \item $\cdot$ -- construct 2-tape NTM for $L_1 \cdot L_2$: Non-deterministically cut a suffix of $x$ from tape 1, and paste on tape $2$, and then simulate both.
            \item $*$ -- while $x \ne \epsilon$: non-deterministically cut a nonempty prefix of $x$ from tape 1, and paste on tape 2, simulate $M$ on tape 2, and if $M$ rejects its input, then
                reject $x$. After the while loop is over, accept $x$.
            \item complementation -- swap $q_{acc}, q_{rej}$ in the DTM (not the NTM).
        \end{enumerate}
\paragraph{Turing-recognizable languages} The class of Turing recognizable languages is closed under $\cap, \cup, \cdot, *$ -- same constructions in case of decidable languages, just take care
of the possibility of non-terminating runs in the proof (just stay positive and think the iff version instead of arguing about the strings not in the language).

\begin{ques}
    Is the class of Turing recognizable languages closed under complementation?
\end{ques}

\begin{ques}
    Is there a language which is recognizable but not decidable?
\end{ques}

\begin{ques}
    Give an explicit unrecognizable language.
\end{ques}

\begin{note}
    co-Turing recognizable is not the same as unrecognizable.\nl
    For instance, if $L$ is regular, then $L^c$ is also regular, so both $L, L^c$ are recognizable as well as co-recognizable.
\end{note}
 \begin{note}
     $L$ is co-recognizable is equivalent to saying that there exists a Turing machine $M$ such that $x \not\in L$ iff $M$ on $x$ halts and rejects -- for a construction, swap accepting and
     rejecting states in the TM of $L^c$.
 \end{note}

 \begin{obs}
     If $L$ is decidable then $L$ is recognizable.\nl
     If $L$ is decidable then $L$ is co-recognizable (since $L$ decidable $\implies$ $L^c$ decidable $\implies$ $L^c$ recognizable).\nl
\end{obs}

\begin{theorem}
    A language $L$ is decidable iff $L$ is recognizable as well as co-recognizable.
\end{theorem}
\begin{proof}
    The forward direction is as in the previous observation.\nl
    For the backward diretion, suppose $L$ is recognizable as well as co-recognizable. Let $M$ be a DTM which recognizes $L$ and $M'$ be a DTM which recognizes $L^c$. Consider the
    2-tape DTM $D$ which on input $x$ does the following:
    \begin{enumerate}
        \item Copy $x$ from tape 1 to tape 2.
        \item Simulate $M$ on tape 1 and $M'$ on tape 2 in parallel.
        \item If $M$ halts earlier, abort $M'$ and follow $M$'s decision. Else if $M'$ halts earlier, abort $M$ and invert $M'$'s decision. The other case never arises, since either $x$ is in $L$
            or in $L^c$, so either $M$ accepts $x$ or $M'$ accepts $x$.
    \end{enumerate}
    Note that if $D$ accepts $x$, then $x \in L$ and if $D$ rejects $x$, then $x \in L$.\nl
    By construction, $D$ is a decider for $L$, so $L$ is decidable.
\end{proof}

Goal: Construct an unrecognizable language.\nl

Recall the proof of the fact that $2^{\mb{N}}$ is uncountable from COL202.\nl
Flipping the incidence matrix diagonal is equivalent to considering $\{i \mid i \not\in S_i\}$ and this set is different from each $S_j$.\nl

Recall that the set of TMs over any fixed finite alphabet $\Sigma$ is countable (using binary encoding).\nl
Let $M_w$ be the DTM whose description is $w$ where $w \in \Sigma$, and if $w$ is not the description of any DTM, then we define $M_w$ to be the DTM where $q_0 = q_{rej}$, $\Gamma = \Sigma \cup
\{\blank\}$, $Q = \{q_{acc}, q_{rej}\}$.\nl

Now consider the set of strings $w_i$ in $\Sigma^*$, and for each of those, consider $M_{w_i}$ in $\Sigma^*$. Let $A$ be a matrix where $a_{ij} = 1$ iff $M_{w_i}$ accepts $w_j$, and $0$
otherwise.\nl

Define $\mf{DIAG} = \{w \in \Sigma^* \mid M_w \text{ does not accept } w\}$.

\begin{theorem}
    $\mf{DIAG}$ is unrecognizable.
\end{theorem}

\begin{proof}
    Suppose not. Let $R$ be a DTM that recognizes $\mf{DIAG}$. Then $R = M_w$ for some $w \in \Sigma^*$. Does $R$ accept $w$?\nl
    If $M_w$ accepts $w$, then by the definition of $\mf{DIAG}$, $w \not\in \mf{DIAG}$. But $M_w$ recognizes $\mf{DIAG}$, so by assumption since $M_w$ accepts $w$, we have $w \in
    \mf{DIAG}$, which is a contradiction.\nl
    If $M_w$ doesn't accept $w$, it follows that $w \not\in \mf{DIAG}$ from the fact that $M_w$ recognizes $\mf{DIAG}$. But from the definition of $\mf{DIAG}$, since $M_w$ doesn't accept $w$, $w \in
    \mf{DIAG}$, which is a contradiction.
\end{proof}

\end{document}
