\documentclass[10pt,addpoints]{exam}
\usepackage[english]{babel}
\usepackage[a4paper,top=2cm,bottom=2cm,left=2cm,right=2cm,marginparwidth=1.75cm]{geometry}
\usepackage{boites}
\usepackage{amsmath}
\usepackage{amsfonts}
% \usepackage{amsthm}
\usepackage{amssymb}
\usepackage{graphicx}
\usepackage[colorinlistoftodos]{todonotes}
\usepackage[colorlinks=true, allcolors=blue]{hyperref}
\usepackage{import}
\usepackage{pdfpages}
\usepackage{transparent}
\usepackage{xcolor}
\usepackage{algorithmicx}
\usepackage{algpseudocode}

\usepackage{thmtools}
\usepackage{enumitem}
\usepackage[framemethod=TikZ]{mdframed}

\usepackage{xpatch}

\makeatletter
\xpatchcmd{\endmdframed}
{\aftergroup\endmdf@trivlist\color@endgroup}
{\endmdf@trivlist\color@endgroup\@doendpe}
{}{}
\makeatother

%\usepackage[poster]{tcolorbox}
%\allowdisplaybreaks
%\sloppy

\usepackage[many]{tcolorbox}

%\xpatchcmd{\proof}{\itshape}{\bfseries\itshape}{}{}

\setlength{\parindent}{0pt}

\setlength{\fboxsep}{1em}
\def\breakboxskip{7pt}
\def\breakboxparindent{0em}

\newenvironment{proof}{\begin{breakbox}\textit{Proof.}}{\hfill$\square$\end{breakbox}}
\newenvironment{ans}{\begin{breakbox}\textit{Answer.}}{\end{breakbox}}
\newenvironment{soln}{\begin{breakbox}\textit{Solution.}}{\end{breakbox}}

% \tcolorboxenvironment{proof}{
%     blanker,
%     before skip=\topsep,
%     after skip=\topsep,
%     borderline={0.4pt}{0.4pt}{black},
%     enforce breakable,
%     left=12pt,
%     right=12pt,
%     top=12pt,
%     bottom=12pt,
% }
% 
% \tcolorboxenvironment{ans}{
%     blanker,
%     before skip=\topsep,
%     after skip=\topsep,
%     borderline={0.4pt}{0.4pt}{black},
%     enforce breakable,
%     left=12pt,
%     right=12pt,
%     top=12pt,
%     bottom=12pt,
% }

\mdfdefinestyle{enclosed}{
    linecolor=black
    ,backgroundcolor=none
    ,apptotikzsetting={\tikzset{mdfbackground/.append style={fill=gray!100,fill opacity=.3}}}
    ,frametitlefont=\sffamily\bfseries\color{black}
    ,splittopskip=.5cm
    ,frametitlebelowskip=.0cm
    ,topline=true
    ,bottomline=true
    ,rightline=true
    ,leftline=true
    ,leftmargin=0.01cm
    ,linewidth=0.02cm
    ,skipabove=0.01cm
    ,innerbottommargin=0.1cm
    ,skipbelow=0.1cm
}

\mdfdefinestyle{proofstyle}{
    linecolor=black
    ,backgroundcolor=none
    ,apptotikzsetting={\tikzset{mdfbackground/.append style={fill=gray!100,fill opacity=.3}}}
    ,frametitlefont=\sffamily\bfseries\color{black}
    ,splittopskip=.5cm
    ,frametitlebelowskip=.0cm
    ,topline=true
    ,bottomline=true
    ,rightline=true
    ,leftline=true
    ,leftmargin=0.01cm
    ,linewidth=0.02cm
    ,skipabove=0.01cm
    ,innerbottommargin=0.1cm
    ,skipbelow=0.1cm
    ,ntheorem=false
}

\mdfsetup{%
    middlelinecolor=black,
    middlelinewidth=1pt,
roundcorner=4pt}


\mdtheorem[style=enclosed]{theorem}{Theorem}
\mdtheorem[style=enclosed]{ques}{Question}
\mdtheorem[style=enclosed]{prob}{Problem}
\mdtheorem[style=enclosed]{lemma}{Lemma}[prob]
\mdtheorem[style=enclosed]{claim}{Claim}[prob]
\mdtheorem[style=enclosed]{defn}{Definition}
\mdtheorem[style=enclosed]{notn}{Notation}
\mdtheorem[style=enclosed]{obs}{Observation}
\mdtheorem[style=enclosed]{eg}{Example}
\mdtheorem[style=enclosed]{cor}{Corollary}
\mdtheorem[style=enclosed]{note}{Note}
%\mdtheorem[style=proofstyle]{proof}{Proof.}
% \mdtheorem[style=proofstyle]{ans}{Answer.}

% \let\theproof=\false
% \let\theans=\false
% \let\thetheorem=\relax
% \let\thelemma=\relax
% \let\theclaim=\relax
% \let\theques=\relax
% \let\thedefn=\relax
% \let\thenotn=\relax
% \let\theobs=\relax
% \let\thecor=\relax
\let\thenote=\relax

%\renewcommand\qedsymbol{$\blacksquare$}
\newcommand{\nl}{\vspace{0.2cm}\\}
\newcommand{\mc}{\mathcal}
\newcommand{\mi}{\mathit}
\newcommand{\mf}{\mathbf}
\newcommand{\mb}{\mathbb}
\renewcommand{\L}{\mc{L}}
\newcommand{\hd}{\hat{\delta}}
\DeclareMathOperator{\takelast}{last}
\DeclareMathOperator{\xor}{xor}
\newcommand{\produces}{\implies}
\newcommand{\derives}{\stackrel{*}{\implies}}
\newcommand{\changesto}{\vdash}

\newcommand{\incfig}[1]{%
    \def\svgwidth{\columnwidth}
    \import{./figures/}{#1.pdf_tex}
}
\pdfsuppresswarningpagegroup=1

\usepackage{tfrupee}

\pagestyle{head}

\firstpageheader{COL352}{}{Homework 2}
\firstpageheadrule

\title{COL352 HW-3}
\author{
Sarthak Agrawal \\ 2018CS10383
\and
Piyush Gupta \\ 2018CS10365
\and
Navneel Singhal \\ 2018CS10360
}

\begin{document}
\maketitle

%\noindent\href{https://moodle.iitd.ac.in/mod/forum/discuss.php?d=19210#p26977}{\textbf{Read the instructions carefully.}}
%
%\vspace{0.5cm}
%

\begin{prob}
Consider the grammar $G=(\{S\},\{a,b\},R,S)$, where the set of production rules is
\[R=\{S\longrightarrow SS\text{, }S\longrightarrow aaSb\text{, }S\longrightarrow bSaa\text{, }S\longrightarrow aSbSa\text{, }S\longrightarrow\varepsilon\}\text{.}\]
Obviously, the language $L$ generated by this grammar contains only those strings in which the number of $a$'s is twice the number of $b$'s. (If you are unable to prove the last statement rigorously, you must contact the instructor asap.) Prove that, in fact, $L$ contains all strings in which  the number of $a$'s is twice the number of $b$'s.
\end{prob}
\begin{soln}
To prove that the language $\comp{E_{TM}}$ is \emph{recognizable} we will show how to construct a non-deterministic turning machine that essentially does the following.
\begin{enumerate}
    \item Non-deterministically choose a string from $\Sigma^*$ (possibly empty).
    \item Simulate the turning machine $M_w$ on the chosen string, and return its output.
\end{enumerate}
First, suppose that such a NTM exists, then we will show that this proves that $\comp{E_{TM}}$ is recognizable.
To see this, consider any turning machine $M_w$. If this machine accepts some string $x \in \Sigma^*$,
then the \emph{run} of our NTM in which it chooses the string $x$ will be an accepting run, therefore this NTM accepts such $M_w$.
Now, suppose $M_w$ does not accept any string. Then, since the NTM ultimately simulates $M_w$ on a string, there can not exist any accepting run of of this NTM on $M_w$.
Hence, we can conclude that existence of such a NTM would imply recognisability of $\comp{E_{TM}}$.\\

Now, we shall give an \emph{implementation description} of a 2-Tape NTM that realizes the above given \emph{vague description}. Unless specified otherwise, it is assumed that in a transition tape heads do not move (i.e. a \emph{stay}).
\begin{enumerate}
    \item Initially, the first tape would contain the input string $w$, and the second tape would be empty.
    \item \textbf{Generate Phase: } Transition into a state $q_{decide}$. Whenever the turning machine is in the state $q_{decide}$ it will non-deterministically select one of the following 2 \emph{epsilon} transitions.
    \begin{enumerate}
        \item It can transition into state $q_{write}$. In this state the following happens.
        \begin{enumerate}
            \item Non-deterministically select a \emph{symbol} from $\Sigma$.
            \item Write the symbol at the position of the current head of the second tape and move it to its right.
            \item Transition back to the state $q_{decide}$.
        \end{enumerate}
        \item It can transition into state $q_{stop}$ (Generate phase ends).
    \end{enumerate}
    \item \textbf{Simulate Phase: } This phase starts once the turning machine hits the $q_{stop}$ state.
    \begin{enumerate}
        \item Copy the contents (string) of tape 2 and paste it after the input stored in tape 1 (separated by a special symbol) so that tape 1 now holds some encoding of form $(\langle M_w \rangle, x)$.
        \item Call the universal turning machine ($A_{TM}$) with contents of tape 1 as its input and return its result (accept \emph{iff} it accepts).
    \end{enumerate}
\end{enumerate}

To see how this NTM is equivalent to the originally envisioned machine note that in the generate phase the NTM can non-deterministically generate any string from $\Sigma^*$ in tape 2. Then in simulate phase we copy it and append it to the end of tape 1 and use the universal turning machine to check if $M_w$ accepts the generated string or not.
\end{soln}

\newpage
\begin{prob}
An \textit{empty stack pushdown automaton} is a tuple $P'=(Q,\Sigma,\Gamma,\Delta,q_0,Z)$ where $Q$, $\Sigma$, $\Gamma$, $\Delta$, and $q_0$ have the same meaning as in definition of a PDA. The last component $Z\in\Gamma$ is the initial stack symbol: it is present on the stack in the beginning of every run. In other words, the initial instantaneous description in a run of $P'$ on $x\in\Sigma^*$ is $(q_0,x,Z)$. We say that such an automaton $P'$ accepts string $x$ if there is a run of $P'$ on $x$ that ends with an empty stack (and in any state). Formally, \text{$P'$ accepts $x$} if there exists a state $q\in Q$ such that $(q_0,x,Z)\vdash^*_{P'}(q,\varepsilon,\varepsilon)$. The language recognized by an empty stack pushdown automaton $P'$ is the set of strings accepted by $P'$.
\begin{enumerate}
\item \textbf{[1 mark]} Prove that if a language $L$ is recognized by some PDA $P$, then it is recognized by some empty stack PDA $P'$. (Describe a construction of $P'$ from $P$.)
\item \textbf{[1 mark]} Prove that if a language $L$ is recognized by some empty stack PDA $P'$, then it is recognized by some PDA $P$. (Describe a construction of $P$ from $P'$.)
\end{enumerate}
\end{prob}
\begin{soln}
    We give a polynomial time reduction from the \texttt{Clique} problem to this problem, and then exhibiting a non-deterministic algorithm that solves \texttt{SubgraphIsomorphism} in
    polynomial time.\nl
    \begin{claim}
        \texttt{Clique} $\le_P$ \texttt{SubgraphIsomorphism}
    \end{claim}
    \begin{proof}
        Consider the reduction as follows. Denote by $K_r$ the complete graph of size $r$ (by size we mean the number of vertices here). Then for input $(G, k)$ to \texttt{Clique}, let $(G, K_k)$ be the corresponding input to
        \texttt{SubgraphIsomorphism}.\nl 
        \begin{note}
            We assume that for $(G, k)$ is a non-trivial input to the \texttt{Clique} problem, i.e., $k \le |V(G)|$ (other inputs are necessarily not in \texttt{Clique} due to size reasons). This is
            necessary, else our reduction won't be polynomial time for large $k$.
        \end{note}
        Now that we have $k \le V(G)$, we can build a clique in time polynomial in $k$, which is at most polynomial in $|V(G)|$, and thus at most polynomial in the input size. Hence, this
        reduction (including the copying of the graph $G$) is polynomial-time. Now we show that it is indeed a valid reduction.\nl
        \begin{claim}
            $(G, k) \in \texttt{Clique} \iff (G, K_k) \in \texttt{SubgraphIsomorphism}$.
        \end{claim}
        \begin{proof}
            First, suppose that $(G, k) \in \texttt{Clique}$. Then there is a clique of size $\ge k$ in $G$, and hence there is also a clique of size $k$ in $G$ (removing the extra vertices
            gives rise to a clique too). Hence, there is a subgraph of $G$ which is a complete graph of size $k$. Since all complete graphs of size $k$ are isomorphic, there exists a subgraph of $G$
            isomorphic to $K_k$, from where we have that $(G, K_k) \in \texttt{SubgraphIsomorphism}$.\nl
            Now suppose that $(G, K_k) \in \texttt{SubgraphIsomorphism}$. Then there exists a subgraph which is isomorphic to the complete graph of size $k$. Hence, there exists a clique in the
            graph of size $\ge k$ (formed by the vertices of this subgraph), which implies that $(G, k) \in \texttt{Clique}$, as needed.
        \end{proof}
        This shows that this reduction is indeed valid, and hence we have proven the claim.
    \end{proof}
    Now we exhibit a non-deterministic algorithm that solves \texttt{SubgraphIsomorphism} in polynomial time. Consider the following algorithm:
    \begin{enumerate}
        \item Non-deterministically choose a subset $V'$ of $V$ and a subset $E'$ of $E$, and remove the edges from $E'$ that are incident to a vertex not in $V'$.
        \item Now call the non-deterministic Turing machine that solves the graph isomorphism problem for the graphs $(G', (V', E'))$.
    \end{enumerate}
    This clearly works in non-deterministic polynomial time, so \texttt{SubgraphIsomorphism} is in NP, which when combined with the first claim (which implies that
    \texttt{SubgraphIsomorphism} is \texttt{NP}-hard), implies that \texttt{SubgraphIsomorphism} is \texttt{NP}-complete.
\end{soln}


\newpage
\begin{prob}
Prove that the class of context-free languages is closed under intersection with regular languages. That is, prove that if $L_1$ is a context-free language and $L_2$ is a regular language, then $L_1\cap L_2$ is a context-free language.
\end{prob}
\begin{soln}
DFA $D=(Q,\{0,1\},\delta,q_0,A)$ is minimal and recognises the above given language, where\\

$Q=\{q_0,q_1,q_2,q_4,q_5\}$
\newline $A=\{q_1,q_5\}$
\newline $\delta$ described in the below table

\begin{center}
\begin{tabular}{|l|l|l|}
\hline
 & 0 & 1 \\
\hline
$q_0$ & $q_0$ & $q_1$  \\
\hline
$q_1$ & $q_2$ & $q_0$  \\
\hline
$q_2$ & $q_4$ & $q_5$  \\
\hline
$q_4$ & $q_2$ & $q_0$  \\
\hline
$q_5$ & $q_4$ & $q_5$  \\
\hline
\end{tabular}
\end{center}

Strings for each distinct pair q,q' is described in the below table
\begin{center}
\begin{tabular}{|l|l|l|l|l|l|}
\hline
 & $q_0$ & $q_1$ &$q_2$& $q_4$ & $q_5$\\
\hline
$q_0$ & - & $\epsilon$ & 01 & 1& $\epsilon$  \\
\hline
$q_1$  & $\epsilon$  & - & $\epsilon$ & $\epsilon$ & 1  \\
\hline
$q_2$  & 01 & $\epsilon$ & - & 1& $\epsilon$  \\
\hline
$q_4$  & 1 & $\epsilon$ & 1 & -& $\epsilon$  \\
\hline
$q_5$  & $\epsilon$ & 1 & $\epsilon$ & $\epsilon$ & -  \\
\hline
\end{tabular}
\end{center}




\end{soln}

\newpage
\begin{prob}
Consider the following two languages over the alphabet $\{a,b,c,d\}$.
\[L_1=\{a^mb^nc^md^n\text{ }|\text{ }m,n\in\mathbb{N}\cup\{0\}\}\text{,}\]
\[L_2=\{a^mb^nc^nd^m\text{ }|\text{ }m,n\in\mathbb{N}\cup\{0\}\}\text{.}\]
Determine whether each of these languages is context-free, and prove your answer. 
\end{prob}
\begin{soln}
\begin{claim}
$L_1$ is not context-free
\end{claim}
\begin{proof}
Suppose $L_1$ is context-free. Then it must satisfy the pumping lemma for context-free languages.
But as we will show in the claim below $L_1$ does not satisfy the pumping lemma. Hence $L_1$ can
not be context-free.
\begin{claim}
For all $p \in \mathbb{N}$ there exists a string $w \in L_1$ such that $|w| \geq p$ and
for all strings $u, v, x, y, z \in \Sigma^*$ such that $w = uvxyz$  and one of the following does 
not hold
\begin{enumerate}
    \item  $|vy| > 0$
    \item $|vxy| \leq p$
    \item  $uv^ixy^iz \in L_1$ for all $i \in \mathbb{N} \cup \{0\}$.
\end{enumerate}
\end{claim}
\begin{proof}
For any $p \in \mathbb{N}$ consider the string $w = a^p b^p c^p d^p$ and its partition $w = uvxyz$.
Then we have 3 cases
\begin{enumerate}
    \item $vxy$ is a substring of $a^p b^p$. In this case if we assume property 1 and 2 then by pumping once (i.e. $i=2$)
    we have that $uv^2xy^2z \notin L_1$ because number of a's increase while number of c's remain the same.
    Hence property 3 is violated here.
    
    \item $vxy$ is a substring of $b^p c^p$. In this case if we assume property 1 and 2 then by pumping once (i.e. $i=2$)
    we have that $uv^2xy^2z \notin L_1$ because number of b's increase while number of d's remain the same.
    Hence property 3 is again violated here.
     
    \item $vxy$ is a substring of $c^p d^p$. In this case if we assume property 1 and 2 then by pumping once (i.e. $i=2$)
    we have that $uv^2xy^2z \notin L_1$ because number of c's increase while number of a's remain the same.
    Hence property 3 is again violated here.
\end{enumerate}
Thus for each case we have proved the claim.
\end{proof}
Hence $L_1$ can not be context-free.
\end{proof}

\begin{claim}
$L_2$ is context-free
\end{claim}
\begin{proof}
Consider the language of grammar of $G = (N, \Sigma, R, S)$ where $N = \{S, S'\}$,
and $\Sigma = \{a, b, c, d\}$ and following rules under $R$:
\begin{align*}
    S &\rightarrow \varepsilon\\
    S &\rightarrow a S d\\
    S &\rightarrow S'\\
    S' &\rightarrow \varepsilon\\
    S' &\rightarrow b S' c\\
\end{align*}

\begin{claim}
$L_2 \subseteq \mathcal{L}(G)$
\end{claim}
\begin{proof}
Consider any string $s \in L_2$ as $a^m b^n c^n d^m$.
Then to generate this string via grammar $G$ we may apply the production rule 2 $m$ times,
followed by production rule 3 and then production rule 5 applied $n$ times finally followed by production rule 4.
In the corner case when $m, n = 0$ we may simply apply production rule 1 and be done with it.
\end{proof}

\begin{claim}
$\mathcal{L}(G) \subseteq L_2$
\end{claim}
\begin{proof}
\begin{claim}
If $S' \derives s$ then $s = b^n c^n$ for some $n \in \mathbb{N} \cup \{0\}$
\end{claim}
\begin{proof}
Proof by induction on number of production rules used in shortest derivation.
\paragraph{Base Case} Number of production rules used = 1, then we have must have used the rule 4 giving us $s = \varepsilon$.
\paragraph{Inductive Case} If we have used more than one production rule then our first production rule must have been
$S' \produces b S' c \derives b b^n c^n c$ where the derivation follows from inductive hypothesis.
\end{proof}

\begin{claim}
If $S \derives s$ then $s = a^m b^n c^n d^m$ for some $m, n \in \mathbb{N} \cup \{0\}$
\end{claim}
\begin{proof}
Proof by induction on number of production rules used in shortest derivation.
\paragraph{Base Case} Number of production rules used = 1, then we have must have used the rule 1 giving us $s = \varepsilon$.
\paragraph{Inductive Case} If we have used more than one production rule then our first production rule must have been either
\begin{enumerate}
    \item  $S \produces a S d \derives a a^m b^n c^n d^m d$ where the derivation follows from inductive hypothesis.
    \item $S \produces S' \derives b^n c^n$ where the derivation follows from the previous claim. Here $m = 0$.
\end{enumerate}
\end{proof}
\end{proof}

\end{proof}
\end{soln}

\newpage
\begin{prob}
Prove that the language $\{a^mb^n\text{ }|\text{ }n\text{ is a multiple of }m\}\subseteq\{a,b\}^*$ is not context-free.
\end{prob}
\begin{soln}
$DIAG$ is an example of a language which is unrecognizable in a real-life TM but is \emph{decidable} by a WTM. As we may simply copy the input of the WTM into its query tape and immediately goto the state $q_{query}$. If we transition into state $q_{yes}$ then we \emph{accept} the string, else we reject it. This, WTM decides the language $DIAG$.

Interestingly, the language $DIAG^{WTM}$ defined as
\[
    DIAG^{WTM} = \{ w \; | \; w \notin \mathcal{L}(W_w) \}
\]
is unrecognizable. Where $W_w$ denotes the wonderland turning machine whose encoding is $w$ (in case $w$ is not a valid encoding, $W_w$ is a WTM which rejects all strings).

\begin{proof}
To prove this, we will use the proof very similar to the proof of unrecognizability of $DIAG$ in real-life.
Suppose, $DIAG^{WTM}$ is recognizable in wonderland. Then let $W_w$ be the WTM that recognizes $DIAG^{WTM}$.
Then consider the string $w$. Two cases arise :
\begin{enumerate}
    \item $W_w$ \emph{accepts} $w$. Since $\mathcal{L}(W_w) = DIAG^{WTM}$, this implies that $w \in DIAG^{WTM}$.
    But $w \in DIAG^{WTM}$ implies that $w \notin \mathcal{L}(W_w)$ by definition of $DIAG^{WTM}$ which is a contradiction.
    
    \item $W_w$ does not \emph{accept} $w$. Since $w \notin \mathcal{L}(W_w)$ this implies that $w \in DIAG^{WTM}$,
    and since $W_w$ recognizes $DIAG^{WTM}$ it must also accept $w$ which is again a contradiction.
\end{enumerate}
Therefore it can not be the case that such a WTM $W_w$ exists which recognizes $DIAG^{WTM}$.
Hence it is proved that $DIAG^{WTM}$ is unrecognizable even in the wonderland.
\end{proof}
\end{soln}


\end{document}
