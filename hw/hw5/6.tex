\begin{soln}
    We use the following algorithm to create a valid solution:
    \begin{enumerate}
        \item First run $B$ on the circuit once to check if the circuit is satisfiable, and if it isn't, then indicate that there is no solution and exit. Else, carry on with the next steps.
        \item Let $A = \{\}$, and $C$ be the original circuit.
        \item For $i \in \{1, \ldots, n\}$, do the following: For $v \in \{\texttt{true}, \texttt{false}\}$, make a copy of $C$ and for all gates with input $x_i$, replace that input with a
            constant equal to $v$, and run $B$ on this circuit. If this fails, check the next value of $v$, else set $C$ to be this circuit, and $A \gets A \cup \{x_i = v\}$, and we go to the next
            $i$.
        \item Return the assignment $A$.
    \end{enumerate}
    In the case where we exit on step 1, there is no solution, so consider the case where there is a solution. Suppose we are at iteration $i$. Then since we have not exited on step 1, so there must exist a solution. Using induction, we can show that at the end of iteration $i$, there is an
        assignment where the first $i$ variables are assigned according to the current $A$, as follows:
        \begin{enumerate}
            \item Base case ($i = 0$) is true since we haven't exited on step 1.
            \item Now if there is a solution with the first $i - 1$ assignments, then consider such a solution. If $x_i = \texttt{false}$ in any such assignment, $B$ shall return true, so we will be
                able to add $x_i = \texttt{false}$ to the assignment $A$ and move on. In the other case, it must happen that $x_i = \texttt{true}$, so we will be able to add this to the
                assignment $A$ and move on.
        \end{enumerate}
    Now using this invariant for $i = n$, we are done.
\end{soln}
