\documentclass[10pt,addpoints]{exam}
\usepackage[english]{babel}
\usepackage[a4paper,top=2cm,bottom=2cm,left=2cm,right=2cm,marginparwidth=1.75cm]{geometry}
\usepackage{boites}
\usepackage{amsmath}
\usepackage{amsfonts}
% \usepackage{amsthm}
\usepackage{amssymb}
\usepackage{graphicx}
\usepackage[colorinlistoftodos]{todonotes}
\usepackage[colorlinks=true, allcolors=blue]{hyperref}
\usepackage{import}
\usepackage{pdfpages}
\usepackage{transparent}
\usepackage{xcolor}
\usepackage{algorithmicx}
\usepackage{algpseudocode}

\usepackage{thmtools}
\usepackage{enumitem}
\usepackage[framemethod=TikZ]{mdframed}

\usepackage{xpatch}

\makeatletter
\xpatchcmd{\endmdframed}
{\aftergroup\endmdf@trivlist\color@endgroup}
{\endmdf@trivlist\color@endgroup\@doendpe}
{}{}
\makeatother

%\usepackage[poster]{tcolorbox}
%\allowdisplaybreaks
%\sloppy

\usepackage[many]{tcolorbox}

%\xpatchcmd{\proof}{\itshape}{\bfseries\itshape}{}{}

\setlength{\parindent}{0pt}

\setlength{\fboxsep}{1em}
\def\breakboxskip{7pt}
\def\breakboxparindent{0em}

\newenvironment{proof}{\begin{breakbox}\textit{Proof.}}{\hfill$\square$\end{breakbox}}
\newenvironment{ans}{\begin{breakbox}\textit{Answer.}}{\end{breakbox}}
\newenvironment{soln}{\begin{breakbox}\textit{Solution.}}{\end{breakbox}}

% \tcolorboxenvironment{proof}{
%     blanker,
%     before skip=\topsep,
%     after skip=\topsep,
%     borderline={0.4pt}{0.4pt}{black},
%     enforce breakable,
%     left=12pt,
%     right=12pt,
%     top=12pt,
%     bottom=12pt,
% }
% 
% \tcolorboxenvironment{ans}{
%     blanker,
%     before skip=\topsep,
%     after skip=\topsep,
%     borderline={0.4pt}{0.4pt}{black},
%     enforce breakable,
%     left=12pt,
%     right=12pt,
%     top=12pt,
%     bottom=12pt,
% }

\mdfdefinestyle{enclosed}{
    linecolor=black
    ,backgroundcolor=none
    ,apptotikzsetting={\tikzset{mdfbackground/.append style={fill=gray!100,fill opacity=.3}}}
    ,frametitlefont=\sffamily\bfseries\color{black}
    ,splittopskip=.5cm
    ,frametitlebelowskip=.0cm
    ,topline=true
    ,bottomline=true
    ,rightline=true
    ,leftline=true
    ,leftmargin=0.01cm
    ,linewidth=0.02cm
    ,skipabove=0.01cm
    ,innerbottommargin=0.1cm
    ,skipbelow=0.1cm
}

\mdfdefinestyle{proofstyle}{
    linecolor=black
    ,backgroundcolor=none
    ,apptotikzsetting={\tikzset{mdfbackground/.append style={fill=gray!100,fill opacity=.3}}}
    ,frametitlefont=\sffamily\bfseries\color{black}
    ,splittopskip=.5cm
    ,frametitlebelowskip=.0cm
    ,topline=true
    ,bottomline=true
    ,rightline=true
    ,leftline=true
    ,leftmargin=0.01cm
    ,linewidth=0.02cm
    ,skipabove=0.01cm
    ,innerbottommargin=0.1cm
    ,skipbelow=0.1cm
    ,ntheorem=false
}

\mdfsetup{%
    middlelinecolor=black,
    middlelinewidth=1pt,
roundcorner=4pt}


\mdtheorem[style=enclosed]{theorem}{Theorem}
\mdtheorem[style=enclosed]{ques}{Question}
\mdtheorem[style=enclosed]{prob}{Problem}
\mdtheorem[style=enclosed]{lemma}{Lemma}[prob]
\mdtheorem[style=enclosed]{claim}{Claim}[prob]
\mdtheorem[style=enclosed]{defn}{Definition}
\mdtheorem[style=enclosed]{notn}{Notation}
\mdtheorem[style=enclosed]{obs}{Observation}
\mdtheorem[style=enclosed]{eg}{Example}
\mdtheorem[style=enclosed]{cor}{Corollary}
\mdtheorem[style=enclosed]{note}{Note}
%\mdtheorem[style=proofstyle]{proof}{Proof.}
% \mdtheorem[style=proofstyle]{ans}{Answer.}

% \let\theproof=\false
% \let\theans=\false
% \let\thetheorem=\relax
% \let\thelemma=\relax
% \let\theclaim=\relax
% \let\theques=\relax
% \let\thedefn=\relax
% \let\thenotn=\relax
% \let\theobs=\relax
% \let\thecor=\relax
\let\thenote=\relax

%\renewcommand\qedsymbol{$\blacksquare$}
\newcommand{\nl}{\vspace{0.2cm}\\}
\newcommand{\mc}{\mathcal}
\newcommand{\mi}{\mathit}
\newcommand{\mf}{\mathbf}
\newcommand{\mb}{\mathbb}
\renewcommand{\L}{\mc{L}}
\newcommand{\hd}{\hat{\delta}}
\DeclareMathOperator{\takelast}{last}
\DeclareMathOperator{\xor}{xor}

\newcommand{\incfig}[1]{%
    \def\svgwidth{\columnwidth}
    \import{./figures/}{#1.pdf_tex}
}
\pdfsuppresswarningpagegroup=1

\usepackage{tfrupee}

\pagestyle{head}

\firstpageheader{COL352\vspace{5mm}\\ Release date: March 6, 2021}{}{Homework 2\\ \vspace{5mm}Deadline: March 13, 2021: 23:00}
\firstpageheadrule

\begin{document}

%\noindent\href{https://moodle.iitd.ac.in/mod/forum/discuss.php?d=19210#p26977}{\textbf{Read the instructions carefully.}}
%
%\vspace{0.5cm}
%
\noindent This homework is primarily about proving that certain languages $L$ are not regular. For this, we have the Pumping Lemma and the Myhill-Nerode Theorem at our disposal. Recall that the Pumping Lemma merely gives a sufficient condition for non-regularity. In some cases, using closure properties might give a much cleaner proof: assume that $L$ is regular, then argue that some other language $L'$ must also be regular, then apply Pumping Lemma to show that $L'$ is, in fact, not regular. Remember that you can use any claim proven in class and in the previous quizzes and homeworks without reproducing its proof.


\begin{prob}
Prove that the language $\{x\text{ }|\text{ }x\text{ is the binary representation of }3^{n^2}\text{ for some }n\in\mathbb{N}\}$ is not regular.
\end{prob}


\begin{soln}
\begin{notn}
Let us define $C(s)$ for $s \in \{a, b\}^*$ to be equal to $\# a's - 2 \# b's$.
Let's call a string $s \in \{a, b\}^*$ \textbf{balanced} if $C(s) = 0$.
\end{notn}
\begin{lemma}
Let $s$ be a non-empty balanced string. If $b$ is the first character of $s$ then either $aa$ is a suffix of $s$, or,
$s$ can be partitioned into 2 (non-empty) substrings $s_1, s_2$ such that $s = s_1 s_2$ and
both $s_1, s_2$ are balanced.
\end{lemma}
\begin{proof}
Firstly, since $s$ is balanced, its length must be three times the number of $b$s in it, so it is at least 3 characters long.
We shall suppose that $s$ can not be partitioned into 2 balanced strings and then show that $s$
must have $aa$ as its suffix using proof by contradiction.\\

Suppose $s$ also does not end in $aa$. Let length of $s$ be $n$.
Now, note that
\begin{enumerate}
    \item $C(s[1 \ldots 1]) = C(b) = -2$
    \item $C(s[1 \ldots n-2]) = - C(s[n-1 \ldots n])$ and hence $> 0$ (since there is at least one $b$ in the last two characters by assumption).
\end{enumerate}
Thus, if we consider the sequence of values $C(s[1 \ldots i])$ for $i \in [1, n-2]$
we shall find that it starts with an initial value of $-2$, and terminates at a positive
value. Since by definition the only possible positive increase in the $C-$value is $1$, it must be the case that
at some point $C$ value must have become zero. In other words, $C(s[1 \ldots k]) = 0$ for
some $1 < k < n-2$. But this is a contradiction because then we can partition $s$ into 2
substrings $s[1 \ldots k]$ and $s[k+1 \ldots n]$ both of which must be balanced. Hence
our supposition that $s$ does not end in $aa$ must be incorrect.
\end{proof}

\begin{cor}
Let $s$ be a non-empty balanced string. If $s$ ends with $b$ then either $aa$ is a prefix of $s$, or,
$s$ can be partitioned into 2 (non-empty) substrings $s_1, s_2$ such that $s = s_1 s_2$ and
both $s_1, s_2$ are balanced.
\end{cor}
\begin{proof}
The corollary follows from the lemma above by observing that if $s$ is balanced then so is $rev(s)$,
and since then $b$ becomes the first character of $rev(s)$ the corollary follows.
\end{proof}

Now, we need to show that if $s$ is a \textbf{balanced} string then $S \derives s$.
We shall show by induction on length of $s$.

\paragraph{Base Case} $|s| = 0$. That is, $s = \varepsilon$. Then by rule 5 we have $S \produces s$

\paragraph{Inductive Case}
We will do case analysis on string $s$
\begin{enumerate}
    \item \textbf{Case 1}. $s$ can be partitioned into 2 (non-empty) substrings $s_1, s_2$
    such that $s = s_1 s_2$ and both $s_1, s_2$ are balanced.
    By inductive hypothesis we must have $S \derives s_1$ and $S \derives s_2$. Thus, we can write
    $SS \derives s_1 s_2$, or in other words, $SS \derives s$. But by rule 1 we have $S \produces SS$.
    Therefore we have that $S \produces SS \derives s$.
    
    \item \textbf{Case 2}. $s$ does not belong to above case and $s$ begins with character $b$.
    Then, by our lemma we know that $s$ must end in suffix $aa$.
    Thus, $s$ must be of form $bs'aa$ and clearly $s'$ is also a balanced string whose length is less than $|s|$.
    So by induction hypothesis $S \derives s'$ which gives $bSaa \derives bs'aa$. Combined with rule 3 this
    results in $S \produces bSaa \derives bs'aa = s$.
    
    \item \textbf{Case 3}. $s$ does not belong to above case(s) and $s$ ends in character $b$.
    Then, by our corollary to the lemma we know that $s$ must begin with prefix $aa$.
    Thus, $s$ must be of form $aas'b$ and clearly $s'$ is also a balanced string whose length is less than $|s|$.
    So by induction hypothesis $S \derives s'$ which gives $aaSb \derives aas'b$.
    Combined with rule 2 this results in $S \produces aaSb \derives aas'b = s$.
    
    \item \textbf{Case 4}. $s$ does not belong to any of the above cases. This means that $s$ must begin with, and end in character $a$ as well as $s$ can not be suitably partitioned into balanced strings.
    So, let $s$ be of form $as'a$. Then, note that because $s$ is balanced $C(s) = 0$ which implies
    $C(as') = -1$. Further, $C(a) = 1$. Note that $C(.)$ value of a prefix of $s$ can only decrease by
    $2$ at a time (and that too at an occurrence of $b$), and it transitioned from a positive value for $C(s[1\ldots 1])$ to a non-negative value for $C(s[1\ldots n - 1])$, so there must exist an index $i > 1$ such that $s[i] = b$ and $C(s[1 \ldots i-1]) = 1$ (the other case with the transition from a balance of 2 to a balance of 0 is impossible, since it would contradict the fact that $s$ can't be partitioned into two balanced substrings).\nl
    Hence the substrings $s[2 \ldots i-1]$ and $s[i+1 \ldots n-1]$ must be balanced ($C(.) = 0$).
    So we have reduced $s$ into the form $s = a s_1 b s_2 a$ where $s_1$ and $s_2$ are balanced.
    By induction, $S \derives s_1$ and $S \derives s_2$. Further, $S \produces a S b S a$ by rule 4.
    So together we have $S \produces a S b S a \derives a s_1 b s_2 a = s$.
\end{enumerate}
This completes our inductive step, establishing the inductive hypothesis.\\

By the inductive hypothesis we have proved that all balanced strings are generated by the given grammar.
\end{soln}

\newpage
\begin{prob}
Prove that the language $\{0^m1^n\text{ }|\text{ }m\neq n\}\subseteq\{0,1\}^*$ is not regular.
As a challenge, construct a clean proof using the pumping lemma only. However, no extra credit will be given for this.
\end{prob}

\begin{soln}
    \paragraph{1.} Consider any PDA $P = (Q, \Sigma, \Gamma, \Delta, q_0, A)$. We shall construct an empty stack PDA $P' = (Q', \Sigma, \Gamma', \Delta', q_{init}, A, Z)$ as follows:
    \begin{enumerate}
        \item $Q' = Q \uplus \{q_{init}, q_{accept}\}$.
        \item $\Gamma' = \Gamma \uplus \{\bot, Z\}$.
        \item $\Delta' = \Delta \cup \{(q_{init}, \epsilon, Z, q_0, \bot)\} \cup \{(q, \epsilon, \epsilon, q_{accept}, \epsilon) \mid q \in A\} \cup \{(q_{accept}, \epsilon, a,
            q_{accept}, \epsilon) \mid a \in \Gamma'\}$.
    \end{enumerate}
    We prove the claim in the problem in two parts.\nl
    \begin{claim}
        If $P$ accepts a string $x$, then $P'$ accepts $x$.
    \end{claim}
    \begin{proof}
        Consider an ``accepting" run of $P$ on $x$. More formally, we have $(q_0, x, \epsilon) \changesto_P^* (q, \epsilon, s')$ for some $q \in A$ and $s' \in \Gamma^*$. Note that this implies that we have $(q_0,
        x, \bot) \changesto_P^* (q, \epsilon, s'\bot)$, and since $\Delta \subseteq \Delta'$, we have $(q_0, x, \bot) \changesto_{P'}^* (q, \epsilon, s'\bot)$. By the definition of $\Delta'$, we have
        $(q_{init}, x, Z) \changesto_{P'} (q_0, x, \bot) \changesto_{P'}^* (q, \epsilon, s)$ for $s = s'\bot \in \Gamma'^*$ and $q \in A$. From the third set in $\Delta'$, we get that $(q,
        \epsilon, s) \changesto_{P'} (q_{accept}, \epsilon, s)$, and since the last set represents transitions that lead to $q_{accept}$ for any stack character, we have $(q_{accept},
        \epsilon, s) \changesto_{P'}^* (q_{accept}, \epsilon, \epsilon)$. Combining this with the previous relation, we get $(q_{init}, x, Z) \changesto_{P'}^* (q_{accept}, \epsilon, \epsilon)$, as needed.
    \end{proof}
    \begin{claim}
        If $P'$ accepts a string $x$, then $P$ accepts $x$.
    \end{claim}
    \begin{proof}
        Consider an ``accepting run" of $P'$ on $x$. We call an i.d. \textit{good} if it is of the form $(q, \epsilon, \epsilon)$ for some $q \in Q'$. An accepting run can end only at a good i.d..\nl
        The first i.d. in the run must be $(q_{init}, x, Z)$ (which is not good), and the second one must exist and be equal to $(q_0, x, \bot)$, which is also not good.\nl
        We first claim that the last i.d. in the run is $(q_{accept}, \epsilon, \epsilon)$.
        Note that since the run is accepting, the last state has to be good. Since the second i.d. in the run has $\bot$ on the stack, and the last doesn't, there has to be some transition in the run
        where $\bot$ is popped. This must be done in a transition from $q_{accept}$ to itself, since no other transition involves popping of $\bot$. Moreover, this transition is the only such
        transition, since we push $\bot$ in only the first transition, and no other transition is able to do so.\nl
        Hence, in the run, we reach $q_{accept}$ at some point, and since any transition out of $q_{accept}$ leads to $q_{accept}$, the last state in the run is $q_{accept}$.\nl
        %Now we have that there is a unique i.d. in the run that pops $\bot$ from the stack.
        Consider the last i.d. in the run that is not at state $q_{accept}$. This is clearly not the last i.d. in the run itself, since the the last i.d. in the run is at state $q_{accept}$. Note that
        the part of the run from this i.d. to the final i.d. doesn't involve reading and discarding characters from the string at that point, so this i.d. is of the form $(q, \epsilon, s)$ for
        some $q \in Q, s \in \Gamma'^*$, and $\bot$ is a character in $s$ (this follows from the fact that popping $\bot$ can happen only in a transition from $q_{accept}$ to itself). From the fact that this is the last i.d. not in $q_{accept}$, we have that $q$ must be in $A$ (by the definition of
        $\Delta'$). From the fact that there is exactly one transition involving popping of $\bot$ and it is after this i.d., we have the fact that the part of the run from the second i.d. to
        this i.d. doesn't involve pushing or popping $\bot$ onto/from the stack, and $\bot$ is on the stack (moreover it is at the bottom position).\nl
        From here, we know that the original run is of the form $$(q_{init}, x, Z) \changesto_{P'} (q_0, x, \bot) \changesto_{P'}^* (q, \epsilon, s) \changesto_{P'}^* (q_{accept}, \epsilon,
        \epsilon)$$
        where the part $(q_0, x, \bot) \changesto_{P'}^* (q, \epsilon, s)$ always has $\bot$ at the bottom of the stack, and no transition in this part involves $\bot$, and $q \in A$. Moreover, there are no
        transitions in this part which are in $\Delta'$ but not in $\Delta$, since $q_{init}$ is always a source and never a sink and $q_0 \ne q_{init}$, and $q_{accept}$ is only in a suffix
        of the run that starts after this part of the run gets over.\nl
        $Z$ is also absent in this part of the run, since $Z$ is involved in only the first transition in the run and nowhere else.\nl 
        From this information, it follows that $(q_0, x, \bot) \changesto_P^* (q, \epsilon, s)$ (note that $\bot, Z$ are not in the stack alphabet of $P$, but we resort to a slight abuse of
        notation here, since we are guaranteed that all transitions in this part of the run are in $\Delta$, and no transition in $\Delta$ involves $\bot, Z$ at all -- we shall do this in later
        parts too when it is obvious), where all transitions in this part are in $\Delta$, and $s = s'\bot$ for some $s \in \Gamma^*$ (note that $s$ doesn't contain $Z$).\nl
        Hence, it follows that $(q_0, x, \epsilon) \changesto_P^* (q, \epsilon, s')$ for $q \in A$, whence we are done.
    \end{proof}
    \paragraph{2.} Consider any empty stack PDA $P' = (Q', \Sigma, \Gamma', \Delta', q_0', A', Z)$. We shall add an extra symbol $\bot$ to $\Gamma'$, and add some new states and relevant transitions to $\Delta'$, and
    change $q_0', A'$ too, to get a PDA $P = (Q, \Sigma, \Gamma, \Delta, q_{init}, \{q_{accept}\})$ as follows:
    \begin{enumerate}
        \item $Q = Q' \uplus \{q_{init}, q_{start}, q_{accept}\}$.
        \item $\Gamma = \Gamma' \uplus \{\bot\}$.
        \item $\Delta = \Delta' \cup \{(q_{init}, \epsilon, \epsilon, q_{start}, \bot), (q_{start}, \epsilon, \epsilon, q_0, Z)\} \cup \{(q, \epsilon, \bot, q_{accept}, \epsilon) \mid \forall q \in Q'\}$.% \cup \{(q_{accept}, a, ) \mid \forall a \in \Sigma\}$.
    \end{enumerate}
    We prove the claim in the problem in two parts.\nl
    \begin{claim}
        If $P$ accepts a string $x$, then $P'$ accepts $x$.
    \end{claim}
    \begin{proof}
        Consider the ``accepting run" of $P$ on $x$. We have $(q_{init}, x, \epsilon) \changesto_{P} (q_{start}, x, \bot) \changesto_{P} (q_0, x, Z\bot) \changesto_{P}^* (q, \epsilon, \bot) \changesto_{P} (q_{accept}, \epsilon, \epsilon)$.\nl
        We first show why the first, second and the last transitions are as claimed, as well as why the last two i.d.s are as claimed, below:\nl
        Note that the first two transitions are always as specified since there is no path from $q_{init}$ to $q_{accept}$ that doesn't go through $q_0$ as well as $q_{start}$.\nl
        Note that $q_{accept}$ can only be a sink state in a transition, so $q_{accept}$ can never be in the middle part of
        the above run, and hence $\bot$ is never popped off the stack in the middle part of the run. So the part of the stack below $\bot$ always remains the same throughout the run (i.e., it always
        remains empty), so when the final state is reached, we should have run out of the string, and since we reach $q_{accept}$, $\bot$ must have been popped out of the stack, and this
        implies (from our previous sentence) that the stack must be empty at the last state. This shows why the run is as claimed.\nl
        Now note that from the same argument, $q_{init}$ and $q_{accept}$ are never present in any sequence of i.d.s that gives rise to $(q_0, x, Z\bot) \changesto_{P}^* (q, \epsilon, \bot)$ (since
        these states can only act as a source and a sink respectively), and $q_{start}$ is never in any sequence of i.d.s that gives rise to the same relation because of never being
        a sink of any transition other than the one in $\Delta'$ but not in $\Delta$, and hence in such a sequence of i.d.s, there is no transition between consecutive i.d.s that is in $\Delta'$ but not in
        $\Delta$. These two observations, combined with the fact that $\Delta$ has no transitions which concern $\bot$, imply that $(q_0, x, Z\bot) \changesto_{P'}^* (q, \epsilon, \bot)$
        without ever popping $\bot$ off the stack, and hence using a simple induction, we have $(q_0, x, Z) \changesto_{P'}^* (q, \epsilon, \epsilon)$ for some $q \in Q'$, which means that
        $P'$ accepts $x$, as required.
    \end{proof}
    \begin{claim}
        If $P'$ accepts a string $x$, then $P$ accepts $x$.
    \end{claim}
    \begin{proof}
        The proof is very similar to the previous part; here, we instead consider an ``accepting run" of $P'$ on $x$, say $(q_0, x, Z) \changesto_{P'}^* (q, \epsilon,
        \epsilon)$ for some $q \in Q'$. Note that since $\Delta' \subseteq \Delta$, we have $(q_0, x, Z) \changesto_{P}^* (q, \epsilon, \epsilon)$ as well. By a simple induction as
        mentioned in the previous part, we have that
        this implies that $(q_0, x, Z\bot) \changesto_{P}^* (q, \epsilon, \bot)$. Noting that $(q_{init}, x, \epsilon) \changesto_{P} (q_{start}, x, \bot) \changesto_{P} (q_0, x, Z\bot)$ and $(q, \epsilon, \bot) \changesto_{P}
        (q_{accept}, \epsilon, \epsilon)$, we see that $(q_{init}, x, \epsilon) \changesto_{P}^* (q_{accept}, \epsilon, \epsilon)$, from where it follows that $P$ accepts $x$.
    \end{proof}
\end{soln}


\newpage
%Prove that the language $\{1^{n_1}01^{n_2}0\cdots01^{n_k}\text{ }|\text{ }k\geq2\text{, and }n_1,\ldots,n_k\in\mathbb{N}\cup\{0\}\text{ are distinct}\}$ is not regular.

\begin{prob}
Construct the minimal DFA $D=(Q,\{0,1\},\delta,q_0,A)$ that recognizes the language
\[\{x\in\{0,1\}^*\text{ }|\text{ }x\text{ is the binary representation of a number coprime with }6\}\text{.}\]
Prove its minimality by giving a string $z_{q,q'}$ for each pair of distinct states $q,q'\in Q$ such that exactly one of $\widehat{\delta}(q,z_{q,q'})$ and $\widehat{\delta}(q',z_{q,q'})$ is in $A$. (Proof of correctness of your automaton is not required.)
\end{prob}

% % Hint: use product construction
% \begin{soln}

% Let the PDA associated with $L_1$ be P=$(Q_1,\Sigma,\Gamma,q_{01},\Delta_1,A_1$) and the DFA associated with $L_2$ be N=$(Q_2,\Sigma,\delta_2,q_{02},A_2)$
% \newline Now consider a PDA $P'$ = $(Q,\Sigma,\Gamma,q_0,\Delta,A)$
% where
% \newline $Q=Q_1 \times Q_2$
% \newline $q_0=(q_{01},q_{02})$
% \newline $\Delta_{01}$=  $\{( (q_1,q_2),a,B,(q_3,q_4),B') \mid \delta(q_1,a)=q_3 \in \Delta_1 , (q_2,a,B,q_4,B') \in \Delta_2$  \}
% \newline $\Delta_{02}$=  $\{( (q_1,q_2),\epsilon,\epsilon,(q3,q2),\epsilon) \mid (q_1,\epsilon,q_3) \in \Delta_2  \}$
% \newline $\Delta_{03}$=  $\{( (q_1,q_2),\epsilon,B,(q_1,q4),B') \mid  (q_2,\epsilon,B,q_4,B') \in \Delta_1$  \}
% \newline $\Delta = \Delta_{01} \cup \Delta_{02} \cup \Delta_{03}$
% \newline $A=\{(q_1,q_2) \mid q_1 \in A_1, q_2 \in A_2$ \}

% \begin{claim}
% $L(P')=L_1 \cap L_2$
% \end{claim}

% \begin{claim}
% $  L_1 \cap L_2 \subseteq L(P')$
% \end{claim}
% \begin{proof}
% Let $x \in L_1 \cap L_2$, now as x exist in $L_1$ and $L_2$ therefore there exist a accepted run in both P and N,
% Now I Claim the induction on length of string
% \paragraph{Induction Hypothesis}
% P(n): Induction on length of string
% If there is a run in N, which reads $x$ and reaches state $q_2'$, also 
% $( q_{01},x,\epsilon) ->_{P} (q_1',\epsilon,\alpha)$ then $( (q_{01},q_{02}),x,\epsilon) ->_{P'} ((q_1',q_2'),\epsilon,\alpha)$
% \paragraph{Base Case}
% n=0, Therefore x=



% \end{proof}

% \begin{claim}
% $L(P') \subseteq L_1 \cap L_2$
% \end{claim}
% \begin{proof}

% Let $x \in L(P')$,
%  Consider those runs, Now I Claim the induction on number of transitions
% \paragraph{Induction Hypothesis}
% P(n): Induction on number of transitions
% if $( (q_{01},q_{02}),x_1x_2,\epsilon) ->_{P'} ((q_1',q_2'),x_2,\alpha)$  then
% there is a run in N, which reads $x_1$ and reaches state $q_2'$, also 
% $( q_{01},x,\epsilon) ->_{P} (q_1',x_2,\alpha)$
% \paragraph{Base Case}
% n=0, therefore $x_1=\epsilon, q_1'=q_{01}, q_2'=q_{02}, \alpha = \epsilon$. This is trivial as for $\epsilon$ there is run from $q_{02}$ to $q_{02}$ in N, also in P $( q_{01},x,\epsilon) ->_{P} (q_{01},x,\epsilon)$
% \paragraph{Inductive Case}
% For $n>0$, There exist x',q3,q4 and alpha' such that  $( (q_{01},q_{02}),x_1x_2,\epsilon) ->_{P'} ((q_3,q_4),x'x_2,\alpha')$  and T transition  $( (q_3,q_4),x'x_2,\aplha') -> ((q_1',q_2'),x_2,\alpha)$ 
% as P(n-1) holds, now the last transition can be of three type according to in which subdelta it is present
% \begin{enumerate}
%     \item T Transition belong to $\Delta_{01}$, so x'=a, $(q_4,a,q_2') \in \Delta_1$, $(q_3,a,B,q_1',B') \in \Delta_2$ , where $B\alpha'=B'\alpha$
%     \newline now as P(n-1) holds so there is a run of $x_1[1...len(x_1)-1]$ in N which ends in $q_4$, now take transition $(q_4,a,q_2')$, therefore there is a run of $x_1$ in n which ends in $q_2'$.
%     Now as P(n-1) holds therefore $(q_{01},x_1x_2,\epsilon) ->_{P} (q_3,x'x_2,\alpha')$, now take the transition $( q_3,x'x_2,\alpha') -> (q_1',x_2,\alpha)$  this is possible as $(q_3,a,B,q_1',B') \in \Delta_2$ and hence induction holds in this case
    
%     \item T Transition belong to $\Delta_{02}$, so $x'=\epsilon$, $\alpha'=\alpha$,$q_{03}=q_1'$,$(q_4,a,q_2') \in \Delta_1$, 
%     \newline now as P(n-1) holds so there is a run of $x_1$ in N which ends in $q_4$, now take transition $(q_4,\epsilon,q_2')$, therefore there is a run of $x_1$ in n which ends in $q_2'$.
%      as P(n-1) holds therefore $(q_{01},q_{02}),x_1x_2,\epsilon) ->_{P} ((q_3,q_4),x'x_2,\alpha')$ therefore  $(q_{01},x_1x_2,\epsilon) ->_{P} (q_3,x'x_2,\alpha')$  and as x' = $\epsilon$,$\alpha'=\alpha$,$q_{03}=q_1'$ therefore this translates to  $(q_{01},x_1x_2,\epsilon) ->_{P} (q_1',x_2,\alpha)$ and hence induction holds in this case
     
%      \item T Transition belong to $\Delta_{03}$, so $x'=\epsilon$,$B\alpha'=B'\alpha$,$q_{04}=q_2'$ and $(q_3,a,B,q_1',B') \in \Delta_2$. 
%      \newline Now as P(n-1) holds so there is a run of $x_1$ in N which ends in $q_4$, now as $q_{04}=q_2'$, therefore there is a run of $x_1$ in N which ends in $q_2'$. 
%      As P(n-1) holds therefore $(q_{01},x_1x_2,\epsilon) ->_{P} (q_3,x'x_2,\alpha')$, now take the transition $( q_3,x'x_2,\alpha') -> (q_1',x_2,\alpha)$  this is possible as $(q_3,a,B,q_1',B') \in \Delta_2$ and hence induction holds in this case.
% \end{enumerate}
% As all the above cases are exhaustive therefore the induction holds.
% \newline Now as $x \in L(P')$ therefore $( (q_{01},q_{02}),x_1x_2,\epsilon) ->_{P'} ((q_1',q_2'),\epsilon,\alpha)$ where $q_1' \in A_1, q_2' \in A_2$
% Now from the above claim there exist a run on x in N that ends in $q_1'$ hence accepting x, i.e. $x \in L_2$, also from the above claim $( q_{01},x,\epsilon) ->_{P} (q_1',\epsilon,\alpha)$ and hence $x \in L_2$, combining both we get $x\in L_1 \cup L_2$ which further implies $L(P') \subseteq L_1 \cap L_2$
% % Now by observing the $\Delta$ we can see that the second element in the tuples of state space, it changed only when transition correspond to $\Delta_{01}$ or $\Delta_{02}$,and if we see the (inital state (second element), alphabet,final state(second element)) tuple of that transition then the same transition is also exist in $\Delta_2$, the starting state' second element also is the starting state of N, also the ending state's second element lies in A2(As the final state is accepted by P')
% % \newline Therefore following the same route as followed by the x in P', except the transition corresponding to $\Delta_{03}$, we can find the path in N, also and therefore x $\in L_2$.
% % \newline
% % \newline Similarly by observing the $\Delta$ we can see that the first element in the tuples of state space, it changed only when transition correspond to $\Delta_{01}$ or $\Delta_{03}$,and if we see the (inital state (first element), alphabet,push stack element,final state(second element),pop stack element) tuple of that transition then the same transition is also exist in $\Delta_1$, the starting state' first element also is the starting state of P, also the ending state's first element lies in A1(As the final state is accepted by P')
% % \newline Therefore following the same route as followed by the x in P', except the transition corresponding to $\Delta_{02}$, we can find the path in P, also and therefore x $\in L_1$.
% % \newline
% % \newline Combining both we get $x \in L1 \cap L2$. and hence $L(P') \subseteq L_1 \cap L_2$


% \end{proof}
% \end{soln}
\begin{soln}

Let the PDA associated with $L_1$ be $P=(Q_1,\Sigma,\Gamma,q_{01},\Delta_1,A_1$) and the DFA associated with $L_2$ be $D=(Q_2,\Sigma,\delta_2,q_{02},A_2)$.\nl
Now consider a PDA $P'$ = $(Q,\Sigma,\Gamma,q_0,\Delta,A)$, where
\begin{enumerate}
\item $Q=Q_1 \times Q_2$
\item $q_0=(q_{01},q_{02})$
\item $\Delta_{01}$=  $\{( (q_1,q_2),a,B,(q_3,q_4),B') \mid \delta(q_2,a)=q_4  , (q_1,a,B,q_3,B') \in \Delta_2$  \}
\item $\Delta_{02}$=  $\{( (q_1,q_2),\epsilon,B,(q_3,q_2),B') \mid  (q_1,\epsilon,B,q_3,B') \in \Delta_1$  \}
\item $\Delta = \Delta_{01} \cup \Delta_{02} $
\item $A=\{(q_1,q_2) \mid q_1 \in A_1, q_2 \in A_2$ \}
\end{enumerate}

\begin{claim}
$L(P')=L_1 \cap L_2$
\end{claim}
\begin{proof}


\begin{claim}
$  L_1 \cap L_2 \subseteq L(P')$
\end{claim}
\begin{proof}
Let $x \in L_1 \cap L_2$, now as $x$ is in $L_1$ and $L_2$, therefore there exist accepting runs of both $P$ and $D$ on $x$.\nl
Now we perform induction on the length of string (say $n$).
\paragraph{Induction Hypothesis}
$P(n):$ for any $x \in \Sigma^*$ with $|x|=n$, if $\hd(q_{02},x)=q_2'$ and $( q_{01},x,\epsilon) \to_{P} (q_1',\epsilon,\alpha)$ then $((q_{01},q_{02}),x,\epsilon) \to_{P'} ((q_1',q_2'),\epsilon,\alpha)$
\paragraph{Base Case} $n=0$, therefore $x=\epsilon$, hence $q_2'=q_{02}$, now as $( q_{01},\epsilon,\epsilon) \to_{P} (q_1',\epsilon,\alpha)$ therefore all transitions take $\epsilon$ as input. Now
considering transitions in $\Delta_{02}$ and following the same transitions as followed in $P$, (the second element in the state tuple remains the same throughout the run) and thus we can reach the
the ID $((q_1',q_2'),\epsilon,\alpha)$ and hence $(q_{01},q_{02}),x,\epsilon) \to_{P'} ((q_1',q_2'),\epsilon,\alpha)$, as needed.

\paragraph{Inductive Case} For $n>0$, let $x_1=x[1,2,..n-1]$. There exist $q_3,q_4$ such that $\hd(q_{02},x_1)=q_4$ and $\delta(q_4,x[n])=q_2'$ and $(q_{01},x_1,\epsilon) \to_{P}^*
(q_3,\epsilon,\alpha')$ and $(q_3,x[n],\alpha') \to_{P}^* (q_1',\epsilon,\alpha)$. Now as $x_1$ is of length $n-1$, and $P(n-1)$ is true, so $((q_{01},q_{02}),x_1,\epsilon) \to_{P'}
((q_3,q_4),\epsilon,\alpha')$.\nl
Now during $(q_3,x[n],\alpha') \to_{P}^* (q_1',\epsilon,\alpha)$ one transition takes $x[n]$ as input and all the other transition take $\epsilon$ as the input. Now take all the transition till the
transition corresponding to $x[n]$. Corresponding to these transitions we get transitions in $\Delta_{02}$ where the only first state of state tuple changes. Now take the transition where $x[n]$ is
the input. Corresponding to this there is a transition in $\Delta_{01}$. Then again take the $\epsilon$ transitions as done previously.\nl
So $((q_3,q_4),\epsilon,\alpha') \to_{P}^* ((q_1',q_2'),\epsilon,\alpha')$, Combining this with $(q_{01},q_{02}),x_1,\epsilon) \to_{P'}^* ((q_3,q_4),\epsilon,\alpha')$ we get 
 $(q_{01},q_{02}),x_1,\epsilon) \to_{P'}^* ((q_1',q_2'),\epsilon,\alpha')$.\nl
Now as $x \in L_1 \cap L_2$, therefore  $\hd(q_{02},x)=q_2'$ where $q_2' \in A_2$ and there exists a run $(q_{01},x,\epsilon) \to_{P}^* (q_1',\epsilon,\alpha)$ where $q_1' \in A_1$. 
Now using the above inductive claim $(q_{01},q_{02}),x,\epsilon) \to_{P'}^* ((q_1',q_2'),\epsilon,\alpha)$, but $(q_1',q_2') \in A$ as $q_1 \in A_1, q_2 \in A_2$. Therefore $x$ is accepted by
$P'$, and hence $x \in L(P')$ and therefore $L_1 \cap L_2 \subseteq L(P')$

\end{proof}

\begin{claim}
$L(P') \subseteq L_1 \cap L_2$
\end{claim}
\begin{proof}
Consider a string $x \in L(P')$. Consider an accepting run of $P'$ on $x$. We perform induction on number of transitions.
\paragraph{Induction Hypothesis}
$P(n)$: if $((q_1,q_2),x,\alpha) \to_{P'}^* ((q_1',q_2'),\epsilon,\alpha')$ in at most $n$ transitions, then $\hd(q_2,x)=q_2'$ and $( q_{1},x,\alpha) \to_{P}^* (q_1',\epsilon,\alpha')$
\paragraph{Base Case}
$n=0$ -- here $x=\epsilon, q_1'=q_1, q_2'=q_2, \alpha = \alpha'$. This is trivial as $\hd(q_2,x)=q_2$ because of $q_2'=q_2$, and in $P$, $(q_{1},x,\alpha) \to_{P}^* (q_{1},\epsilon,\alpha')$
\paragraph{Inductive Case}
For $n>0$, there exist $x',q_3,q_4$ and $\alpha'$ such that $x=ax'$ for some $a \in \Sigma \cup \{\varepsilon\}, x' \in \Sigma^*$
\[
((q_1, q_2), x, \alpha) \changesto_P ((q_3, q_4), x', \alpha'') \to_{P}^* ((q_1',q_2'),\epsilon,\alpha')
\]
The first transition can be of two type according to in which subdelta it is present

\begin{enumerate}
    \item First transition belong to $\Delta_{01}$, so $\delta(q_2,a)=q_4$, $(q_1,a,B,q_3,B') \in \Delta_1$ , where B and B' are such that after poping B and pushing B' to $\alpha$ we get $\alpha''$
    \newline now as P(n-1) holds so $\hd(q_{4},x')=q_2'$ and now using $\delta(q_2,a)=q_4$ we get $\hd(q_{2},x)=q_2'$. 
    Now as P(n-1) holds therefore $(q_{3},x',\alpha'') \to_{P}^* (q_1',x',\alpha')$, now the transition $( q_1,a,\alpha) \changesto (q_3,\epsilon,\alpha'')$  this is possible as $(q_1,a,B,q_3,B') \in \Delta_2$ combining both we get $( q_{1},x,\alpha) \to_{P}^* (q_1',\epsilon,\alpha')$ hence induction holds in this case
    
    % \item T Transition belong to $\Delta_{02}$, so $x'=\epsilon$, $\alpha'=\alpha$,$q_{03}=q_1'$,$(q_4,a,q_2') \in \Delta_1$, 
    % \newline now as P(n-1) holds so there is a run of $x_1$ in N which ends in $q_4$, now take transition $(q_4,\epsilon,q_2')$, therefore there is a run of $x_1$ in n which ends in $q_2'$.
    %  as P(n-1) holds therefore $(q_{01},q_{02}),x_1x_2,\epsilon) ->_{P} ((q_3,q_4),x'x_2,\alpha')$ therefore  $(q_{01},x_1x_2,\epsilon) ->_{P} (q_3,x'x_2,\alpha')$  and as x' = $\epsilon$,$\alpha'=\alpha$,$q_{03}=q_1'$ therefore this translates to  $(q_{01},x_1x_2,\epsilon) ->_{P} (q_1',x_2,\alpha)$ and hence induction holds in this case
     
     \item Transition belongs to $\Delta_{02}$, so $a=\epsilon$, $q_2=q_4$, $(q_1,a,B,q_3,B') \in \Delta_1$ , where B and B' are such that after popping B and pushing B' to $\alpha$ we get $\alpha''$
     \newline Now as P(n-1) holds $\hd(q_{4},x')=q_2'$, now as $q_{4}=q_2$ and $x=x'$, therefore  $\hd(q_{2},x_1)=q_2'$. 
     \newline Now as P(n-1) holds therefore $(q_{3},x',\alpha'') \to_{P}^* (q_1',x',\alpha')$, now the transition $( q_1,a,\alpha) \changesto (q_3,\epsilon,\alpha'')$  this is possible as $(q_1,a,B,q_3,B') \in \Delta_2$ combining both we get $( q_{1},x,\alpha) \to_{P}^* (q_1',\epsilon,\alpha')$ hence induction holds in this case
\end{enumerate}
As all the above cases are exhaustive therefore the induction holds.
\newline Now as $x \in L(P')$, therefore $( (q_{01},q_{02}),x,\epsilon) ->_{P'}^* ((q_1',q_2'),\epsilon,\alpha)$ where $q_1' \in A_1, q_2' \in A_2$
Now from the above claim $\hd(q_{02},x)=q_1'$ hence accepting x, i.e. $x \in L_2$, also from the above claim $( q_{01},x,\epsilon) ->_{P}^* (q_1',\epsilon,\alpha)$ and hence $x \in L_2$, combining both we get $x\in L_1 \cup L_2$ which further implies $L(P') \subseteq L_1 \cap L_2$
% Now by observing the $\Delta$ we can see that the second element in the tuples of state space, it changed only when transition correspond to $\Delta_{01}$ or $\Delta_{02}$,and if we see the (inital state (second element), alphabet,final state(second element)) tuple of that transition then the same transition is also exist in $\Delta_2$, the starting state' second element also is the starting state of N, also the ending state's second element lies in A2(As the final state is accepted by P')
% \newline Therefore following the same route as followed by the x in P', except the transition corresponding to $\Delta_{03}$, we can find the path in N, also and therefore x $\in L_2$.
% \newline
% \newline Similarly by observing the $\Delta$ we can see that the first element in the tuples of state space, it changed only when transition correspond to $\Delta_{01}$ or $\Delta_{03}$,and if we see the (inital state (first element), alphabet,push stack element,final state(second element),pop stack element) tuple of that transition then the same transition is also exist in $\Delta_1$, the starting state' first element also is the starting state of P, also the ending state's first element lies in A1(As the final state is accepted by P')
% \newline Therefore following the same route as followed by the x in P', except the transition corresponding to $\Delta_{02}$, we can find the path in P, also and therefore x $\in L_1$.
% \newline
% \newline Combining both we get $x \in L1 \cap L2$. and hence $L(P') \subseteq L_1 \cap L_2$

\end{proof}
Combining both of the above claims be get $L(P')=L_1 \cap L_2$


\end{proof}
\end{soln}


\newpage
\begin{prob}
Let $L_k\subseteq\{0,1\}^*$ be the language defined as $L_k=\{x\text{ }|\text{ }|x|\geq k\text{ and the EXOR of the last }k\text{ bits of }x\text{ is }1\}$. Prove that any DFA that recognizes $L_k$ has at least $2^k$ states. (By the way, observe that $L_k$ is recognized by an NFA with $O(k)$ states.)
\end{prob}
\begin{soln}
    Consider the following reduction, where we convert an input 3CNF formula $\phi$ for the \texttt{3SAT} problem to an input struct 3CNF formula $\varphi$ for the \texttt{Strict3SAT} problem. Here, $x, y, z$ are extra \emph{dummy} variables. Instantiation of a variable $x$ is either $x$ or $\overline x$.
    \begin{enumerate}
        \item Let $\varphi$ be initially empty (and-ing a clause with an empty clause leads to replacement of the empty clause with that clause).
        \item Break $\phi$ into clauses, and do the next 3 steps for each clause.
        \item If the clause is of the form $(a \lor b \lor c)$, set $\varphi \gets \varphi \land (a \lor b \lor c)$.
        \item Else if the clause is of the form $(a \lor b)$, set $\varphi \gets \varphi \land (a \lor b \lor x)$.
        \item Else, the clause is of the form $(a)$, so set $\varphi \gets \varphi \land (a \lor x \lor y)$.
        \item Finally, add the following $2^3 - 1$ extra clauses to $\varphi$ : For each possible \emph{instantiation} of variables $x, y, z$ $(x', y', z')$ except the instance $(x, y, z)$,  set $\varphi \gets \varphi \land (x' \lor y' \lor z')$.
    \end{enumerate}
    Clearly, this reduction takes time polynomial in the input size. So it suffices to show that this is a valid reduction.\nl
    \begin{claim}
        $\phi \in \texttt{3SAT} \implies \varphi \in \texttt{Strict3SAT}$.
    \end{claim}
    \begin{proof}
        For any clause $C$ in $\phi$, the corresponding clause in $\varphi$ is of the form $C$ or $C \lor x$ or $C \lor x \lor y$. Since $\phi$ is satisfiable, $C$ must evaluate to true, and hence
        since or-ing true with anything gives true, the corresponding clause in $\varphi$ must also be satisfiable. The only remaining clauses in $\varphi$ that need to be satisfied are those of form $x' \lor y' \lor z'$. But these clauses can be satisfied by selection $x, y, z = 0$ as each of these clause contains at least one of either $\overline x$ or $\overline y$ or $\overline z$. Hence we are done.
    \end{proof}
    \begin{claim}
        $\phi \in \texttt{3SAT} \impliedby \varphi \in \texttt{Strict3SAT}$.
    \end{claim}
    \begin{proof}
        Consider a satisfying assignment for literals in $\varphi$. First we claim that in any satisfying assignment the values of $x, y, z$ must be all $0$. Suppose not, and atleast one of them has value $1$. Then, one of the additional $2^3-1$ clauses will become false, hence this is a contradiction.\\
        
        Now, consider any clause $C$ in $\phi$, and let $C'$ be the corresponding clause in $\varphi$. Note that $C'$ evaluates to true.
        Then we have one of the following
        three cases:
        \begin{enumerate}
            \item $C = (a \lor b \lor c)$: in this case, since $C' = C$, $C$ evaluates to true as well.
            \item $C = (a \lor b)$: in this case, we have $C' = (a \lor b \lor x)$. Since $C'$ evaluates to true and $x$ is false as established before, at least one of $a, b$ are true, so $C$ evaluates to true as well.
            \item $C = (a)$: in this case, we have $C' = (a \lor x \lor y)$. Since $C'$ evaluate to true and both $x$ and $y$ are false, $a$ must be true, so $C$ evaluates to true as well.
        \end{enumerate}
        In all cases, $C$ evaluates to true, so each clause in $\phi$ evaluates to true, and thus $\phi$ is satisfiable using the same assignment.
    \end{proof}
    These two claims show that this is indeed a valid reduction.
\end{soln}


\newpage
\begin{prob}
We all know that the set of strings over the alphabet $\{a,b\}$ containing an equal number of occurrences of $ab$ and $ba$ is regular. However, what if the alphabet is $\{a,b,c\}$? Prove that the language
\[\{x\in\{a,b,c\}^*\text{ }|\text{ }x\text{ contains an equal number of occurrences of }ab\text{ and }ba\}\]
is not regular. Here are some hints.
\begin{enumerate}
\item Take help of the regular expression $(abc\cup bac)^*$.
\item Use closure under inverse homomorphisms from Homework 1.
\end{enumerate}
\end{prob}

\begin{soln}
Let $L_R$ be the language corresponding to the regular expression $R = (abc \cup bac)^*$. $L_R$ is, of course, regular because regular expressions only generate regular languages. Now, denote by $L$ the following language (which we need to show is not regular) -
\[
L = \{x\in\{a,b,c\}^*\text{ }|\text{ }x\text{ contains an equal number of occurrences of }ab\text{ and }ba\}
\]

Suppose $L$ is regular. Then this implies that the language $L' = L \cap L_R$ must also be regular as regular languages are closed under intersection operation. Therefore, to show that that $L$ is not regular it is sufficient to show that $L'$ is not regular.

\begin{claim}
$L'$ is not a regular language.
\end{claim}
\begin{proof}
By definition $L' = L \cap L_R$ therefore $L'$ can be written as
\[
    L' = \{ x \in L_R | \quad x \text{ has equal number of occurrences of }ab\text{ and }ba\}
\]
\[
    \implies L' = \{ x \in L_R | \quad x \text{ has equal number of occurrences of }abc\text{ and }bac\}
\]
If $L'$ is regular then it must satisfy the pumping lemma. But, consider the string $s = (abc)^p (bac)^p \in L'$ for any $p \in \mathbb{N}$.
\begin{claim}
    It is not possible to partition $s$ into 3 strings $x, y, z$ such that
    \begin{enumerate}
        \item $|y| > 0$, and
        \item $|xy| \leq p$, and
        \item $x y^i z \in L'$ for all $i \in \mathbb{N} \cup \{0\}$
    \end{enumerate}
\end{claim}
\begin{proof}
Consider any partition of $s = xyz$ such that conditions $(1)$ and $(2)$ hold. Then $xy$ will be a prefix of $s$ strictly smaller than the prefix $(abc)^p$. Now we make cases on the string $y$.
\begin{enumerate}
    \item $y$ is of form $(abc)^k$ for some $k > 0$. This implies $x$ must also be of form $(abc)^{k_1}$ for some choice of $k_1 \geq 0$, and $z$ will be of form $(abc)^{k_2} (bac)^p$ for some $k_2 > 0$. Hence the \emph{pumped down} string $x y^0 z$ will be of form $(abc)^{k_1} (abc)^{k_2} (bac)^p$ where $k_1 + k_2 < p$ as $k_1 + k_2 + k = p$ and $k > 0$. Note that this \emph{pumped down} string is in $L_R$ but not in the language $L'$.
    \item $y$ is not of form $(abc)^k$. Even in this case, the \emph{pumped down} string $x y^0 z$ will either be not in $L_R$ or it will not be in $L'$ because of removal of atleast 1 character from the string as we have that $xy$ is a prefix smaller than $(abc)^p$.
\end{enumerate}
Hence, in either case we have shown that the $(3)$ condition can never hold for all $i \in \mathbb{N}$.
\end{proof}
Since $L'$ does not satisfy the pumping lemma which is the necessary condition for a language to be regular, we have that $L'$ can not be regular. Hence, the proof is complete.
\end{proof}
Since $L'$ is not regular, $L$ can not be regular as well, by contradiction. Hence the solution is complete.
\end{soln}


\newpage
\begin{prob}
Prove that for any infinite regular language $L$, there exist two infinite regular languages $L_1,L_2$ such that $L=L_1\cup L_2$ and $L_1\cap L_2=\emptyset$. Here are some hints.
\begin{enumerate}
\item Let $D$ be any DFA and $q$ be any one of its states. Prove informally that the language
\[L_q=\{x\text{ }|\text{ }x\in\mathcal{L}(D)\text{ and the run of }D\text{ on }x\text{ visits }q\text{ an odd number of times}\}\]
is regular.
\item Recall the proof of the pumping lemma.
\end{enumerate}
\end{prob}

\begin{soln}
To start with, we can assume WLOG that both $D_1$ and $D_2$ are DFAs over the same alphabet $\Sigma$ (for if not, let $\Sigma = \Sigma_1 \bigcup \Sigma_2$ and add appropriate error state transitions in both the DFA).

Now, let the regular language recognized by the DFA $D_1$ and $D_2$ be $L_1$ and $L_2$ respectively. To design an algorithm which can recognize whether $L_1$ and $L_2$ are same we will use following bi-implication from set theory (note that $L_1, L_2 \subseteq \Sigma^*$)
\begin{claim}
    $$L_1 = L_2 \quad \Longleftrightarrow \quad L_1 \cap L_2^C = \emptyset \; \text{and} \; L_1^C \cap L_2 = \emptyset$$
    Where $L^C = \Sigma^* \setminus L$.
\end{claim}
\begin{proof}
The forward direction follows directly from the definition of set complement, so we shall focus our proof on showing the backwards implication. Specifically we shall prove
\[
L_1 \cap L_2^C = \emptyset \; \text{and} \; L_1^C \cap L_2 = \emptyset \quad \implies \quad L_1 = L_2
\]
To prove this we prove the following 2 statements.
\begin{enumerate}
    \item $L_1^C \cap L_2 = \emptyset \; \implies \; L_2 \subseteq L_1$. Proof by contradiction. Suppose there exists some $x \in L_2$ such that $x \notin L_1$. But $x \notin L_1 \implies x \in L_1^C$ which therefore implies that $x \in L_1^C \cap L_2$, which is a contradiction.
    \item $L_1 \cap L_2^C = \emptyset \; \implies \; L_1 \subseteq L_2$. Proof by contradiction. Suppose there exists some $x \in L_1$ such that $x \notin L_2$. But $x \notin L_2 \implies x \in L_2^C$ which therefore implies that $x \in L_1 \cap L_2^C$, which is a contradiction.
\end{enumerate}
Thus, from both the above statements we have $L_1 \subseteq L_2$ and $L_2 \subseteq L_1$ which implies $L_1 = L_2$ and hence the proof is complete.
\end{proof}

This bi-implication lays down the foundation of our algorithm. By closure properties of regular languages we know that they are closed under both complementation and intersection. Further, we already know how to derive DFA for these cases starting from the DFA(s) of the initial languages. Thus, we can obtain the DFAs for the languages $L' = L_1 \cap L_2^C$ and $L'' = L_1^C \cap L_2$. Let's call the DFAs $D'$ and $D''$.

\begin{align*}
    L_1 = L_2 &\iff L', L'' = \emptyset\\
              &\iff \L(D'), \L(D'') = \emptyset\\
              &\iff \text{none of } D', D'' \text{ accepts any string over } \Sigma
\end{align*}


For such DFAs, no accepting state can be reachable from the initial state through any sequence of transitions. Thus, to determine if $L_1 = L_2$ we construct the DFAs $D'$ and $D''$ and verify the set of reachable states from the initial state is disjoint with set of accepting states.\\

In the algorithm below we define the routine \textsc{CheckIfSame} as the required algorithm. It uses helper routines \textsc{ComplementDFA}, \textsc{IntersectionDFA} and \textsc{DFS}. The briefs of each routine is given alonside the code. The formal proof of correctness of these sub-routines is skipped as they have already been covered in the class.\\

\begin{algorithmic}[1]
\State \Comment{Helper functions for the main function below}
\Function{ComplementDFA}{$D$}
\Comment{Returns the DFA $D'$ such that $\mathcal{L}(D') = \mathcal{L}(D)^C$}
\State unpack $(Q, \Sigma, \delta, q_0, A) \gets D$
\State \Return $(Q, \Sigma, \delta, q_0, Q \setminus A)$
\EndFunction
\\
\Function{IntersectionDFA}{$D_1$, $D_2$}
\Comment{Returns the DFA $D'$ such that $\mathcal{L}(D') = \mathcal{L}(D_1) \cap \mathcal{L}(D_2)$}
\State unpack $(Q_1, \Sigma, \delta_1, q_{10}, A_1) \gets D_1$
\State unpack $(Q_2, \Sigma, \delta_2, q_{20}, A_2) \gets D_2$
\Function{$\delta$}{$(q_1, q_2), a$}
    \State \Return $(\delta_1(q_1, a), \delta_2(q_2, a))$
\EndFunction
%\State let $\delta : (Q_1 \times Q_2) \times \Sigma \rightarrow (Q_1 \times Q_2) \;$ be such that $\delta ((q_1, q_2), a) = (\delta_1(q_1, a), \delta_2(q_2, a))$
\State \Return $(Q_1 \times Q_2, \Sigma, \delta, (q_{10}, q_{20}), A_1 \times A_2)$
\EndFunction
\\
\Function{CheckIfSame}{$D_1, D_2$}
\State let $D_1^C \gets$ \Call{ComplementDFA}{$D_1$}
\State let $D_2^C \gets$ \Call{ComplementDFA}{$D_2$}
\State let $D' \gets$ \Call{IntersectionDFA}{$D_1, D_2^C$}
\State let $D'' \gets$ \Call{IntersectionDFA}{$D_1^C, D_2$}
\State unpack $(Q', \Sigma, \delta', q_0', A') \gets D'$
\State unpack $(Q'', \Sigma, \delta'', q_0', A'') \gets D''$
\State \Comment{Reduce DFAs to graphs}
\State let graph $G' \gets (Q', \{(q, q') \mid \exists a \in \Sigma \ni \delta'(q, a) = q'\})$
\State let graph $G'' \gets (Q'', \{(q, q') \mid \exists a \in \Sigma \ni \delta''(q, a) = q'\})$
\State \Comment{Assume DFS returns boolean array denoting reachability for each vertex}
\State let $\mathit{reachable}' [1 \ldots |Q'|] \gets \Call{DFS}{G', q_0'}$
\State let $\mathit{reachable}'' [1 \ldots |Q''|] \gets \Call{DFS}{G'', q_0''}$
\ForAll{$q \in A'$}
    \If{$\mathit{reachable}'[q]$}
        \State \Return False \Comment{An accepting state is reachable in $D'$ implies $L_1 \cap L_2^C \ne \emptyset$}
    \EndIf
\EndFor
\ForAll{$q \in A''$}
    \If{$\mathit{reachable}''[q]$}
        \State \Return False \Comment{An accepting state is reachable in $D''$ implies $L_1^C \cap L_2 \ne \emptyset$}
    \EndIf
\EndFor
\State \Return True
\Comment {No accepting state is reachable in either $D'$ or $D''$}
\EndFunction
\\
\Function{DFS}{$G(V, E), v_0$}
    \Comment{Standard iterative version of DFS algorithm}
    \State let $reachable \gets$ boolean array of size $|V|$ initialized to all False.
    \State let $stack \gets$ empty stack
    \State push $v_0$ into $stack$
    \State $reachable[v_0] \gets $ True
    \While{$stack$ is not empty}
        \State let $v \gets$ pop from $stack$
        \ForAll{$u : (v, u) \in E$}
            \If{not $reachable[u]$}
                \State $reachable[u] \gets$ True
                \State push $u$ into $stack$
            \EndIf
        \EndFor
    \EndWhile
    \State \Return reachable
\EndFunction
\end{algorithmic}
\end{soln}


\end{document}
